%=======================================================================================%
\chapter{Molecular Dynamics and Functional Characterization of I37R-CFTR Lasso Mutation Provide Insights into Channel Gating Activity}
\label{chap:I37R}
\section*{\centering Abstract} 
Characterization of I37R, a mutation located in the lasso motif of the CFTR chloride channel, was conducted by theratyping several CFTR modulators from both potentiator and corrector classes. Intestinal current measurements in rectal biopsies, forskolin-induced swelling (FIS) in intestinal organoids, and short circuit current measurements in organoid-derived monolayers from an individual with I37R/Fidel CFTR genotype demonstrated that the I37R-CFTR results in a residual function defect amenable to treatment with potentiators and type III, but not type I, correctors. Molecular dynamics of I37R using an extended model of the phosphorylated, ATP-bound human CFTR identified an altered lasso motif conformation which results in an unfavorable strengthening of the interactions between the lasso motif, the regulatory (R) domain, and the transmembrane domain 2 (TMD2). Structural and functional characterization of the I37R-CFTR mutation increases understanding of CFTR channel regulation and provides a potential pathway to expand drug access to CF patients with ultra-rare genotypes.

\section{Introduction}
Cystic fibrosis (CF) is a life-limiting genetic disease resulting from mutations in the CF transmembrane conductance regulator (CFTR) gene \cite{ratjen2015}. CFTR—the only member of the ABC transporter family known to be an ion channel—consists of two transmembrane domains (TMD1 and TMD2) which form an anion-selective pore, two highly conserved nucleotide-binding domains (NBD1 and NBD2) with ATP-binding pockets and a newly described N-terminal lasso motif \cite{hwang2013a, zhang2016a}. In addition, CFTR has a unique, disordered regulatory (R) domain which contains protein kinase A (PKA) phosphorylation sites. For the CFTR channel to open and close (gate), cAMP-dependent PKA phosphorylation of the R domain first activates the CFTR (\cite{gadsby1994}). Then, ATP-binding induces the dimerization of the two NBDs which opens the channel pore and ATP hydrolysis closes the pore.

The lasso motif (amino acids (aa) M1-L69), which is partially embedded in the bilayer and interacts with the R domain, was recently resolved following advancements in cryo-electron microscopy (cryo-EM) of the CFTR structure \cite{liu2017a, zhang2018a}. The first 40 amino acids of the lasso motif, which include lasso helix 1 (Lh1, aa V11–R29), form a circular “noose” structure (\cite{hoffman2018}). The noose structure wraps around the transmembrane helices (TM2, TM6 of TMD1 and TM10, TM11 of TMD2) and is held in place by hydrophobic interactions with L15, F16, F17, T20, L24, and Y28. The C-terminal end of the lasso, which includes the lasso helix 2 (Lh2, aa A46–L61), is tucked under the elbow helix (aa I70–R75) (Hoffmann et al., 2018). Variable disease severity and heterogeneous clinical presentation have been reported for the 78 CFTR variants identified so far in the lasso motif (CFTR1 and CFTR2 databases, Table S1). Evidently, the lasso motif has a multifunctional role in CFTR regulation with variants impacting folding, gating, and stability of the CFTR protein (Fu et al., 2001; Gené et al., 2008; Jurkuvenaite et al., 2006; Sabusap et al., 2021; Thelin et al., 2007).

CFTR modulators, small molecules which directly target CFTR dysfunction, are now available to certain individuals with CF. Currently, two classes are approved; (1) potentiators, which open the channel pore such as ivacaftor (VX-770) and (2) correctors, which assist CFTR protein folding and delivery to the cell membrane. Type I correctors (lumacaftor/VX-809, tezacaftor/VX-661) stabilize the NBD1-TMD1 and/or NBD1-TMD2 interface by binding directly to TMD1 (Loo et al., 2013; Ren et al., 2013) or NBD1 which improves the interaction between NBD1 and the intracellular loops (Hudson et al., 2017; Loo and Clarke, 2017). Type II correctors (C4) stabilize NBD2 and its interface with other CFTR domains while type III correctors (elexacaftor/VX-445) directly stabilize NBD1 (Okiyoneda et al., 2013). Combination therapies of corrector(s) and a potentiator (Orkambi®, Symdeko/Symkevi®, Trikafta/Kaftrio®) have been approved for CF individuals with F508del, the most common CFTR mutation, as well as several specific residual function mutations. Most recently, Trikafta/Kaftrio has been approved for patients with a single F508del mutation in combination with a minimal function mutation, broadening the population of patients with CF eligible for treatment with CFTR modulator therapy.

Mounting evidence has shown that in vitro functional studies in patient-derived cell models successfully predict clinical benefit of available CFTR modulators for individuals bearing ultra-rare mutations (Berkers et al., 2019; McCarthy et al., 2018; Ramalho et al., 2021). In individuals with CF, adult stem cells are usually collected by taking either airway brushings or rectal biopsies. Single Lgr5+ stem cells, derived from crypts within a patient's intestinal epithelium, can be expanded in culture medium and differentiated into organized multicellular structures complete with the donor patient's genetic mutation(s), thus representing the individual patient (Sato et al., 2009). Stem cell models can be used for personalized drug screening to theratype and characterize rare CFTR mutations (Awatade et al., 2018; Berkers et al., 2019; Pollard and Pollard, 2018). Determining the functional response of rare, uncharacterized CFTR mutations to modulator agents with known CFTR correction mechanisms enables characterization of CFTR structural defects and enhances our understanding of CFTR function.

I37R-CFTR is a novel missense mutation in the lasso motif, detected in an Australian male child diagnosed through newborn screening with elevated immunoreactive trypsinogen, raised sweat chloride (>60 mmol/L), and CFTR Sanger sequencing identifying c.1521-1523del (F508del) and c.110C > T (I37R) mutations (Table S2). We used functional studies and molecular dynamics (MD) simulations to characterize the functional and structural defects of I37R-CFTR. CFTR function was assessed using intestinal current measurements (ICM) in rectal biopsies, forskolin-induced swelling (FIS) assays in intestinal organoids, and short circuit current measurements (Isc) in I37R/F508del organoid-derived monolayers, respectively. The potentiators VX-770 (approved), GLPG1837 (phase II clinical trials), and genistein (a natural food component with potentiator activity (Dey et al., 2016)) were tested as monotherapies, dual potentiator therapies, or in combination with correctors (VX-809, VX-661, and VX-445). We compared this to our laboratory reference intestinal organoids. For MD simulations, we modeled and examined the structural defect of the I37R mutation on an extended cryo-EM structure of ATP-bound, phosphorylated human CFTR (PDB ID code 6MSM) (Zhang et al., 2018).
