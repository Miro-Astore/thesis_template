%=======================================================================================%
\pagenumbering{arabic}
\chapter{Introduction}
\setcounter{page}{1}
\label{chap:intro}

\section{Physics in a test tube}
\chapquote{}{}

\vskip 0.5cm

Biology distinguishes itself amongst scientific disciplines requiring the study of systems that are both complex and heterogeneous. In the study of more simple physical systems a simple analogy such as a mass on a spring or a gas of hard spheres can be extremely successful in explaining quantitative phenomena. For biological systems there appears to be too much complexity for such analogies to have the same level of success. There is limited applicability for such models. Find examples in the BIALEK lectures. There has been significant sucess . Although successful, in a certain sense idealised models are still descriptive rather than predictive. They may struggle to answer questions such as "If this gene mutates how will that affect this physiological parameter?" "If this drug were given at a higher dosage what would its effect be?" "What if we change this chemical moiety?" Such questions unfortunately require a lot of forethought and computaitonal investment to answer. 

It seems like a silly question but it seems important to ask why we can't just use a device similar to a harmonic oscilator or a perfect black body to speculate at useful answers for these quantitative questions. The answer is just as silly. If you look with your naked eye at your arm, you will notice hair, pores, dry skin, dead skin, perhaps even tendons and muscles under the the skin. If you take a miscroscope you will notice the 3 layers to your skin with different functions and composition. If you were to take a single cell from any of those layers and stain it to distinguish features in an electron miscroscope you would notice all sorts of complex structures and the size and number of these structures would vary depending on where you took the cell from in the body. Within and between each those structures is a salty, wet dance of moelcules large and small. This heterogeneity on length scales hints at the reasons behind biology's physical complexity. Plasma physics is often characterised by the density of the plasma studied. This parameter may span 28 orders of magnitude from a dense stellar core to the sparse intergallactice nebulae. The same mathematical tools can be used to map any plasma in these energy scales. Were we so lucky in biology that we could use the same physical laws to deal with phenomena accross a single order of magnitude.  


Thus, in order to move towards more predictive theories of biology it is necessary to consider much more of the fundamental physical processes occuring within biological systems than simply searching for statistical trends. One form of this from fundamentals approach is the simulation of every atom in a biological system. Although computationally expensive, this approach appears necessary due to the heterogenous nature of biological systems. 

 

%=======================================================================================%
