\definecolor{codegreen}{rgb}{0,0.6,0}
\definecolor{codegray}{rgb}{0.5,0.5,0.5}
\definecolor{codepurple}{rgb}{0.58,0,0.82}
\definecolor{codeorange}{rgb}{0.98,0.6,0.01}
\definecolor{codeorange2}{rgb}{1.0,0.5,0.0}
\definecolor{backcolour}{rgb}{0.95,0.95,0.92}

\lstdefinestyle{customTCL}{
    backgroundcolor=\color{backcolour},   
    commentstyle=\color{codegreen},
    keywordstyle=\color{blue},
    numberstyle=\tiny\color{codegray},
    stringstyle=\color{codeorange},
    basicstyle=\footnotesize,
    breakatwhitespace=false,
    language=tcl, 
    breaklines=true,                 
    captionpos=b,                    
    keepspaces=true,                 
    numbers=left,                    
    numbersep=5pt,                  
    showspaces=false,                
    showstringspaces=false,
    showtabs=false,                  
    tabsize=4
}

\lstdefinestyle{customF90}{
    backgroundcolor=\color{backcolour},   
    commentstyle=\color{codegreen},
    keywordstyle=\color{blue},
    numberstyle=\tiny\color{codegray},
    stringstyle=\color{codeorange},
    basicstyle=\footnotesize,
    breakatwhitespace=false,
    language=fortran, 
    morekeywords={*,getarg},
    breaklines=true,                 
    captionpos=b,                    
    keepspaces=true,                 
    numbers=left,                    
    numbersep=5pt,                  
    showspaces=false,                
    showstringspaces=false,
    showtabs=false,                  
    tabsize=4
}

\lstdefinestyle{customPython}{
    backgroundcolor=\color{backcolour},   
    commentstyle=\color{blue},
    keywordstyle=\color{codeorange2},
    numberstyle=\tiny\color{codegray},
    stringstyle=\color{codegreen},
    basicstyle=\footnotesize,
    breakatwhitespace=false,
    language=Python, 
    breaklines=true,                 
    captionpos=b,                    
    keepspaces=true,                 
    numbers=left,                    
    numbersep=5pt,                  
    showspaces=false,                
    showstringspaces=false,
    showtabs=false,                  
    tabsize=4
}

\newpage
%\addcontentsline{toc}{chapter}{\hyperref[apx:charges]{Appendix}}
\appendix


\chapter{Ion Self-Energy Python Script}
\label{apx:charges}
The self-energy of ions in Eq.~\eqref{eq:mse} is implemented in \verb+Python+ using the \verb+MDAnalysis+ 
module to read in the MD trajectory~\cite{Hashmi2015,Gowers2016}. Since \verb+MDAnalysis+ can load 
trajectories written by a number of MD programs, the code below can be used with any of the programs 
\verb+MDAnalysis+ supports. The scripts used to run the solvation free energy calculations in OpenMM 
are uploaded in GitHub (\url{https://github.com/jeff231li/solvation-openmm}).

\vspace{0.5cm}
\verb+* self-energy.py+
\lstinputlisting[style=customPython]{Codes/image-charges.py}

\chapter{Funnel Potential tclForces Script}
\label{apx:funnel}
The source code below is a Tcl implementation of the funnel potential that was invented by Parrinello 
and co-workers for use with metadynamics~\cite{Limongelli2013}. NAMD provides a high-level Tcl 
scripting interface for user-defined potentials without the need to recompile the program. 

In the NAMD configuration file, the section below defines the funnel potential parameters. The script 
is written so that the user can define the endpoints \textbf{A} and \textbf{B} arbitrarily so that the 
potential is not limited to the three Cartesian axes. The atoms involved in the restraint is chosen 
by reading a PDB file with columns \verb+beta+ or \verb+occupancy+ filled with the given user value 
(similar interface to the \verb+colvar+ module). The script also prints the positions of the molecule 
along the $r_{\text{xy}}$ and $r_{\text{z}}$ vectors. This code is used in \chapref{chap:unbind} for 
the Na1\dprim\ $\rightarrow$ Bulk transition. These scripts are available on GitHub (\url{https://github.com/jeff231li/funnel_potential}).

\vspace{0.5cm}
\verb+* In NAMD config file: +
\lstinputlisting[style=customTCL]{Codes/namd-funnel.tcl}
\vspace{0.5cm}
\verb+* funnel.tcl+
\lstinputlisting[style=customTCL]{Codes/funnel_potential.tcl}

\chapter{Position-Dependent Diffusion Coefficient}
\label{apx:diff}

\chapquote{``FORTRAN's tragic fate has been its wide acceptance, mentally chaining thousands and thousands of programmers to our past mistakes."}{Edsger W. Dijkstra}

\vskip 0.5cm

The position-dependent diffusion coefficient for a system restrained in a harmonic potential was first 
derived by Woolf and Roux~\cite{Woolf1994}, which requires a Laplace transformation of the velocity 
autocorrelation. Hummer simplified the procedure by using the position autocorrelation 
instead~\cite{Hummer2005}. The code below is a Fortran implementation for calculating the diffusion 
coefficient using the method derived by Hummer. This is used to calculate the diffusion coefficient in 
\chapref{chap:unbind}. This code is available on GitHub (\url{https://github.com/jeff231li/position_diffusion}).

\vspace{0.5cm}
\verb+* x_acf.f90+
\lstinputlisting[style=customF90]{Codes/diffusion.f90}

