%=======================================================================================%
\chapter{Review of the Cystic Fibrosis Transmembrane Conductance Regulator}
\label{chap:cftr_review}
\newpage
%Authors note:
% We begin with a breif overview of the disease Cystic Fibrosis as it is the main motiation for this project. A horrendous disease for which we will soon find a cure.
% The purpose of this chapter is to present an overview of CFTR's structure and mechanism of action as discovered by a combination of structural and physiological studies. In addition to the actions of CFTR modulators.
\section{Clinical outcomes of Cystic Fibrosis}
Cystic Fibrosis (CF) is the most common fatal genetic condition in caucasian populations. 90 000 people are afflicted globally. Even with decades of research there is no known cure for CF. With the average life expectancy of patients falling below 50 even in countries with developed health care systems such as the USA and Australia\cite{}\cite{}. The cause is from a build up of salts inside epithelial cells. This causes the surface of the epithelium to dehydrate. When dehydrated the clilia on the epithelium collapse leaving them unable to clear the mucus that naturally lines the airway\cite{boucher2006}. The dehydration mentioned earlier causes the mucus to thicken. This buildup has two pathogenic functions. Firstly it inhibits the normal function of the organ, as mucus fills ducts that would normally pass nutrients in the pancreas or absorb gasses in the lungs. Secondly, the stationary mucus allows bacterial infection, this can further degrade lung function and remains one of the most troublesome chronic complications in CF patients. 

Much of the clinical research into CF has been managing the movement of this mucus and the populations of bacterium in it. Patients often require to hours of physical therapy to help clear this mucus since their lungs are unable to. They must also inhale saline solutions in order to counteract the osmotic pressure in their epithelium. This helps draw more moisture out of the epithelial cells to allow the cilia to move. 

CF patients struggle to intake nutrients due to the misfunction of their pancreas and large intestines. This leads to CF related diabetes which afflicts roughly half of adults with CF \cite{kayani2018}. Meaning CF patients have to consume an array of enzymes and adhere to a specific diet, in addition to which they must be careful not to consume foods which may interfere with the array of medications they may be on.

\section{ABC Transporters}
CFTR belongs to a super family of proteins known as ATP Binding Cassette Transporters,  many of these proteins perform active transport across the cell membrane of different substrates including lipids and drug molecules. They are so named because they bind ATP in catalytic sub units on the protein called Nucleotide Binding Domains. These domains act as ATPases, accelerating the hydrolysis of ATP. The energy from hydrolysis is then transferred into the protein in order for it to pump its substrate against a concentration gradient. 

\section{CFTR classification and structure}

The primary cause of the disease Cystic Fibrosis (CF) is the misfunction of a chloride channel, the Cystic Fibrosis Transmembrane Conductance Regulator (CFTR). This ion channel is a member of the ABCC subfamily of ABC transporters, designated ABCC7. This channel is unique amongst this family because it is not generally considered an active transporter but something of a low conductivity channel or a "weak pump"\cite{Linsdell2018}.

CFTR is distinguished by a regulatory region known as the R-domain (residues 645-845) which links NBD1 to TMD2. This region acts to lock the channel in the closed state by wedging itself between the TMDs and dislodging when any one of 3 sites are phosphorylated \cite{Mihalyi2020}. In experimentally determined structures of human CFTR the secondary structure of a section of the R-domain but not at high enough resolution to determine the identity of individual sidechains \cite{Zhang2018}\cite{Zhang2016}. Further secondary structure information can be found through experiments with NMR \cite{Baker2007}.

Previous computional studies of CFTR have been used homology models based on the phosphorylated zebra fish protein PDBID:5W81 \cite{Zhang2017}. This differs substantially from the human form of the channel with a significant rearrangement of the helices in the channel (go through and actually figure out what these are). These have yielded interesting results but the sequence similarity between human and zebrafish CFTR is only 55\% \cite{}. For a protein structure where a single amino acid mutation leads to misfunction, more precision can only help. In fact the activity of CF treating drugs is not well conserved in the zebra fish structure. In order to do precision medicine we need precision structures. 

An open state of the channel has been proposed by combining both the zebra fish homology model and the fully outward facing conformer of a bacterial ABC transporter Sav1866 \cite{Hoffman2018}. Although this model has several characteristics expected of the open channel, such as the critical R352-D993 salt bridge, it lacks the R104-E116 salt bride. In experiments, tehse residues could be replaced by cysteines and the channel would still function. However, when reducing agents were added to the system the channel lost its ability to open fully. This indicates that in the oxidised environment the C104-C116 cysteins formed a disulfide bridge but its breaking upon exposure to reducing agents caused a loss of function in the channel. This indicates that in the WT channel R104-E116 form a stable salt bridge. 

This salt bridge is clearly visible in the recent cryo-EM structure of ATP-bound human CFTR \cite{Zhang2018}.

\section{The Gating Cycle}
The conformational transition from inactive to active differs significantly in CFTR compared to other ABC transporters. The NBD domains are largely similar to other to those found in other ABC transporters, they dimerise in what is termed a head to tail configuration so both subunits contact both bound ATP molecules \cite{}. Residue E1371 allows neucleophilic attack on the $\gamma$ phosphate of the ATP bound to Walker B \cite{Stratford2007}. 

The NBD 

\section{The perturbations of TM8 and its Ability to Bind Drugs}
A strange feature of the human CFTR structure is the unfolded helix TM8. This helix unwinds in the middle of the bilayer, this feature is conserved in both the Zebra fish structure and the human structure of CFTR \cite{Zhang2017}\cite{Zhang2018}. There is a significant conformational change between the open and closed CFTR channels with this region with the top of TM8 swinging 55$^o$ during opening. In humans the L927P mutation is known to cause disease. It is hypothesised that this mutation impedes this motion.



\section{Lipid Interactions with CFTR}
CF afflicted cells have a perturbed lipidome compared to healthy cells.\cite{Cottrill2020} Thus it is important to understand lipid interactions with the CFTR channel itself.
