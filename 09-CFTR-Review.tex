%=======================================================================================%
\chapter{Review of the Cystic Fibrosis Transmembrane Conductance Regulator}
\label{chap:cftr_review}
\newpage
\section{Introduction}

The primary cause of the disease Cystic Fibrosis (CF) is the misfunction of a chloride channel, the Cystic Fibrosis Transmembrane Conductance Regulator (CFTR). This ion channel is a member of the super family, the ATP Binding Cassette Transporters (ABC transporters). This channel is unique amongst this family because it is not generally considered an active transporter but something of a low conductivity channel or a "weak pump".

The ABC family distinguishes itself by two conserved Nucleotide Binding Domains (NBD's) which bind Adenosine Triphosphate (ATP) and use the energyfrom hydrolysis to induce a conformational change which enables the transporter to move from an inward facing to an outward facing conformation. In the case of most ABC transporters this induces the transport of an array of cellular products, such as lipids, or other nutrients. They are characterised by a the mentioned NBDs as well as transmebrane domains which directly mediate the passage of the substrate across the membrane. This family of proteins are the subject of multiple investigations because of their range of importance. In particular much work has been done on the dynamics of P-glycoprotein due to its affinity for many small molecule drugs. \cite{OMara2012}


