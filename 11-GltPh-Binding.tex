%=======================================================================================%
\chapter{Mechanism of Ligand Binding in \GltPh}
\label{chap:bind}

ABSTRACT \newline
Glutamate transporters clear up excess extracellular glutamate by co-transporting three \Na\ and one 
\Hi\ with the counter-transport of one \K. The archaeal homologs are selective to aspartate and only 
co-transport three \Na. The crystal structures of \GltPh\ from archaea have been used in 
computational studies to understand the transport mechanism. While some progress has been made with 
regard to the ligand-binding sites, a consistent picture of transport still eludes us. A major 
concern is the discrepancy between the computed binding free energies, which predict high-affinity 
\Na-- low-affinity aspartate binding, and the experimental results where the opposite is observed. 
Here we show that the binding of the first two \Na\ ions involve an intermediate state near the Na1 
site, where two \Na\ ions coexist and couple to aspartate with similar strengths boosting its 
affinity. Binding free energies for \Na\ and aspartate obtained using this intermediate state are in 
good agreement with the experimental values. Thus the paradox in binding affinities arises from the 
assumption that the ligands bind to the sites observed in the crystal structure following the order 
dictated by their binding free energies with no intermediate states. In fact, the presence of an 
intermediate state eliminates such a correlation between the binding free energies and the binding 
order. The intermediate state also the facilitates transition of the first \Na\ ion to its final binding 
site via a knock-on mechanism, which induces substantial conformational changes in the protein 
consistent with experimental observations.

\newpage

\section{Introduction}
\label{bind:intro}
Glutamate transporters (Glts) are membrane proteins responsible for removing excess extracellular 
glutamate in the synaptic cleft. The transport highly depends on the electrochemical gradient of 
\Na\ ions~\cite{Danbolt2001}. In mammals, Glts are called the excitatory amino-acid transporters 
(EAATs). In one cycle, they co-transport three \Na\ and a \Hi\ with glutamate and counter-transport 
one \K~\cite{Vandenberg2013}. Failed regulation of Glts can lead to cell death due to overexcitation 
of receptors (excitotoxicity), which has been implicated in neurological diseases such as Alzheimer’s 
and amyotrophic lateral sclerosis~\cite{Danbolt2001}. As in ion channels, the first crystal structure 
of Glts was resolved in prokaryotes, namely, \GltPh\ from the archaeal homologue \textit{Pyrococcus 
horikoshii}~\cite{Yernool2004,Boudker2007}. Although \GltPh\ has only 36\% amino-acid sequence identity 
with EAATs overall, the sequence identity is much higher in the binding pocket, and they share some 
common properties. As in EAATs, \GltPh\ requires coupling of three \Na\ ions but does not require the 
co-transport of \Hi\ and the counter-transport of \K~\cite{Groeneveld2010}. In addition, \GltPh\ is 
selective for Asp over Glu, which is not seen EAATs~\cite{Vandenberg2013}. 

In the initial crystal structures of \GltPh, the binding sites for two \Na\ ions (dubbed Na1 and 
Na2) and Asp were resolved but not the Na3 site for the third \Na\ ion. It is likely that the 
substitution of \Tl\ ions for \Na\ to obtain stronger signals has prevented the larger \Tl\ ions 
from entering the Na3 binding site. Several computational studies have been performed to determine 
the third sodium binding site labelled as Na3~\cite{Holley2009,Larsson2010,Tao2010,Huang2010,Bastug2012}. 
In the recent crystal structure of \GltTk~\cite{Guskov2016}, all the \Na\ binding sites were resolved 
with \Na\ instead of \Tl, which confirmed the Na3 site predicted from molecular dynamics (MD) simulations 
and mutagenesis experiments~\cite{Bastug2012}. In addition to the Na3 binding site, the various steps 
in ligand binding in the outward-facing conformation~\cite{Shrivastava2008,Huang2008,Heinzelmann2011,Grazioso2012}, 
and release in the inward-facing conformation~\cite{DeChancie2011a,Zomot2013,Heinzelmann2013,Heinzelmann2014a} 
have also been investigated with MD simulations (see Ref.~\cite{Setiadi2015} for a recent review). In 
general, MD simulations provide a satisfactory description of the substrate and \Na\ binding sites. A 
remaining discrepancy about the coordination of \Na\ at the Na2 site~\cite{Venkatesan2015,Heinzelmann2011} 
has been recently clarified using the \GltTk\ structure \cite{Guskov2016} as a guide, which indicated that 
the S atom of a methionine (M311) coordinated \Na. The discrepancy arose from the undercharging of S in the 
CHARMM force field, and once it was properly charged, agreement with the experimental coordination shell 
and the binding free energy of \Na\ at the Na2 site was restored~\cite{Setiadi2017}. 

In contrast, the situation with regard to the binding free energies of the other ligands has 
remained baffling. The initial free energy calculations of ligand binding in \GltPh\ gave the order 
of binding as Na3 ($-18.7$~kcal/mol), Na1 ($-7.1$~kcal/mol), and Asp ($-3.8$~kcal/mol), following 
the order of the binding free energies quoted in parenthesis~\cite{Heinzelmann2011}. Similar results 
were obtained for the release of these ligands in the inward-facing conformation~\cite{Heinzelmann2013}. 
Thus, two \Na\ ions were predicted to bind before Asp, consistent with the earlier observations 
\cite{Vandenberg2013,Tao2010}. However, in subsequent experiments, almost the opposite results were 
obtained for the binding free energies with \Na\ binding with low affinity and Asp with high 
affinity~\cite{Reyes2013b,Ewers2013,Hanelt2015}. Given that the free energy calculations performed using 
crystal structures usually retain chemical accuracy, the size of the discrepancy is rather unsettling 
and has compelled us to look beyond the static picture of binding provided by the crystal structure. 
For guidance, it is best to focus on Asp as there is a unique and well-defined binding site. We know 
that Asp does not bind in the absence of \Na\ \cite{Vandenberg2013}. Free energy calculations indicate that 
Asp can bind with one \Na\ when it is at Na1 but not when it is at Na3~\cite{Heinzelmann2013,Heinzelmann2011}. 
That is, \Na\ is strongly coupled to Asp at the Na1 site but not at the Na3 site. This suggests that the 
Asp affinity can be significantly boosted if there is an intermediate site for the first \Na\ in the 
vicinity of the Na1 site (dubbed Na1\prim), and two \Na\ ions can coexist at the Na1 and Na1\prim\ sites. 
The Coulomb repulsion between the two \Na\ ions in close proximity would then naturally explain the low 
affinity and slow binding of the second \Na. A strong candidate for such a Na1\prim\ site is the one 
found by Huang and Tajkhorshid while searching for the Na3 site, where \Na\ is coordinated simultaneously 
by the D312 and D405 side chains~\cite{Huang2010}. 

Here we investigate the feasibility of such a scenario by performing MD simulations and free energy 
calculations, where the Na1\prim\ site is included as an intermediate state during the ligand-binding 
process. We show that Na1\prim\ provides the deepest binding site so it will trap the first 
\Na\ and the second \Na\ can still bind to the Na1 site while Na1\prim\ is occupied. We then probe 
Asp binding to \GltPh\ for different ion configurations and draw inferences to resolve the discrepancy 
between the computed and experimental affinities.

\section{Method}
\label{bind:method}
\subsection{Model System and Simulation Details}
The crystal structure of the outward-facing conformation of \GltPh\ with an Asp and two \Na\ is used 
in MD simulations (PDB ID: 2NWX)~\cite{Boudker2007}. The third \Na\ is placed at the Na3 site
as determined in Ref.~\cite{Bastug2012} and the crystal structure of \GltTk~\cite{Guskov2016}. 
The simulation system is prepared using the  VMD software package~\cite{Humphrey1996}. First, the 
trimer structure is embedded in a 1-palmitoyl-2-oleoyl-sn-glycero-3-phosphatidylethanolamine (POPE) 
phospholipid bilayer. The system is then solvated in a box of TIP3P water molecules~\cite{Jorgensen1983,Neria1996} 
with 0.15~M NaCl and neutralised with extra Cl$^-$ ions. The final system contains a total of 
$\sim$100,000 atoms in a simulation box of size $110 \times 110 \times 100$ \AA$^3$. All MD simulations 
are performed using the NAMD program (version 2.12)~\cite{Phillips2005} together with the CHARMM36 force 
field~\cite{Klauda2012}, where the NBFIX correction for \Na\ interactions is the default option. The 
default (pH 7) values are used for the charge states of the amino acids. The temperature is kept constant 
at 300~K using Langevin dynamics with a damping coefficient of 1~ps$^{-1}$. The pressure of the system is 
maintained at 1~atm using the Langevin piston method~\cite{Feller1996} with a coefficient of 20~ps$^{-1}$. 
Short-range Lennard-Jones (LJ) interactions are truncated at 12~\angs\ and a smooth switching function is 
used starting at 10~\angs. Periodic boundary conditions are employed, and the long-range interactions are 
calculated with the particle-mesh Ewald method~\cite{Darden1993}. A time step of 2~fs is used in all MD 
simulations.

\subsection{Free Energy Calculations}
The standard binding free energy of an ion to a protein is expressed as
\begin{equation}
    \Delta G_{\text{b}}^{0} = \Delta\Delta G_{\text{int}} + \Delta\Delta G_{\text{tr}}^{0} +
    \Delta\Delta 
G_{\text{c,P}} ,
\end{equation}
where the first term is the free energy change in translocating the ion from bulk water
to the binding site, and the second term represents the loss of translational entropy during this
process. The last term is used when the protein residues coordinating the ligand are restrained 
to improve convergence in the calculation of $\Delta G_{\text{int}}$.
If no restraints are applied, $\Delta\Delta G_{\text{c,P}}=0$.
Assuming a Gaussian distribution for the ion positions in the binding site, the translational 
free energy difference is estimated using~\cite{Carlsson2005}
\begin{equation}
    \Delta\Delta G_{\text{tr}}^{0} = -k_{\text{B}}T\, \text{ln} \left[\frac{(2\pi 
e)^{3/2}\sigma_{x}\sigma_{y}\sigma_{z}}{V_{0}} \right],
    \label{eq:tr}
\end{equation}
where $V_{0}=1661\, \angs^3$ is the reference volume for the standard concentration of 1~M, 
$e$ is Euler's number, and the $\sigma$'s are the principal root mean square fluctuations of the ion 
positions. The restraint-free energy is calculated using the thermodynamic integration 
(TI) approach of Cecchini et al.~\cite{Cecchini2009}
\begin{equation}
    \Delta G_{\text{c}} = \frac{1}{2} \int_{0}^{k} \langle |\pmb X- \pmb X_{0}|^2 \rangle_{k}\, dk.
\end{equation}
Here, $|\pmb X- \pmb X_{0}|^2$ is the mean squared displacement of the 
restrained residues from a reference structure at a given spring constant $k$. We find 
that a value of 1.0~\spring\ per atom is sufficient to reduce the fluctuations of the 
coordinating residues. The TI calculations are performed using 11 windows, and each window is 
simulated 200~ps for equilibration and 800~ps for production. To obtain $\Delta\Delta 
G_{\text{c,P}}$, two separate calculations are performed, one with the ion at the 
binding site and another with the ion in bulk, i.e., $\Delta\Delta G_{\text{c,P}} = \Delta
G_{\text{c,P}}^{\rm ion-bulk} - \Delta G_{\text{c,P}}^{\rm ion-site}$.

The free energy of translocation is calculated using both free energy perturbation (FEP) 
and TI~\cite{Chipot2007}. The FEP calculation is performed first, and the trajectory is used as the starting 
point for the TI calculations. As in the restraint-free energy, the free energy change between 
the states with the ion at the site and in bulk are calculated, i.e., $\Delta\Delta G_{\text{int}} 
= \Delta G_{\text{int}}^{\rm ion-bulk}-\Delta G_{\text{int}}^{\rm ion-site}$.
However, only one calculation is needed if we create and destroy the ion simultaneously in the same 
system. For the FEP calculations, we use 66 exponentially spaced $\lambda$ values as this was 
shown in previous studies to be adequate for charged molecules~\cite{Heinzelmann2011}. At each $\lambda$ value, 
we perform 50~ps of equilibration, followed by a further 50~ps of production run. In the TI method, 
the ensemble of the free energy derivative is calculated at each $\lambda$ value. The free energy 
is recovered by integrating the free energy derivative with respect to the $\lambda$ values
\begin{equation}
    \Delta G_{\text{int}} = \int_{0}^{1} \left\langle \frac{\partial U(\lambda)}{\partial \lambda} 
\right\rangle_{\lambda} d\lambda .
\end{equation}
For charged molecules, the integral can be done using Gaussian quadrature. A seven-point 
quadrature is used as this was shown to be adequate in a previous study~\cite{Bastug2007}. The starting 
point for each TI window is obtained from the corresponding FEP calculation, and each window is run 
for a total of 1~ns with the first 200~ps discarded as the equilibration phase. The convergence of 
the TI calculations for the ions is shown in \figref{bind:figS1}.

For the substrate, the standard binding free energy is expressed as
\begin{equation}
    \Delta G_{\text{b}}^{0} = \Delta\Delta G_{\text{int}} + \Delta\Delta G_{\text{tr}}^{0} +
    \Delta\Delta 
G_{\text{rot}} + \Delta\Delta G_{\text{c,L}} ,
\end{equation}
which includes the free energy loss due to rotational entropy $\Delta\Delta G_{\text{rot}}$,
and the conformational restraints are applied on the substrate instead of the protein as indicated 
in the last term, $\Delta\Delta G_{\text{c,L}}$. 
The rotational free energy difference is calculated in a similar manner to Eq.~\eqref{eq:tr}~\cite{Carlsson2005}
\begin{equation}
    \Delta\Delta G_{\text{rot}} = -k_{\text{B}}T\, \text{ln} \left[\frac{(2\pi 
e)^{3/2}\sigma_{\phi_{1}}\sigma_{\phi_{2}}\sigma_{\phi_{3}}}{8\pi^2} \right] .
    \label{eq:rot}
\end{equation}
Here $\sigma_{\phi}$'s are the rotational root mean square fluctuations of the substrate
at the binding site calculated using the quaternion representation. For the conformational restraint, 
a spring constant of 5.0~\spring\ is applied to the heavy atoms of the substrate. The bulk 
calculation is performed separately with the substrate placed in a 25~\angs\ cubic water box with 
a neutralising ion. As in previous calculations~\cite{Heinzelmann2013,Heinzelmann2011}, the 
translocation free energy is split into three stages to improve convergence
\begin{equation}
    \Delta\Delta G_{\text{int}} = \Delta\Delta G_{\text{elec}} + \Delta\Delta G_{\text{LJ--bb}} +
    \Delta\Delta G_{\text{LJ--sc}}.
\end{equation}
The first term on the right-hand side is the electrostatic contribution, and the last two terms
are the LJ contributions. The LJ term is split into the backbone (LJ--bb) and the side chain (LJ--sc) 
parts in order to prevent a water molecule getting trapped behind the substrate in the binding site 
as noted previously~\cite{Heinzelmann2011}. For the electrostatic free energies, we report the values 
obtained with the TI method with 0.5/1.0~ns for equilibration/production. For the LJ free energies, we 
report the FEP values instead due to the lack of convergence with quadrature windows. A total of 52 and 
37 $\lambda$ values are used for the LJ--bb and LJ--sc calculations, respectively. For each window, we 
run a 100/100~ps for equilibration/production (see \figref{bind:figS2} for the convergence of the forward 
and backward transformation for both electrostatic and LJ free energies). A soft-core potential is 
used in the LJ calculations with a shifting coefficient of 6.0~\cite{Zacharias1994}. 

Umbrella sampling simulations are performed for the Na1$^\prime\rightarrow$ Na3 transition of \Na\ 
to find the potential of mean force (PMF) and study the conformational changes caused by this transition.
The reaction coordinate is chosen along the vector from the Na1\prim\ site to the Na3 site, and umbrella 
windows are generated at 0.5~\AA\ intervals using $k=10$~kcal/mol/\AA$^2$ for the harmonic biasing potential. 
Overlaps between the samples of neighbouring windows are found to be $>10\%$, which assures a robust 
construction of the PMF using the weighted histogram analysis method. Each window is simulated for 8~ns.
Convergence of the PMF up to the barrier is obtained after 3~ns, so the final PMF is constructed from the 
last 5~ns.

\section{Results and Discussion}
\subsection{Characterisation of the \Na\ Binding Sites}
Here we use the 2NWX structure for convenience, but it could be argued that the recent apo \GltPh\ 
crystal structure 4OYE \cite{Verdon2014} may provide a more appropriate starting point for the 
binding free energy calculations. Thus, it is prudent to show that there is no difference between 
the ligand-free 2NWX and 4OYE systems once they are equilibrated. To this end, we have performed 
60~ns MD simulations of the apo-2NWX and 4OYE systems under the same conditions. Superposition of 
the final snapshots of the two systems shows that there is a good overlap between the two structures 
(\figrefn{bind:figS3}{A}). For a more quantitative analysis, we compare the time series of the RMSDs 
of the two systems relative to 4OYE (\figrefn{bind:figS3}{B}). The apo-2NWX system is seen to relax 
to that of 4OYE within 25~ns. After the relaxation, the average RMSDs of the apo-2NWX and 4OYE systems 
relative to 4OYE are 1.09~\AA\ and 1.06~\AA, respectively. Thus either crystal structure can be used 
in binding free-energy calculations as long as the simulation system is equilibrated appropriately.

Starting with the fully bound equilibrated structure from a previous study~\cite{Setiadi2017}, 
we remove Asp and \Na\ at Na2. Following a 50~ns MD simulation, \Na\ ions at Na1 and Na3 are 
stable with a water molecule coordinating the ion at Na1. The binding modes listed in 
\tabref{bind:tab1} (columns 3 and 4) are in good agreement with those observed in the crystal 
structures~\cite{Boudker2007,Boudker2007}. We note that the \Na\ coordination at the Na3 site 
is slightly different when no ion is present at Na1 (\tabref{bind:tab1}, column 2). The ion 
moves away from the S93 residue but is more tightly coordinated by both O atoms of D312. To 
generate the intermediate site Na1\prim, we remove the \Na\ ion from Na3 and perform MD simulations 
of the system for another 50~ns. As seen in earlier MD simulations of \GltPh\ without a \Na\ at 
Na3~\cite{Huang2010,Bastug2012}, the side chains of N310 and D312 swing towards the Na1 site and 
start coordinating the \Na\ ion there at a somewhat different site called Na1\prim\ 
(\figrefi{bind:fig1}{A}). To achieve the coordination by N310 and D312, the \Na\ ion moves away 
from G306 and N310 by about 2~\angs. Comparing the binding mode of the Na1\prim\ site with that of 
Na1 (Na3) (\tabref{bind:tab1}, columns 5 and 4) , they appear qualitatively similar. However, having 
two Asp side chains instead of one, confers a distinct advantage to the Na1\prim\ site as will be 
demonstrated with the binding free energies in the next section. 

\begin{table}[tb]
\begin{center}
        \caption{Binding modes of the \Na\ ions in \GltPh\ for various states of occupation.$^{a}$}
        \label{bind:tab1}
        \resizebox{\textwidth}{!}{\begin{tabular}{lcccccc}
        \hline
        Helix-Residue & Na3 & Na3 (Na1) & Na1 (Na3) & Na1$^\prime$ & Na1$^\prime$ (Na1) & Na1 (Na1$^\prime$) \\ \hline
        TM3-Y89 (O)           & 2.6 $\pm$ 0.2 & 2.4 $\pm$ 0.2 & -             & -             & -             & -             \\
        TM3-T92 (OH)          & 2.4 $\pm$ 0.1 & 2.4 $\pm$ 0.1 & -             & -             & -             & -             \\
        TM3-S93 (OH)          & 3.4 $\pm$ 0.3 & 2.6 $\pm$ 0.3 & -             & -             & -             & -             \\
        TM7-G306 (O)          & -             & -             & 2.5 $\pm$ 0.2 & -             & -             & 2.3 $\pm$ 0.1 \\
        TM7-N310 (O)          & -             & -             & 2.4 $\pm$ 0.1 & -             & -             & 3.7 $\pm$ 0.2 \\
        TM7-N310 (O$_\delta$) & 2.3 $\pm$ 0.1 & 2.2 $\pm$ 0.1 & -             & 2.4 $\pm$ 0.1 & 2.3 $\pm$ 0.1 & -             \\
        TM7-D312 (O$_1$)      & 2.2 $\pm$ 0.1 & 2.4 $\pm$ 0.5 & -             & 2.4 $\pm$ 0.2 & 2.4 $\pm$ 0.1 & -             \\
        TM7-D312 (O$_2$)      & 2.3 $\pm$ 0.1 & 2.9 $\pm$ 0.6 & -             & 2.7 $\pm$ 0.3 & 2.7 $\pm$ 0.3 & -             \\
        TM8-N401 (O)          & -             & -             & 2.6 $\pm$ 0.3 & 2.6 $\pm$ 0.2 & 2.7 $\pm$ 0.2 & 3.5 $\pm$ 0.3 \\
        TM8-D405 (O$_1$)      & -             & -             & 2.3 $\pm$ 0.1 & 2.4 $\pm$ 0.1 & 2.4 $\pm$ 0.1 & 2.5 $\pm$ 0.2 \\
        TM8-D405 (O$_2$)      & -             & -             & 2.2 $\pm$ 0.1 & 3.8 $\pm$ 0.3 & 4.4 $\pm$ 0.1 & 2.3 $\pm$ 0.1 \\
        H2O (1)               & -             & -             & 2.3 $\pm$ 0.1 & 2.5 $\pm$ 0.2 & -             & 2.4 $\pm$ 0.1 \\
        H2O (2)               & -             & -             & -             & -             & -             & 2.5 $\pm$ 0.4 \\
        H2O (3)               & -             & -             & -             & -             & -             & 2.4 $\pm$ 0.1 \\ \hline
        \end{tabular}}
        \end{center}
        {\footnotesize $^{a}$Average distances (in \angs) of protein atoms coordinating \Na\ ions. 
        Other ions present are indicated in parenthesis at the top row. The average distances and 
        errors are estimated from a 5~ns unrestrained MD simulation.}
\end{table}

\begin{figure}[t!]
\centering
\includegraphics[width=1.0\linewidth]{Figures/Binding/fig1.jpg}
\caption{(\textit{Left column}) Coordination of the first two \Na\ ions in \GltPh: 
         (\textit{A}) The first ion binds to Na1\prim\ coordinated by both D312 and 
         D405, (\textit{B}) the initial position of the second \Na\ held with a harmonic 
         restraint, and (\textit{C}) the final position of the second \Na\ in the two 
         ion-state configuration after it is released (Na1). (\textit{Right column}) Time 
         evolution of distances (in \angs) between atoms with the first 9~ns representing 
         the release of restraints: (\textit{D}) Na1\prim\ and protein atoms, (\textit{E}) 
         Na1 and protein atoms and (\textit{F}) Na1 and Na1\prim. For clarity, the 
         running averages of the distances are plotted.}
\label{bind:fig1}
\end{figure}

The prevailing assumption at this stage is that the first \Na\ ion moves from Na1\prim\ to Na3 
before a second \Na\ ion can bind to Na1~\cite{Huang2010,Bastug2012,Setiadi2015}. But, as discussed 
in the \hyperref[bind:intro]{Introduction}, the binding free energies of \Na\ and Asp calculated with 
this assumption are in conflict with the experimental results. Thus we are compelled to examine this 
assumption more closely and look for alternative scenarios of binding. MD simulations lasting 40~ns 
indicate that the \Na\ ion at the Na1\prim\ site is very stable and exhibits no tendency to move towards 
Na3. We will show in the next section that such a transition is, in fact, energetically not feasible. 
Thus the presence of a second \Na\ in the vicinity of Na1\prim\ is necessary to enable the transition 
of the first \Na\ from Na1\prim\ to Na3. To search for such a site, we place the second \Na\ at 
the tip of the TM7a helix near the beginning of the NMDGT motif, where the carbonyl O atoms of 
G306, A307, T308, and N310 provide an oxygen-rich environment (\figrefi{bind:fig1}{B}). The ion is 
placed at the centre of these carbonyl O atoms and harmonically restrained with a spring constant 
of 5~\spring. We release the ion in 9 steps with $k=\{5.0,4.0,3.0,2.0,1.0,0.5,0.2,0.1,0.0\}$~\spring, 
performing 1~ns MD simulation at each step. The two \Na\ ions are initially separated by $\sim$6.5~\angs\ 
and remain in this position up to $k=2.0$~\spring. At $k=1.0$~\spring, the second \Na\ moves closer 
to Na1\prim\ with a water molecule separating the two ions and is weakly coordinated by G306 (O) and 
N310 (O) (\figrefn{bind:fig1}{E--D}). When the restraint is reduced to 0.1~\spring, the strong negative 
charge from D405 attracts \Na\ and knocks out the water molecule between the two ions 
(\figrefi{bind:fig1}{C}). The distance between the two \Na\ ions in this state is $\sim$3.5~\angs\ 
(\figrefi{bind:fig1}{F}). We subject the unrestrained system to a further 45~ns simulation to check 
the stability of the two \Na\ ions in close proximity. As shown in \figrefn{bind:fig1}{D--F}, both \Na\ 
ions remain well-coordinated and do not exhibit any tendency to break away from the coexistent state.

Firmness of the Na1\prim\ site becomes more evident after the binding of the second \Na, which has 
hardly made any dent on the Na1\prim\ coordination despite being so close (\tabref{bind:tab1}, columns 
5 and 6). In contrast, the presence of \Na\ at Na1\prim\ has a detrimental effect on Na1 coordination as 
seen from the comparison of the binding modes in columns 4 and 7 of \tabref{bind:tab1}. The stress on 
\Na\ at Na1 will be relieved after the transition of the first \Na\ ion from Na1\prim\ to Na3, which 
will be discussed further after the binding free energy calculations.

\begin{figure}[t!]
\centering
\includegraphics[width=1.0\linewidth]{Figures/Binding/fig2.jpg}
\caption{(\textit{Top row}) Asp bound to \GltPh\ with the \Na\ ions in the (\textit{A})
         Na1--Na3 state, and (\textit{B}) Na1--Na1\prim\ state. The dashed lines 
         represent the hydrogen-bond network between Asp and the Na1 site bridged with 
         a water molecule. (\textit{Bottom row}) The distance between the \Na\ ions and 
         the C$\alpha$ atom of Asp when it is bound to \GltPh\ in the (\textit{C})
         Na1--Na3 state, and (\textit{D}) Na1--Na1\prim\ state.}
\label{bind:fig2}
\end{figure}

\begin{figure}[t!]
\centering
\includegraphics[width=0.6\textwidth]{Figures/Binding/fig3.jpg}
\caption{Time evolution of the centre of mass of Asp in \GltPh\ with the two 
ion state Na1--Na1\prim. Black, red, and green represent Asp in chains A, B, 
and C, respectively. The running averages of the distances are plotted.}
\label{bind:fig3}
\end{figure}

\subsection{Binding Free Energies}
Here we calculate the binding free energies of ligands in various states of occupation to turn 
the qualitative insights gathered in the last section to quantitative predictions.
The results for the binding free energies of the \Na\ ions are summarised in \tabref{bind:tab2}.
As shown in \tabref{bind:tab1}, there is no stable Na1 site in the apo \GltPh, therefore it is not 
included in the binding free energy calculations for the first \Na. The first \Na\ can bind to either 
Na1\prim\ or Na3. The binding free energies in \tabref{bind:tab2} clearly favour the Na1\prim\ site 
by a margin of $-3.3$~kcal/mol. As mentioned before, simultaneous coordination by the D312 and D405 
side chains is responsible for the deeper binding and extreme stability of the first \Na\ at the 
Na1\prim\ site. 

\begin{table}[b!]
  \begin{center}
  \caption{Binding free energies of \Na\ ions to \GltPh\ in different states of occupation.$^{a}$}
  \label{bind:tab2}
    \begin{tabular}{lcccr}
    \hline
    Ion State & $\Delta\Delta G_{\text{int}}$ & $\Delta\Delta G_{\text{tr}}^{0}$ &
    $\Delta\Delta G_{\text{c,P}}$ & \multicolumn{1}{c}{$\Delta G_{\text{b}}^{0}$} \\ \hline
    Na1\prim\               & $-$26.7 $\pm$ 1.3 & $+$4.6 & $+$2.0 & $-$20.1 $\pm$ 1.3 \\
    Na3                     & $-$21.4 $\pm$ 1.4 & $+$4.6 &    -   & $-$16.8 $\pm$ 1.4 \\
    Na1 (Na1\prim)          & $-$10.8 $\pm$ 1.5 & $+$4.8 & $+$4.2 &  $-$1.8 $\pm$ 1.5 \\
    Na1 (Na3)               & $-$11.8 $\pm$ 1.5 & $+$4.9 &    -   &  $-$6.9 $\pm$ 1.5 \\
    Na1\prim\ (Asp, Na1)    & $-$24.2 $\pm$ 1.0 & $+$4.0 & $+$7.0 & $-$13.2 $\pm$ 1.0 \\
    Na3 (Asp, Na1)          & $-$24.0 $\pm$ 1.1 & $+$4.0 & $+$2.5 & $-$17.6 $\pm$ 1.1 \\ \hline
    \end{tabular}
    \end{center}
    {\footnotesize
    $^{a}$Labels inside the parenthesis indicate other ligands 
    present. $\Delta\Delta G_{\text{int}}$ is the average of the 
    forward and backward transformation. Convergence of the 
    transformation are shown in \figref{bind:figS1}. No restraints 
    are needed for the Na3 and Na1 (Na3) calculations. All energies 
    are reported in kcal/mol.}
\end{table}

We next consider the binding of the second \Na\ ion. The crucial question here is whether the
second \Na\ can bind to the Na1 site while Na1\prim\ is occupied so that the two \Na\ ions can 
coexist in close proximity. In order to calculate this binding free energy, we have included \Na\ 
at Na1\prim\ as part of the surrounding atoms restrained. Otherwise the FEP calculation becomes 
numerically unstable as the charge at Na1 is alchemically turned off. The effect of this restraint 
is included as part of $\Delta\Delta G_{\text{c,P}}$ as described in the \hyperref[bind:method]{Methods} 
section. The binding free energy of the second \Na\ at Na1 is found to be $-1.8$~kcal/mol. So the 
answer to the coexistence question is affirmative. This value is also in good agreement with the 
experimental values from Ewers et al.~\cite{Ewers2013}, Reyes at al.~\cite{Reyes2013b}, and H\"{a}nelt 
et al.~\cite{Hanelt2015} ($-2.3$, $-1.4$, and $-1.3$~kcal/mol, respectively). We stress that the slow 
binding of the second \Na\ with low affinity is made possible by the Coulomb repulsion of the first 
\Na\ at Na1\prim. If we use the alternative scenario as in previous calculations and assume that the 
first \Na\ is at Na3, the binding free energy of the second \Na\ to Na1 becomes $-6.9$~kcal/mol, which 
is in conflict with the experimental values. Thus we can infer from the free energy simulations that 
the \Na\ ions must be in the intermediate Na1--Na1\prim\ state during the measurements instead of the 
final Na1--Na3 configuration.

% Talk about Aspartate binding free energies
Having resolved the discrepancy in \Na\ binding affinity, we turn to the discrepancy in Asp affinity
for \GltPh. To present an unbiased view, we consider all possible ion configurations in the binding 
free energy calculations for Asp, which are listed in \tabref{bind:tab3}. Systems consisting of only 
Asp and one \Na\ at Na1 or Na3 are created by removing the ion at Na1 or Na3 from, respectively, from 
the Na1--Na3 configuration. The state with one \Na\ at Na1\prim\ is obtained similarly from the 
Na1--Na1\prim\ configuration. Finally, both \Na\ ions are removed for Asp binding to the apo \GltPh. 
Inspection of \tabref{bind:tab3} shows that positive binding free energies are obtained for the apo 
\GltPh\ and also with one \Na\ at Na3 or Na1\prim. A negative free energy is obtained for a \Na\ at 
Na1, but this site does not form with only one \Na. Thus two \Na\ ions are required for the binding of 
Asp consistent with experiments~\cite{Vandenberg2013,Tao2010,Reyes2013b,Ewers2013,Hanelt2015}.

\begin{table}[b!]
  \begin{center}
  \caption{Binding free energies of Asp bound to \GltPh\ with different ion 
  configurations.$^{a}$}
  \label{bind:tab3}
    \resizebox{\textwidth}{!}{\begin{tabular}{lrcccccr}
    \hline
    Substrate State & \multicolumn{1}{c}{$\Delta\Delta G_{\text{elec}}$} & 
    $\Delta\Delta G_{\text{LJ--bb}}$ & $\Delta\Delta G_{\text{LJ--sc}}$ & 
    $\Delta\Delta G_{\text{tr}}^{0}$ & $\Delta\Delta G_{\text{rot}}$ & 
    $\Delta\Delta G_{\text{c,L}}$ & \multicolumn{1}{c}{$\Delta G_{\text{b}}^{0}$} \\ \hline
    Asp                          &  $-$4.7 $\pm$ 0.8 & $+$0.5 $\pm$ 0.3 & $+$0.8 $\pm$ 0.2 & $+$2.8 & $+$3.4 & 
$+$0.2 & $+$3.0 $\pm$ 0.9 \\ 
    Asp (Na3)                    &  $-$8.6 $\pm$ 1.5 & $+$1.8 $\pm$ 0.3 & $+$1.0 $\pm$ 0.3 & $+$3.9 & $+$4.1 & 
$-$0.1 & $+$2.1 $\pm$ 1.6 \\ 
    Asp (Na1\prim)               & $-$10.7 $\pm$ 1.1 & $+$1.5 $\pm$ 0.3 & $+$0.9 $\pm$ 0.3 & $+$3.8 & $+$3.6 &
$+$1.7 & $+$0.8 $\pm$ 1.2 \\
    Asp (Na1)                    & $-$16.2 $\pm$ 1.2 & $+$3.1 $\pm$ 0.3 & $+$1.3 $\pm$ 0.3 & $+$4.1 & $+$4.1 & 
$+$1.2 & $-$2.4 $\pm$ 1.3 \\
    Asp (Na1, Na3)               & $-$16.6 $\pm$ 1.3 & $+$4.4 $\pm$ 0.4 & $+$1.1 $\pm$ 0.3 & $+$4.2 & $+$4.1 & 
$-$0.8 & $-$3.6 $\pm$ 1.4 \\
    Asp (Na1, Na1\prim)          & $-$22.5 $\pm$ 0.8 & $+$4.4 $\pm$ 0.3 & $+$0.6 $\pm$ 0.3 & $+$4.0 & $+$4.2 & 
$+$1.9 & $-$7.4 $\pm$ 0.9 \\ \hline
    \end{tabular}}
    \end{center}
    {\footnotesize
        $^{a}$Labels inside the parenthesis indicate the ion states. 
        The electrostatic and LJ energies are averages of the forward 
        and backward transformation. Convergence of the transformation 
        for the electrostatic and LJ interactions are shown in 
        \hyperref[bind:figS2]{Fig.~\ref{bind:figS2}}. All energies are 
        reported in kcal/mol.}
\end{table}

The possible ion configurations with two \Na\ are Na1--Na3 and Na1--Na1\prim. The Na1--Na3 configuration 
comes from the crystal structure and results in a low-affinity binding of Asp ($-3.6$~kcal/mol) as found 
in previous calculations~\cite{Heinzelmann2011,Heinzelmann2013}. This is in conflict with experiments 
where Asp is determined to bind with high affinity, e.g., $-7.3$~kcal/mol~\cite{Ewers2013} and 
$-6.8$~kcal/mol \cite{Hanelt2015} (both were measured at 6~$^\circ$C). The binding free energy of Asp 
is significantly boosted when the calculation is done using the Na1--Na1\prim\ configuration. The 
calculated value ($-7.4$~kcal/mol) is in very good agreement with the experimental values quoted above. 
The difference between the binding free energies of the two configurations arises from the distance of 
the first \Na\ to Asp at the Na3 and Na1\prim\ sites, which is depicted in 
\figrefn{bind:fig2}{C {\color{black} and} D}, respectively. The distance between the \Na\ at Na1 and the 
C$\alpha$ atom of Asp is about 7~\angs\ for both systems. The C$\alpha$ distance with the first \Na\ ion, 
however, is 9.5~\angs\ at Na1\prim\ and increases to 12.5~\angs\ at Na3. This 3~\angs\ increase in the 
distance significantly reduces the electrostatic attraction, leading to a smaller binding free energy of 
Asp for the Na1--Na3 configuration. 

As in the case of the binding of the second \Na, the free energy simulations imply that Asp must 
bind while the first \Na\ is still at the Na1\prim\ site. That is the only way one can explain the
high-affinity binding of Asp. We expect that the binding of the second \Na\ to Na1 will sufficiently 
destabilise the first \Na\ at Na1\prim\ so that it will not be the deepest binding site anymore.
To confirm this, we have repeated the free energy calculations for the binding of \Na\ to the
Na1\prim\ and Na3 sites in the presence of the second \Na\ at Na1 and bound Asp. As shown in the 
last two rows of \tabref{bind:tab2}, the Na3 site is now 4.4~kcal/mol deeper than the Na1\prim\ site, 
so the first \Na\ will eventually move about $\sim$4~\AA\ from Na1\prim\ to Na3. But this transition is 
clearly delayed by an energetic barrier caused by the detachment of the D405 side chain, and the 
flipping of the D312 and N310 side chains which coordinate \Na\ at both sites (cf. columns 2 and 6 
of \tabref{bind:tab1}). 

In order to get more insights for this barrier, we have performed potential of mean force (PMF)
calculations for the Na1\prim\ $\rightarrow$ Na3 transition of the first \Na. The D312 side chain is 
found to chaperone \Na\ throughout the transition while the N310 side chain flips at the top of the 
barrier. The flipping of the N310 side chain is an unfavourable motion and causes some stress on the 
NMDGT motif, which has to move to allow the transition. Unfortunately, we could not obtain a converged 
PMF for the whole path due to the sampling problems caused by the large conformational changes in the 
protein. Because the existence of this barrier is essential for the stability of the proposed (Na1, 
Na1\prim) state, we show the initial part of the PMF, where convergence is not an issue yet 
(\figrefn{bind:fig4}{A}). As the \Na\ ion is pushed from the Na1 site towards Na3, the PMF steeply rises, 
reaching a height of over 10~kcal/mol, which is sufficient to trap the first \Na\ at Na1\prim.

The residue-specific RMSDs obtained from the initial and final umbrella windows show that the movement 
of the \Na\ ion induces substantial conformational changes in the NMDGT motif and the neighbouring 
residues when the ion is at the top of the barrier (\figrefn{bind:fig4}{B}, red bars). In contrast, 
the RMSDs of the same residues obtained from the equilibrium simulations of the (Na1, Na3) state indicates 
that they are completely relaxed after the \Na\ ion moves to the Na3 site (\figrefn{bind:fig4}{B}, black 
bars). The PMF and RMSD results in \figref{bind:fig4} show that the protein is in a slightly excited in 
the (Na1, Na1\prim) state and is driven into a highly excited state during the Na1\prim\ $\rightarrow$ Na3 
transition of \Na. After \Na\ binds to Na3, the protein relaxes to a minimum free energy state and the 
conformational energy stored in the protein is completely dissipated. The above depiction of events during 
the Na1\prim\ $\rightarrow$ Na3 transition is consistent with the experimental observations that the \Na\ 
binding induces structural changes in the protein~\cite{Reyes2013b,Hanelt2015}, and provides a physical 
mechanism for how these changes are induced. Here we stress the role of the (Na1, Na1\prim) intermediate 
state in facilitating the Na1\prim\ $\rightarrow$ Na3 transition of the first \Na. Without the Coulomb 
repulsion of the second \Na\ ion at Na1, the first \Na\ would have a much harder time pushing through 
the conformational energy barrier. Such a knock-on effect is well-known in ion channels. Nature seems 
to have exploited it in glutamate transporters as well. 

\begin{figure}[t!]
\centering
\includegraphics[width=1.0\linewidth]{Figures/Binding/fig4.png}
\caption{(A) The PMF of the \Na\ ion at Na1\prim\ as it is pushed towards Na3. 
Only the converged part of the PMF is shown. (B) Residue-specific RMSDs of the 
NMDGT (310-314) and neighbouring residues (306-309) affected by the Na1\prim\ 
$\rightarrow$ Na3 transition with respect to the 2NWX crystal structure. The 
RMSDs obtained from equilibrium simulations when the two \Na\ ions are at the 
(Na1, Na1\prim) and (Na1, Na3) states are shown with blue and black, respectively. 
The RMSDs obtained from the last umbrella window in the PMF, which indicates the 
barrier position, is shown with red.}
\label{bind:fig4}
\end{figure}

\begin{figure}[t!]
\centering
\includegraphics[width=1.0\linewidth]{Figures/Binding/fig5.png}
\caption{Schematic diagram of the ligand-binding order in \GltPh. 
(\textit{A}) Initially, \GltPh\ is in the open apo state. A \Na\ ion binds to the Na1\prim\ site, 
coordinated by the D405 and D312 residues. 
(\textit{B}) A second \Na\ ion pushes the first ion slightly and binds to the Na1 site coordinated by 
the D405 residue, thus forming the Na1--Na1\prim\ two ion state. 
(\textit{C}) The Asp substrate binds to \GltPh\ in this two ion configuration. 
(\textit{D}) The first \Na\ ion moves from Na1\prim\ to Na3 chaperoned by the D312 side chain. 
The flipping of the N310 side chain oxygen from the Na1\prim\ site to the Na3 site is the critical event
that enables this transition.}
\label{bind:fig5}
\end{figure}

From the binding free energy results, we propose the following mechanism for the binding of the ligands 
to \GltPh\ (illustrated in \figref{bind:fig5}). Starting with the outward-facing apo state of \GltPh, 
the first \Na\ ion binds to the Na1\prim\ site with a very high affinity, where it is coordinated by 
both D405 and D312. This is followed by the entry of the second \Na\ ion (Asp cannot bind yet), which 
binds to the Na1 site, where it is coordinated by D405. The distance between the two ions is about 
3.5~\angs, which results in low-affinity binding of the second \Na. In this two ion Na1--Na1\prim\ 
state, the Asp substrate binds to \GltPh\ with high affinity. Finally, the first \Na\ moves from 
Na1\prim\ to Na3 through conformational changes involving the N310 and D312 side chains, and the NMDGT 
motif. Although we couldn't resolve this transition, both experiments~\cite{Reyes2013b,Hanelt2015} 
and preliminary PMF calculations indicate that it will be associated with a large energy barrier, and 
hence it will be a slow process. In the final state, the HP2 gate is closed and locked by the binding
of the third \Na\ to the Na2 site. MD simulations and binding free 
energies~\cite{Ryan2009,Grazioso2012,Setiadi2017} suggest that both of these processes are relatively 
fast, so it is likely that the binding of the first \Na\ to its final destination at the Na3 site will 
occur last.

\section{Conclusion}
In order to resolve the large discrepancies found in the calculation of the binding free energies of 
ligands, we have re-examined the ligand-binding mechanism in the outward-facing conformation of \GltPh. 
We have included the intermediate \Na\ binding site labelled Na1\prim, and found it to be very stable. 
The coordination of the Na1\prim\ site is tighter compared to Na3, and as evidenced by the binding free
energy calculations, the Na1\prim\ site is at a lower energy state than Na3. In addition, the transition 
of \Na\ from Na1\prim\ to Na3 requires flipping of the N310 side chain, which is an unfavourable motion 
due to the stress it causes on the NMDGT motif. We have then considered the possibility that the second 
\Na\ ion could bind to Na1 while the first \Na\ is still at Na1\prim. The two ions at the Na1--Na1\prim\ 
sites are found to be stable, separated by 3.5~\angs. The Coulomb repulsion of the first \Na\ at Na1\prim\ 
ensures that the second \Na\ binds to Na1 with a low affinity in good agreement with the experimental 
measurements~\cite{Reyes2013b,Ewers2013,Hanelt2015}. We have next calculated the binding free energy of 
Asp for all ion occupation states and found that the lowest free energy is obtained when the two \Na\ ions 
are at the Na1--Na1\prim\ sites. The calculated binding free energy is again in good agreement with the 
experimental measurements, which indicate high-affinity Asp binding~\cite{Hanelt2015,Ewers2013}. 

The intermediate Na1\prim\ site, which has helped to resolve the low-affinity \Na -- high-affinity Asp 
binding paradox, is identified through the free energy simulations in this work. This site has not been
observed in previous crystal structures, because it disappears in the fully-bound \GltPh. However, if 
the transition of the first \Na\ from Na1\prim\ to Na3 can be prevented through mutations, \GltPh\ can 
be trapped in a state with two \Na\ ions at the Na1--Na1\prim\ sites. Inspection of \tabref{bind:tab1} 
shows that two such mutations are T92A and S93A. In fact, these mutations have been considered before 
to identify the Na3 site \cite{Bastug2012}. Binding free energy calculations indicate that these 
mutations will reduce the \Na\ affinity to Na3 by 6--7 kcal/mol, which is sufficient to prevent the 
occupation of Na3. Thus the proposed Na1--Na1\prim\ intermediate binding site can be resolved if \GltPh\ 
is crystallised with one of the above mutations.

The existence of the Na1\prim\ site will also affect the release mechanism of the ligands in the inward
conformation of \GltPh, which has to be reconsidered. More interestingly, the Na1--Na1\prim\ sites could 
help to clarify the role of the \K\ ion in the transport cycle of EAATs. It has been hypothesised that 
the release of the last \Na\ ion is a rate-limiting step in \GltPh, and its exchange with \K\ in EAATs 
avoids this slow process \cite{Heinzelmann2013,Heinzelmann2014a}. But no concrete mechanism could be 
proposed for such a \K--\Na\ exchange. The Na1--Na1\prim\ sites provide a natural location for the 
\K--\Na\ exchange to occur in EAATs. Finally, such an intermediate \Na\ binding site could exist in other 
\Na--coupled transporters. Because they are not likely to be observed in crystal structures, one may have 
to rely on discrepancies in ligand binding free energies to infer their existence. We hope to report on 
the implications of the intermediate Na1\prim\ site on the transport cycle of EAATs and other transporters 
in feature publications.

The existence of the Na1\prim\ site will also affect the release mechanism of the last \Na\ from the Na3 
site in the inward conformation of \GltPh, which has to be reconsidered. More interestingly, the 
Na1--Na1\prim\ sites could help to clarify the role of the \K\ ion in the transport cycle of EAATs, 
which have much faster turnover rates compared to \GltPh. It has been hypothesised that the release of the 
last \Na\ ion---without any external help---could be too slow for EAATs, and its exchange with \K\ could 
avoid this slow process \cite{Heinzelmann2014a}. But no concrete mechanism could be proposed for such a 
\K--\Na\ exchange. The Na1--Na1\prim\ sites provide a natural location for the \K--\Na\ exchange to occur 
in EAATs. Finally, such an intermediate \Na\ binding site could exist in other \Na--coupled transporters. 
Because they are not likely to be observed in crystal structures, one may have to rely on discrepancies 
in ligand binding free energies to infer their existence. We hope to report on the implications of the 
intermediate Na1\prim\ site on the transport cycle of EAATs and other transporters in feature publications.

\pagebreak
\begin{subappendices}
\counterwithin{figure}{section}
{
\hypersetup{linkcolor=black}
\section{Appendix for Chapter~\ref*{chap:bind}}
}

% Figure S1
\begin{figure}[b!]
    \centering
    \includegraphics[width=0.8\linewidth]{Figures/Binding/figS1.jpg}
    \caption{The convergence of TI results for the binding free energies of \Na\ ions.
    The black and red lines are the forward and backward transformations,
    respectively.}
    \label{bind:figS1}
\end{figure}

\pagebreak
% Figure S2
\begin{sidewaysfigure}
    \centering
    \includegraphics[scale=0.55]{Figures/Binding/figS2.jpg}
    \caption{The convergence of the forward (black) and backward (red) 
    transformations of Asp. The first row is the electrostatic interaction
    presented as the running averages of the TI calculations. The middle 
    and bottom rows show the FEP calculations for the LJ--bb and LJ--sc interactions.
    These plots show the FEP transformation as a function of the coupling 
    parameter $\lambda$.}
    \label{bind:figS2}
\end{sidewaysfigure}

\pagebreak
% Figure S3
\begin{figure}
    \centering
    \includegraphics[width=0.5\linewidth]{Figures/Binding/figS3.jpg}
    \caption{(A) Superposition of the ligand-free 2NWX and 4OYE \GltPh\ structures
    after 60~ns of MD simulations, represented in orange and yellow, respectively.
    (B) The time evolution of the RMSDs of the two structures with respect to 
    the 4OYE crystal structure. The average RMSDs 
    after equilibration are 1.09~\AA\ and 1.06~\AA\ for the apo-2NWX and 4OYE 
    systems, respectively. The HP2 gate moves freely in the apo state, and therefore 
    excluded from the RMSD calculations.}
    \label{bind:figS3}
\end{figure}

\end{subappendices}

\counterwithin{figure}{chapter}
%=======================================================================================%