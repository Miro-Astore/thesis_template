%=======================================================================================%
\chapter{Ion Solvation Energy with Spherical Boundary Conditions}
\label{chap:ions}
ABSTRACT \newline
Solvation free energies of ions are difficult to determine from molecular 
dynamics simulations due to the long-range Coulomb interactions. Various 
approximations and corrections are introduced to enable their calculation in 
small systems, which however raises issues of robustness. We show that 
solvation free energies of ions can be calculated using the spherical boundary 
conditions without introducing any corrections at the boundary via a buffer 
zone. The results are shown to converge for a droplet size of 21~\angs\ and are 
independent of the parameters used for confining water or restraining the ion. 
The proposed method thus resolves the robustness issues in solvation free 
energy calculations of ions and can be used with confidence to determine
force field parameters from such calculations. We apply the method to calculate 
the solvation free energies of the side chain analogues of charged amino acids. 
Tests using periodic boundary conditions show that similar results are also 
obtained in that case. 

\newpage
\section{Introduction}
Solvation of ions in water and other solvents is a fundamental phenomenon in physical chemistry with 
applications in many other fields such as biochemistry, molecular biology and 
physiology~\cite{Hunenberger2011,Marcus1986}. Properties of solvated ions can be described at several 
levels depending on the desired level of accuracy and detail. Two of the most common approaches are the 
Poisson-Boltzmann equation with implicit solvent and molecular dynamics (MD) simulations with explicit 
solvent molecules. Thanks to the improvements in computer hardware and software, MD simulations have 
now become the primary tool for atomic-level description of biomolecules and their interactions. The 
accuracy of simulation results depends on the quality of the force fields used in MD simulations. In 
this regard, accurate calculation of the solvation free energies of solutes is of primary importance 
because the balance between the solute--solute and solute--solvent interactions determines whether two 
solute molecules bind or not. For this reason, solvation free energies are routinely used for determining 
the parameters in force fields~\cite{Brooks2009b,Wang2004,Oostenbrink2004,Jorgensen1996}. The long-range 
Coulomb potential has been the main obstacle in the calculation of the solvation free energies of ions from 
MD simulations. It was shown early on that the use of a cut-off radius in the calculation of the Coulomb 
interactions gave spurious results and had to be avoided~\cite{Straatsma1988a}. Because the size of 
simulation systems was limited by the available computer power, the only alternatives were using either 
a simulation box with periodic boundary conditions (PBC) and Ewald sum for long-range interactions,~\cite{Hummer1998,Hunenberger2011} or a droplet with spherical boundary conditions (SBC) 
and a buffer zone with implicit solvent beyond~\cite{Aqvist1990,Beglov1994}. In the former method, a 
number of corrections arising from periodicity and finite-size effects have to be applied to the raw 
solvation free energy obtained from MD simulations, which raises doubts about the robustness of the 
results~\cite{Lin2014a}. Using a droplet with SBC avoids these issues, but the introduction of a buffer 
zone, where external forces are applied to water molecules to counter the effect of the vacuum outside, 
brings back the robustness question. It is difficult to quantify how successful the buffer zone is in 
mimicking the actual behaviour of water molecules embedded in a much larger system.

The issue with the robustness of the computations in the SBC method could be resolved if a large enough 
system is used so that one can dispense with the buffer zone and just embed the droplet in a cavity 
of implicit water (\figref{solv:fig1}). The currently available computer power certainly allows 
consideration of much larger simulation systems compared to two decades ago. All one needs to do is 
to show that the computed solvation energy of an ion converges to a unique value with increasing droplet 
size independent of the parameters used for the confinement of water molecules or restraining of the ion. 
Here we demonstrate the feasibility of this simpler approach by computing the solvation free energies 
of \Na\ and \Cl\ ions using a droplet embedded in a cavity whose size is varied. We show that the 
correction terms arising from the confinement of the droplet and its interface with vacuum become 
negligibly small for droplet radii greater than 20~\angs. We then generalise the cavity energy to a 
molecular ion with arbitrary partial charges and use our method to compute the solvation free energies 
of the side chain analogues of the charged amino acids. Tests with PBC show that convergent solvation 
free energies are also obtained in that case when simulation boxes with volumes similar to droplets 
are used. Finally, we discuss the implications of the results for the current force fields.

\begin{figure}[t!]
\includegraphics[width=1.0\textwidth]{Figures/Solvation/fig1.jpg} 
\caption{(A) The full system of an ion embedded in a bath of explicit water. A 
         spherical boundary with radius $R$ is placed around the ion to split the 
         system into two parts. (B) A water droplet with explicit water is cut 
         out from the full system and placed in a vacuum. (C) A hollow spherical 
         cavity (vacuum) embedded in an implicit water with the ion at the centre. 
         The full ion-water interaction in (A) is obtained by adding the ion-water 
         interactions in (B) and (C).} 
\label{solv:fig1}
\end{figure}

\section{Method}
\subsection{Solvation Free Energy with SBC}
In the SBC method, the simulation system is divided into two parts using a spherical boundary around 
the ion, and the water outside the boundary is represented implicitly (\figref{solv:fig1}). In 
previous applications of the SBC method~\cite{Aqvist1990,Beglov1994}, a buffer zone is used to rectify 
the effect of vacuum on water molecules near the boundary. This was necessary for the relatively small 
spheres ($R\sim 10$~\angs) employed in these work because the structural order imposed by the vacuum 
boundary could affect the hydration shells of the ion. This would not be a problem when a sufficiently 
large sphere is employed. A second artefact introduced by the vacuum boundary is the monopole ($l=0$) 
and higher-order contributions to the solvation energy, which depend on $R$ as $1/R^{l+1}$ and would 
be negligible for large $R$. We note that the monopole term is cancelled by the interaction of the ion 
with the cavity, given by the Born energy
\begin{equation}
\Delta G_{\text{cav}} = -\frac{(1-1/\epsilon)q^2}{8\pi\epsilon_0 R},
\label{eq:born}
\end{equation}
where $q$ is the charge of the ion and $\epsilon$ is the dielectric constant of water. The remaining 
artefacts fall off as $1/R^2$ or faster so with an appropriate choice of $R$, they can be made negligibly 
small. 

Here we investigate the feasibility of this approach for practical applications. The system is decomposed 
into a water droplet and a cavity part as shown in \hyperref[solv:fig1]{Fig.~\ref{solv:fig1}}, and the solvation free energy of the ion is calculated from
\begin{equation}
\Delta G_{\text{solv}} = \Delta G_{\text{drop}} + \Delta G_{\text{cav}}.
\label{eq:solvation}
\end{equation}
Here $\Delta G_{\text{drop}}$ is the solvation free energy of the ion in the droplet, and $\Delta G_{\text{cav}}$ 
is the self-energy of the ion in the cavity, which is given by the Born energy, Eq.~\eqref{eq:born}, 
for a central ion.  The solvation free energy is usually calculated by an alchemical route, i.e., by 
slowly turning off the interaction between the ion and water molecules in reference to vacuum. To 
improve convergence, the contribution of the electrostatic and Lennard-Jones (LJ) interactions are 
calculated separately
\begin{equation}
\Delta G_{\text{drop}} = \Delta G_{\text{drop-el}} + \Delta G_{\text{drop-LJ}} .
\end{equation}
Note that due to the $1/r^6$ dependence of the LJ interactions, their contribution to the cavity term 
can be neglected. The electrostatic and LJ interactions of the ion with the water molecules inside the 
droplet can be calculated by either free energy perturbation (FEP) or thermodynamic integration 
(TI)~\cite{Chipot2007}. In both methods, the ion-water interaction is coupled/decoupled over $n$ 
simulation windows chosen between $\lambda=0$ and 1. The sum of the change in free energy or the 
integral of the free energy gradient over $\lambda$ gives the final free energy for FEP and TI, 
respectively. 

As mentioned above, the monopole contribution to $\Delta G_{\text{drop-el}}$ is cancelled by 
$\Delta G_{\text{cav}}$, which removes the $1/R$ dependence from the solvation free energy. In 
contrast, there is no $1/R$ dependence in $\Delta G_{\text{drop-LJ}}$. Because this will be important 
when we study the convergence of the solvation free energy with $R$, we rewrite the solvation free 
energy of the ion in the form
\begin{equation}
\Delta G_{\text{solv}} = \Delta G_{\text{el}} + \Delta G_{\text{LJ}} ,
\end{equation}
where 
\begin{equation}
\Delta G_{\text{el}} = \Delta G_{\text{drop-el}} + \Delta G_{\text{cav}}
\label{eq:split}
\end{equation}
represents the total electrostatic contribution to the free energy, and we have removed the ``drop'' 
label from the LJ term to simplify the notation. Representation of the cavity term with the Born energy 
is valid only for a central ion. For a molecule with distributed partial charges, the self-energy 
contribution due to the cavity needs to be generalised, which we consider next.

\subsection{Self-energy of Ions and Molecules in a Cavity}
\begin{figure}[b!]
\centering
\includegraphics[width=0.45\textwidth]{Figures/Solvation/fig2.jpg}
\caption{A spherical cavity with radius $R$ is embedded in implicit 
         water medium. A charge $q$ is placed on the $z$-axis at a 
         distance $d$ from the origin.}
\label{solv:fig2}
\end{figure}

We first consider the self-energy of a point charge $q$ placed at $d \hat{\bf z}$ in a spherical cavity 
of radius $R$ (see \figref{solv:fig2}). The dielectric constant outside the sphere is $\epsilon=80$ 
(implicit water), and inside it is 1 (vacuum). The potentials inside and outside the cavity can be 
determined by expanding the potentials in terms of Legendre polynomials and applying the appropriate 
boundary conditions at $r=R$~\cite{Jackson1999}. The potential inside the cavity due to the induced 
charges at the boundary (reaction field) is given by
\begin{equation}
\varphi_{\text{RF}} = 
-\frac{q}{4\pi\epsilon_0}\sum_{l}\frac{(l+1)(\epsilon-1)}
{(l+1)\epsilon+l } \frac{d^l r^l}{R^{2l+1}}P_{l}(\theta).
\label{eq:rf1}
\end{equation}
Because $\epsilon \gg 1$, we can neglect the second term in the denominator of Eq.~\eqref{eq:rf1} 
without much error. Then the factors of $l+1$ from the numerator and denominator cancel, yielding
\begin{equation}
\varphi_{\text{RF}} \simeq 
-\frac{(qR/d)}{4\pi\epsilon_0}\frac{(\epsilon-1)}{\epsilon}
\sum_{l}\frac{r^l}{(R^2/d)}P_{l}(\theta).
\label{eq:rf2}
\end{equation}
Here, the sum over $l$ corresponds to the expansion of $1/|{\bf r} -(R^2/d)\hat{\bf z}|$. Thus the 
reaction potential in Eq.~\eqref{eq:rf2} can be represented by an image charge of magnitude 
$q'=-(1-1/\epsilon)qR/d$ placed at $R^2/d$ on the $z$-axis. Generalising this result to a charge $q$ 
at an arbitrary position ${\bf r'}$ in the cavity, we obtain
\begin{equation}
\varphi_{\text{RF}} = -\frac{(1-1/\epsilon)}{4\pi\epsilon_0} \frac{qR/r'}
{|{\bf r}-(R/r')^2{\bf r'}|}.
\label{eq:rf3}
\end{equation}
The self-energy of the charge $q$ in the cavity follows from 
\begin{eqnarray}
\Delta G_{\text{cav}} &=& \frac{1}{2}q\varphi_{\text{RF}}({\bf r}={\bf r'})
\nonumber\\
&=& 
-\frac{(1-1/\epsilon)}{8\pi\epsilon_{0}} \frac{q^2R}{R^2 - r'^2},
\label{eq:ise}
\end{eqnarray}
which reduces to the Born energy for $r'=0$. Thus the approximation made to obtain Eq.~\eqref{eq:rf2} 
does not affect the leading monopole term; the  $\simeq 1$\% error affects only the higher-order terms.

We next use the superposition principle to generalise the above results to an ionic molecule in a 
cavity, which has $N$ partial charges $q_i$ at positions ${\bf r}_i$. From Eq.~\eqref{eq:rf3}, the 
reaction potential in the cavity is given by
\begin{equation}
\varphi_{\text{RF}} ({\bf r}) = -\frac{(1-1/\epsilon)}{4\pi\epsilon_0} 
\sum_{i} \frac{ q_i R/r_i}{|{\bf r} - (R/r_i)^2 {\bf r}_i |},
\end{equation}
and the self-energy of the molecule follows from
\begin{eqnarray}
\Delta G_{\text{cav}} &=& \frac{1}{2}\sum_{j}q_j\varphi_{\text{RF}}({\bf r} = 
{\bf r}_j ) \nonumber\\
&=& -\frac{(1-1/\epsilon)}{8\pi\epsilon_{0}} \sum_{ij} 
\frac{q_i q_j R}{|\ R^2\hat{\bf r}_i -r_i r_j \hat{\bf r}_j |}.
\end{eqnarray}
Separating the diagonal and cross-terms, the self-energy can be written in the form
\begin{equation}
\Delta G_{\text{cav}} = 
-\frac{(1-1/\epsilon)}{8\pi\epsilon_{0}}
\left[\sum_{i}\frac{q_{i}^2 R}{R^2 - r_{i}^2} 
+ \sum_{i>j}\frac{2q_i q_j 
R}{|R^2\hat{\bf r}_{i}-r_{i}r_{j}\hat{\bf r}_{j}|}\right].
\label{eq:mse}
\end{equation}
The distance in the denominator of the cross term can be calculated using 
\begin{equation}
| R^2 \hat{\bf r}_{i}-r_{i}r_{j}\hat{\bf r}_{j}|= \left[R^4 + (r_{i}r_{j})^2 - 
2R^2 {\bf r}_i \cdot {\bf r}_j\right]^{1/2}.
\end{equation}
Eq.~\eqref{eq:mse} generalises the self-energy of an ion in a cavity to a molecule with arbitrary 
partial charges on its atoms. It will be used in calculating the cavity contribution to the solvation 
free energy of a molecule.

\subsection{Computational Details for SBC}
We first consider the solvation free energies of \Na\ and \Cl\ to establish the bona fides of the 
proposed method. The method is then used to investigate the solvation free energies of the side chain 
analogues of the charged amino acids. Side chain analogues are built by removing the backbone of the 
amino acids and terminating the dangling carbon atom with a hydrogen atom. The partial charge of the 
dangling atom is adjusted to preserve the total charge of the side chain analogue. The topologies are 
available as part of the CHARMM36 CGenFF~\cite{Brooks2009} distribution. The solute molecule is solvated 
in a sphere of a given radius $R$ with TIP3P~\cite{Jorgensen1983,Neria1996} molecules using the PACKMOL 
program~\cite{Martinez2009}. All MD simulations are performed in OpenMM version 7.2.2~\cite{Eastman2016} 
with a time step of 2~fs, and the temperature is kept constant at 300~K with a Langevin damping of 
1~ps$^{-1}$. Calculations are run on V100 Tesla graphical processing units (GPU) at the high-performance 
computing facility {\it Artemis} with mixed precision (a test with double precision gave virtually the 
same result). 

In order to maintain the density of the water droplet, a half-harmonic potential is applied to the 
oxygen atom of water molecules
\begin{equation}
U_{\text{sphere}} = 
\begin{cases}
\frac{1}{2} k_{s} (r-r_{0})^2,& \text{if } r>r_{0}, \\
0, & \text{otherwise},
\end{cases}
\label{eq:drop_pot}
\end{equation}
where $k_{s}$ is the spring constant and the equilibrium $r_{0}$ is defined as the droplet radius $R$ 
minus the average fluctuation from the restraint at temperature $T$, i.e., 
$r_{0}=R-\sqrt{k_{\text{B}} T/k_{s}}$. The offset ensures that the monopole contribution to the solvation 
free energy from the droplet at the boundary is cancelled by the cavity term.  This is similar to the 
potential used in the MD program Q~\cite{Marelius1999} but without an attractive Morse potential. A 
spring constant of 10~\spring\ is employed in most of our simulations. In addition to the confinement 
potential, a harmonic restraint of 10~\spring\ is applied to the centre of charge of the solute to keep 
it close to the origin. The choice of these two parameters does not affect the final solvation free 
energy significantly as will be shown in the results and discussion section. In previous calculations 
using the SBC method, angular restraints were applied on water molecules in the boundary region to 
prevent their ordering at the vacuum interface so that their behaviour was consistent with those in 
a much larger droplet~\cite{Aqvist1990,Beglov1994}. Here we do not use any angular restraints. As will 
be shown in the results section, such boundary effects become negligible when the system is chosen 
large enough.

FEP calculations are performed to obtain the free energy terms $\Delta G_{\text{drop-el}}$ and 
$\Delta G_{\text{LJ}}$. For both the droplet electrostatic and LJ terms, we use 21 equally spaced 
windows with a sampling time of 0.1 ns/1.0 ns for equilibration/production at each $\lambda$ value. 
To improve convergence in the LJ calculations, we use a soft-core potential with a radius-shifting 
coefficient of 5.0~\angs. The free energy is calculated with the multistate Bennett-Acceptance-Ratio 
(MBAR) method~\cite{Shirts2008} using the python implementation {\it pymbar} available at 
\url{https://github.com/choderalab/pymbar}. The cavity correction $\Delta G_{\text{cav}}$ is 
calculated using Eq.~\eqref{eq:ise} for an ion and Eq.~\eqref{eq:mse} for a molecule, and averaged 
over a 2~ns MD simulation for each system. See \appref{apx:charges} for the python implementation.

\subsection{Solvation Free Energy with PBC}
The intrinsic solvation free energy with PBC is defined as 
\begin{equation}
\Delta G_{\text{solv}}^{\text{intr}} = \Delta G_{\text{el}} + \Delta G_{\text{LJ}},
\end{equation}
where the two terms represent the electrostatic and LJ contributions, respectively. In order to compare 
the experimental results, a surface term needs to be added to the intrinsic solvation free energy due to 
the absence of a vacuum interface in PBC. Thus the “real” solvation free energy is given by~\cite{Lin2014a}
\begin{equation}
\Delta G_{\text{solv}}^{\text{real}} = \Delta G_{\text{el}} + \Delta G_{\text{LJ}} + zF\phi_{\text{lv}},
\end{equation}
where $z$ is the valence of the ion, $F$ is the Faraday constant, and $\phi_{\text{lv}}$ is the 
electrostatic potential that arises from a liquid-vacuum interface.  For the TIP3P water model used in 
this work, $\phi_{\text{lv}}=-0.52$~V~\cite{Lin2014a}, so $-$12.0$z$ kcal/mol needs to be added to the 
intrinsic value to obtain the real solvation free energy. The electrostatic contribution is split into 
two terms in the spirit of Eq.~\eqref{eq:split},
\begin{equation}
\Delta G_{\text{el}} = \Delta G_{\text{box-el}} + \Delta G_{\text{SE}},
\end{equation}
where $\Delta G_{\text{box-el}}$ gives the electrostatic contribution of the water molecules inside 
the PBC box and $\Delta G_{\text{SE}}$ describes the self-energy of the ion in the box, similar to the 
$\Delta G_{\text{cav}}$ term in Eq.~\eqref{eq:split}. For  a cubic box with size $L$, the self-energy 
is given by~\cite{Hummer1996}
\begin{equation}
\Delta G_{\text{SE}} = -\frac{2.837297}{4\pi\epsilon_{0}}\frac{q^2}{2L}.
\end{equation}
Like the Born energy~\eqref{eq:born}, the self-energy drops as $1/L$ and is not negligible even for 
relatively large box sizes. 

Simulations of ion solvation with PBC are done using the NAMD (version 2.12)~\cite{Phillips2005} program 
that implements a particle mesh Ewald (PME) summation with a tinfoil boundary. A cut-off of 12~\angs\ for 
the non-bonded interactions and a smooth switching is applied at 10~\angs. The long-range LJ interactions 
between the ion and water are included using an analytical correction~\cite{Shirts2007}. A constant 
temperature of 300 K is set using the Langevin thermostat with a coefficient of 1 ps$^{-1}$. The pressure 
of the box is maintained at 1 bar using the Langevin piston method with a damping coefficient of 20 
ps$^{-1}$ (NPT ensemble)~\cite{Feller1995}. The ion is restrained at the centre of the box with a spring 
constant of 10~\spring\ to prevent the ion drifting to the boundary. Both the forward and backward FEP 
transformations are combined with the Bennett acceptance ratio (BAR) to get the final free energy 
implemented using the VMD \verb+ParseFEP+ plugin~\cite{Liu2012}. 

\section{Results and Discussion}
\subsection{Ion Solvation}
Solvation free energies of ions have been measured using different methods, and therefore exhibit some 
variation, as shown in \tabref{solv:tab1}. The sum of the solvation free energy for a cation--anion pair 
is determined from the heat of formation measurements, which are reasonably well known. The main source 
of the uncertainty comes the absolute solvation free energy of the proton, $\Delta G^{\circ} [{\rm H}^+]$, 
which is used as a reference in finding the individual solvation free energies of ions. For example, the 
large difference of the Gomer~\cite{Gomer1977} and Marcus~\cite{Marcus1994} values from the others is due 
to the use of substantially larger $\Delta  G^{\circ} [{\rm H}^+]$ values (11 and 12 kcal/mol, 
respectively) in these work compared to that of Tissandier et al.~\cite{Tissandier1998} The more recent 
$\Delta G^{\circ} [{\rm H}^+]$ value determined by Tissandier et al.~\cite{Tissandier1998} 
($-264$~kcal/mol) is generally considered more accurate and adapted in other work~\cite{Kelly2006}. 
The $+$1.9 kcal/mol difference between the values of Tissandier et al. and Kelly et al. is simply due 
to the choice of the standard state---in the former, a concentration of 1~bar in the gas phase and 1~M 
in the aqueous phase is used, whereas a concentration of 1~M is used in both phases in the latter. Because 
the latter condition is assumed in the simulations, we will use the solvation free energies from Kelly 
et al.~\cite{Kelly2006} in comparisons with the calculated values below.

\begin{table}[b!]
\caption{\label{solv:tab1}A compilation of experimental solvation free energies of
\Na\ and \Cl\ from various sources.}
\resizebox{\textwidth}{!}{\begin{tabular}{ccccccc}
\hline
& Noyes~\cite{Noyes1962} & Gomer~\cite{Gomer1977} & Klots~\cite{Klots1981}
& Marcus~\cite{Marcus1994} & Tissan.~\cite{Tissandier1998} & Kelly~\cite{Kelly2006} \\ 
\hline
 \Na & $-97.0$ & $-90.6$ & $-100.1$ & $-87.2$ & $-101.3$ & $-103.2$ \\
 \Cl & $-74.8$ & $-81.4$ & $ -73.9$ & $-81.3$ & $ -72.7$ & $-74.5$ \\
 Sum    & $-171.8$ & $-172.0$ & $-174.0$ & $-168.5$ & $-174.0$ & $-177.7$ \\ \hline
\end{tabular}}
\end{table}

In order to discuss the issues with using small water droplets in solvation free energy calculations 
and motivate the proposed method, we first consider the variation of the average charge density around 
the \Na\ and \Cl\ ions in a water droplet (\figref{solv:fig3}). The charge densities are obtained 
from 30~ns MD simulations of a central ion in a water droplet with $R=24$~\angs. It is seen that the 
effect of both ions on the charge density in the water extends up to 9~\angs\ from the centre. The charge 
density then remains flat until about 6~\angs\ from the boundary, where it again exhibits variations due 
to the ordering of water molecules by the vacuum interface. For water droplets with $R<15$~\angs, the two 
regions overlap, and the water molecules in the inner shells are affected by vacuum. This perturbs the 
ion-water interactions in the inner shells, which are clearly not acceptable. Thus the use of angular 
restraints to mimic the polarisation of water molecules in a larger system is necessary when small 
droplets with $R<15$~\angs\ are employed~\cite{King1989}. Use of such small systems was unavoidable 2-3 
decades ago due to the limited computational resources. However, this is no more the case, and one can 
avoid the overlap problem by using droplets with $R>15$~\angs.

\begin{figure}[b!]
\centering
\includegraphics[width=0.65\textwidth]{Figures/Solvation/fig3.jpg}
\caption{The radial dependence of the charge density in a water droplet of size $R=24$~\angs\ 
for \Na\ (black) and \Cl\ (red) ions. }
\label{solv:fig3}
\end{figure}

\begin{table}[t!]
\caption{\label{solv:tab2} Break down of the solvation free energies as a function of the droplet 
size $R$ for both \Na\ and \Cl. Water molecules are confined using a half-harmonic potential 
with $k_{s}=10$~\spring, and the ion is restrained at the centre with 10~\spring.  
All energies are reported in kcal/mol, and the maximum statistical error for the solvation free energies 
is 0.3~kcal/mol.}
\resizebox{\textwidth}{!}{\begin{tabular}{ccccccccccc}
\hline
& \multicolumn{5}{c}{\Na} & \multicolumn{5}{c}{\Cl} \\
$R$ (\angs)& $\Delta G_{\text{drop-el}}$ & $\Delta G_{\text{cav}}$ & $\Delta 
G_{\text{el}}$ & $\Delta G_{\text{LJ}}$ & $\Delta G_{\text{solv}}$ & $\Delta 
G_{\text{drop-el}}$ & $\Delta G_{\text{cav}}$ & $\Delta G_{\text{el}}$ & 
$\Delta G_{\text{LJ}}$ & $\Delta G_{\text{solv}}$ \\ \hline
 6 & $-81.7$ & $-27.4$ & $-109.1$ & $3.2$ & $-105.9$ & $-60.2$ & $-27.4$ & 
$-87.6$ & $ 6.0$ & $-81.6$ \\ 
 9 & $-88.2$ & $-18.2$ & $-106.4$ & $2.8$ & $-103.6$ & $-68.0$ & $-18.2$ & 
$-86.2$ & $ 5.6$ & $-80.6$ \\ 
12 & $-91.5$ & $-13.6$ & $-105.1$ & $2.7$ & $-102.4$ & $-72.3$ & $-13.6$ & 
$-86.0$ & $ 5.3$ & $-80.7$ \\
15 & $-93.6$ & $-10.9$ & $-104.5$ & $2.6$ & $-101.9$ & $-74.7$ & $-10.9$ & 
$-85.6$ & $ 5.2$ & $-80.4$ \\ 
18 & $-95.1$ & $ -9.1$ & $-104.2$ & $2.5$ & $-101.7$ & $-76.3$ & $ -9.1$ & 
$-85.4$ & $ 5.0$ & $-80.4$ \\ 
21 & $-96.0$ & $ -7.8$ & $-103.8$ & $2.6$ & $-101.2$ & $-77.6$ & $ -7.8$ & 
$-85.4$ & $ 5.0$ & $-80.5$ \\
24 & $-96.9$ & $ -6.8$ & $-103.7$ & $2.5$ & $-101.2$ & $-78.5$ & $ -6.8$ & 
$-85.3$ & $ 4.7$ & $-80.6$ \\ 
30 & $-97.9$ & $ -5.5$ & $-103.4$ & $2.4$ & $-101.0$ & $-79.9$ & $ -5.5$ & 
$-85.4$ & $ 4.5$ & $-80.8$ \\
36 & $-98.9$ & $ -4.5$ & $-103.4$ & $2.4$ & $-101.0$ & $-80.5$ & $ -4.5$ & 
$-85.0$ & $ 4.4$ & $-80.7$ \\ \hline
\end{tabular}}
\end{table}

For larger droplets with $R > 15$~\angs, we only need to worry about the artefacts introduced by the 
variations in the charge density near the boundary. Integration of the charge density at the boundary 
region shows that there is a net charge given by $(1-1/\epsilon)q$, consistent with the continuum 
calculations. This monopole contribution to the ion's interaction energy is cancelled by the cavity 
term, which leaves the dipole and higher-order interactions behind. Because these terms fall as $1/R^2$ 
or faster, it should be possible to make them negligibly small by choosing a large enough $R$. To this 
end, we investigate the convergence of the solvation free energies with $R$ by varying it from 6 to 
36~\angs\ for both \Na\ and \Cl. The results are shown in \tabref{solv:tab2}, where individual 
contributions to the free energies are listed separately. For both ions, the convergence of $\Delta 
G_{\text{solv}}$ occurs around $R=21$~\angs. The $\Delta G_{\text{solv}}$ values for $R \geq 21$~\angs\ 
differ from that at $R=36$~\angs\ by 0.1--0.2 kcal/mol, which is within the statistical errors in the 
calculations. Inspection of the individual contributions in \tabref{solv:tab2} shows that cancellation 
of the $1/R$ dependent term in $\Delta G_{\text{drop-el}}$ with $\Delta G_{\text{cav}}$ ensures the 
fast convergence of the results observed in $\Delta G_{\text{el}}$.

The converged $\Delta G_{\text{solv}}$ values in \tabref{solv:tab2} are in good agreement with previous 
calculations where the spherical solvent boundary potential method~\cite{Beglov1994} was used with the 
CHARMM parameters to obtain  $-101.1$~kcal/mol~\cite{Lin2014a} for \Na\ and 
$-80.0$~kcal/mol~\cite{Roux1996} for \Cl. Comparison of the calculated $\Delta G_{\text{solv}}$ 
values with the experimental ones from Kelly et al. in \tabref{solv:tab1} yields mixed results. There 
is a small deviation for \Na\ because the earlier experimental values were targeted in the calculations 
where \Na\ parameters were determined. A larger discrepancy (6~kcal/mol) occurs for \Cl. Presumably, 
\Cl\ suffers from the neglected cousin syndrome as it plays a relatively minor role in biochemical 
processes compared to \Na. But its proper parametrisation should nevertheless be considered in future 
versions of the force field.

We next demonstrate the robustness of the results by showing that they are not affected by the 
parameters of the potentials used in confining the water molecules in the droplet and restraining 
the ion at the centre. For the confining potential, we repeat the previous calculations with 
$k_{s}=10$~\spring\ using $k_{s}=5$ and 20~\spring. The solvation free energy results are shown in 
\figref{solv:fig4} for both \Na\ and \Cl. The solvation free energy differs by a maximum of 
0.3 kcal/mol between the three $k_{s}$ values, which is within the statistical errors. Statistical 
nature of the variations is also apparent from the fact that there is no clear trend in the $\Delta 
G_{\text{solv}}$ values with increasing $k_{s}$. Thus the strength of the confinement potential does 
not have an appreciable effect on the stability of the results. The value of 10~\spring\ is sufficient 
to confine the water molecules inside the droplet and will be used in the following calculations. 
The second parameter used in the calculations is the harmonic restraint that keeps the ion at the 
centre of the droplet. We test the use of different centre of charge restraints for \Na\ only, and 
the results are summarised in \tabref{solv:tab3}. Overall, the solvation energy differs by only 
0.1--0.3~kcal/mol. The largest deviation occurs for the 0.1~\spring\ restraint, which is presumably 
a consequence of under-sampling due to the larger phase-space resulting from the weaker restraint.

\begin{table}[t!]
\caption{\label{solv:tab3} The effect of the harmonic restraint (\spring) applied 
to the \Na\ ion on its solvation free energy. Calculations are performed in 
a droplet with $R=24$~\angs. All energies are reported in kcal/mol.}
\centering
\begin{tabular}{ccccccc}
\hline
$k$ & $\Delta G_{\text{drop-el}}$ & $\Delta G_{\text{cav}}$ & $\Delta 
G_{\text{el}}$ & $\Delta G_{\text{LJ}}$ & $\Delta G_{\text{solv}}$ \\ \hline
 0.1 & $-97.0$ & $-7.0$ & $-104.0$ & $2.5$ & $-101.5$ \\ 
 0.5 & $-96.8$ & $-6.9$ & $-103.7$ & $2.5$ & $-101.2$ \\ 
 1.0 & $-97.0$ & $-6.8$ & $-103.8$ & $2.5$ & $-101.3$ \\
10.0 & $-96.9$ & $-6.8$ & $-103.7$ & $2.5$ & $-101.2$ \\ \hline
\end{tabular}
\end{table}

For a molecule with distributed partial charges, most of the charges will necessarily be at off-centre 
positions when the centre of the charge is restrained at the origin. In order to ensure that the 
proposed method can be applied to molecules, we need to show that its accuracy is maintained for 
off-centre positions of charges, and find the maximum distance of charges from the centre for which 
the method still gives an acceptable result. Such information will be useful for choosing an appropriate 
system size when larger molecules are considered without the need to check the system size dependence 
for each molecule. We place the ion at off-centre positions by applying a harmonic restraint of 
10~\spring\ along the $z$-axis. Calculations are performed using a droplet size of 
$R=24$~\angs\ and the results are shown in \tabref{solv:tab4}. We see that the converged value of the 
solvation free energy is maintained up to 4~\angs\ from the centre. The maximum change in the energy is 
about 0.3~kcal/mol for \Na\ and 0.1~kcal/mol for \Cl, which are within the statistical errors. At 
8~\angs\ the ion solvation free energy is still within 1~kcal/mol, but at 12~\angs\ the inner and outer 
shells start overlapping (\figref{solv:fig3}), which has a larger effect on the solvation free energy. 
These results suggest that we can use a droplet size of $R=24$~\angs\ to determine the solvation free 
energy accurately for solute molecules up to 8~\angs\ in length. Even for molecules of length 16~\angs, 
the expected accuracy is about 1~kcal/mol. Thus using a droplet with radius 24~\angs\ should be adequate 
for the charged amino acid side chain calculations performed in \secref{sec:chargedamino}.

\begin{figure}[t!]
\centering
\includegraphics[width=0.6\textwidth]{Figures/Solvation/fig4.jpg}
\caption{Solvation free energy as a function of the droplet radius $R$ for 
         \Na\ (top) and \Cl\ (bottom). Three sets of calculations are 
         compared in the plot, corresponding to the three different spring 
         constants used for the confinement of water molecules, 
         $k_{s}=\{5, 10, 20\}$~\spring.}
\label{solv:fig4}
\end{figure}

\begin{table}[b!]
\caption{Same as \tabref{solv:tab2} but as a function of the ion 
         off-centre positions along the $z$-axis with a droplet 
         size of $R=24$~\angs.} 
\label{solv:tab4}
\resizebox{\textwidth}{!}{\begin{tabular}{ccccccccccc}
\hline
& \multicolumn{5}{c}{\Na} & \multicolumn{5}{c}{\Cl} \\
$z$ (\angs)& $\Delta G_{\text{drop-el}}$ & $\Delta G_{\text{cav}}$ & $\Delta 
G_{\text{el}}$ & $\Delta G_{\text{LJ}}$ & $\Delta G_{\text{solv}}$ & $\Delta 
G_{\text{drop-el}}$ & $\Delta G_{\text{cav}}$ & $\Delta G_{\text{el}}$ & 
$\Delta G_{\text{LJ}}$ & $\Delta G_{\text{solv}}$ \\ \hline 
 0 & $-96.9$ & $-6.8$ & $-103.7$ & $2.5$ & $-101.2$ & $-78.5$ & $-6.8$ & 
$-85.3$ & 4.7 & $-80.6$ \\
 1 & $-96.8$ & $-6.8$ & $-103.6$ & $2.5$ & $-101.2$ & $-78.5$ & $-6.8$ & 
$-85.4$ & 4.6 & $-80.7$ \\ 
 2 & $-97.0$ & $-6.9$ & $-103.8$ & $2.5$ & $-101.3$ & $-78.7$ & $-6.9$ & 
$-85.5$ & 4.8 & $-80.8$ \\ 
 4 & $-96.9$ & $-7.0$ & $-104.0$ & $2.4$ & $-101.5$ & $-78.5$ & $-7.0$ & 
$-85.5$ & 4.7 & $-80.8$ \\ 
 8 & $-96.8$ & $-7.7$ & $-104.5$ & $2.5$ & $-101.8$ & $-78.5$ & $-7.7$ & 
$-86.2$ & 4.6 & $-81.6$ \\ 
12 & $-97.0$ & $-9.1$ & $-106.1$ & $2.5$ & $-103.6$ & $-78.7$ & $-9.1$ & 
$-87.8$ & 4.8 & $-83.0$ \\ \hline
\end{tabular}}
\end{table}

A droplet of size 24~\angs\ contains about 2000 water molecules, which is not a large system for today's 
standards. However, repeated calculations of solvation free energies for parameter fitting purposes can 
still turn this into a computationally expensive exercise. This is one of the main reasons why a much 
smaller droplet radius has been used in previous calculations of solvation free energies. Nevertheless, 
running these calculations on a GPU can significantly reduce the computation time. This is made possible 
through GPU optimised codes like OpenMM. For example, for a droplet with radius 21~\angs, MD simulations 
can be run at about 800 ns/day on a Tesla V100 GPU. Increasing the radius to 24~\angs, one can still get 
600 ns/day, which is sufficient to perform 13 solvation free energy calculations per day. On a consumer-grade 
GPU card (GTX 1080 Ti), the corresponding run times are 570 and 450~ns/day for 21~\angs\ and 
24~\angs\ droplets, respectively. Thus, even using a system size of 24~\angs, 10--13 calculations can be 
performed per day on a GPU, which should be sufficient for optimisation of force fields. These numbers 
are for 1 ns production time per window, which is probably excessive. For large scale calculations, 
production times can be optimised which will shorten the computation time further.

Arguments made for the convergence of solvation free energies using SBC can also be made for PBC. Thus 
it is of interest to repeat the calculations for the \Na\ and \Cl\ ions using PBC with a variable box 
size. The results are presented in \tabref{solv:tab5} for box sizes varying from 20 to 60~\angs. 
The convergence of the solvation free energies occurs around $L = 35$~\angs, which has a similar volume 
to a droplet with $R = 21$~\angs, where the convergence occurs in a droplet. Thus at those volumes, boundary 
artefacts cease to contribute to the solvation free energy of the ion regardless of the specific geometry 
used. Comparing the SBC and PBC results in \tabrefn{solv:tab2} and~\ref{solv:tab5}, we see that there is 
very good agreement between the two methods for \Na, but they differ by about a kcal/mol for \Cl.

\begin{table}[b!]
\caption{\label{solv:tab5} Break down of the solvation free energies as a function of the 
box length $L$ for both \Na\ and \Cl. $\Delta G_{\text{solv}}^{\text{real}}$ is obtained 
from $\Delta G_{\text{solv}}^{\text{intr}}$ by adding $−12.0z$ kcal/mol for the interfacial 
potential. All energies are reported in kcal/mol, and the maximum statistical error for the 
solvation free energies is 0.3 kcal/mol.}
    \centering
    %\resizebox{\textwidth}{!}{
    \begin{adjustbox}{width=1.0\textwidth,center=\textwidth}
    \begin{tabular}{ccccccccccccc}
    \hline
     & \multicolumn{6}{c}{\Na} & \multicolumn{6}{c}{\Cl} \\
$L$ (\angs)& $\Delta G_{\text{box-el}}$ & $\Delta G_{\text{SE}}$ & $\Delta 
G_{\text{el}}$ & $\Delta G_{\text{LJ}}$ & $\Delta G_{\text{solv}}^{\text{intr}}$ &  $\Delta G_{\text{solv}}^{\text{real}}$ & $\Delta 
G_{\text{box-el}}$ & $\Delta G_{\text{SE}}$ & $\Delta G_{\text{el}}$ & 
$\Delta G_{\text{LJ}}$ & $\Delta G_{\text{solv}}^{\text{intr}}$ &  $\Delta G_{\text{solv}}^{\text{real}}$ \\ \hline
 20 & -67.8 & -23.5 & -91.3 & 2.0 & -89.2 & -101.2 & -71.8 & -23.5 & -95.3 & 3.3 & -92.0 & -80.0 \\
 25 & -72.3 & -18.8 & -91.1 & 2.0 & -89.1 & -101.1 & -76.5 & -18.8 & -95.3 & 3.4 & -91.8 & -79.8 \\
 30 & -75.5 & -15.7 & -91.1 & 2.0 & -89.2 & -101.2 & -79.3 & -15.7 & -95.0 & 3.6 & -91.4 & -79.4 \\
 35 & -77.5 & -13.4 & -90.9 & 2.0 & -88.9 & -100.9 & -81.6 & -13.4 & -95.0 & 3.6 & -91.4 & -79.4 \\
 40 & -79.2 & -11.8 & -90.9 & 2.0 & -88.9 & -100.9 & -83.2 & -11.8 & -95.0 & 3.6 & -91.4 & -79.4 \\
 50 & -81.6 &  -9.4 & -91.0 & 2.0 & -89.0 & -101.0 & -85.4 &  -9.4 & -94.8 & 3.4 & -91.5 & -79.5 \\
 60 & -83.0 &  -7.8 & -90.8 & 2.0 & -88.9 & -100.9 & -87.1 &  -7.8 & -94.9 & 3.5 & -91.4 & -79.4 \\ \hline
    \end{tabular}
    \end{adjustbox}
    %}
\end{table}

\subsection{Charged Amino Acids}
\label{sec:chargedamino}
As an application of the method, we consider the solvation free energies of the side chain analogues 
for the four charged amino acids, which play important roles in protein interactions. Calculations 
are performed using a droplet of size $R=24$~\angs\ and the same confinement potential with 
$k_{s}=10$~\spring. The centre of charge of the molecules is restrained at the centre 
with 10~\spring. The results are listed in \tabref{solv:tab6}, together with the experimental 
values~\cite{Reif2012,Kelly2006}. To ensure that convergent results are also obtained for molecules, 
we have repeated the acetate calculations using droplet sizes $R=36$ and 48~\angs. The results differed 
from that of $R=24$~\angs\ by at most 0.1~kcal/mol, which is within the statistical errors. We have also 
performed the acetate calculations using PBCs with box sizes varying from 20 to 60~\angs\ and obtained 
a converged value of $−84.7$ kcal/mol. As in the case of \Cl, there is about a kcal/mol difference 
between the SBC and PBC results. This is presumably caused by the interfacial potential, which shows 
small variations depending on the model used in its computation. One consequence of using a relatively 
large droplet is that the multipole contributions to the cavity term envisaged in Eq.~\eqref{eq:mse} 
become negligibly small so that this term is essentially given by the Born energy in Eq.~\eqref{eq:born}.

Comparing the calculated solvation free energies with the experimental values in \tabref{solv:tab6}, 
it is seen that good agreement is obtained for the arginine analogue guanidinium. For the three lysine 
analogues listed, the discrepancy between the experimental and calculated values varies from 
4--6~kcal/mol. The solvation free energies are systematically under-predicted, suggesting that the 
ammonium partial charges need to be boosted slightly in the lysine analogues. The discrepancy grows 
to about 9~kcal/mol for the aspartate and glutamate side-chain analogues. The solvation free energies 
are over predicted in both cases. Thus the partial charges on the O atoms of these side chains need 
to be reduced to improve the agreement with the experiments. We note that in earlier calculations, 
the experimental values from Sitkoff et al.~\cite{Sitkoff1994} were targeted where 
$\Delta G^{\circ} [{\rm H}^+] =-261$~kcal/mol is used. This shifts the solvation free energies by 
3 kcal/mol up for the positive side chains and down for the negative ones, making the calculated 
lysine results more agreeable with experiment but sizeable discrepancies remain for the negative 
side chains. To indicate the size of the correction needed, we have performed the acetate calculation 
using the optimised OPLS-AA force field~\cite{Jensen2008}, which has a lower O charge ($-0.67e$ 
compared to $-0.76e$ in CHARMM). As seen in \tabref{solv:tab6}, the discrepancy is reduced to about 
4~kcal/mol, so there is further room for improving the aspartate and glutamate charges.

\begin{table}[t!]
\caption{Solvation free energies of the charged amino-acid side chain 
analogues obtained using a droplet size of $R=24$~\angs. All energies are in 
kcal/mol, and the maximum statistical error for the solvation free energies is 
0.3~kcal/mol.}
\label{solv:tab6}
\resizebox{\textwidth}{!}{
\begin{tabular}{clcccccc}
\hline
Amino acid & \multicolumn{1}{c}{Analogue} & $\Delta G_{\text{drop-el}}$ & 
$\Delta G_{\text{cav}}$ & $\Delta G_{\text{el}}$ & $\Delta G_{\text{LJ}}$ & 
$\Delta G_{\text{solv}}$ &  Expt.$^{a}$ \\ 
\hline 
\multicolumn{ 1}{l}{arginine}  
&guanidinium&        $-62.9$ & $-6.8$ & $-69.7$ & 1.4 & $-68.3$ & $-67.3$  \\ 
&methyl-guanidinium& $-57.1$ & $-6.8$ & $-64.0$ & 1.9 & $-62.1$ &      \\ 
&ethyl-guanidinium&  $-56.7$ & $-6.8$ & $-63.5$ & 2.5 & $-61.0$ &      \\ 
&n-propyl-guanidinium& $-56.6$ & $-6.8$ & $-63.4$ & 3.3 & $-60.1$ &    \\[1mm]
\multicolumn{ 1}{l}{lysine}    
& ammonium & $-75.8$ & $-6.8$ & $-82.6$ & 2.5 & $-80.1$ & $-85.2$\\ 
& methyl-ammonium & $-66.4$ & $-6.8$ & $-73.2$ & 2.9 & $-70.3$ & 
$-76.4$ \\ 
& ethyl-ammonium & $-65.0$ & $-6.8$ & $-71.8$ & 3.2 & $-68.6$    &  \\ 
& n-propyl-ammonium & $-64.3$ & $-6.8$ & $-71.1$ & 3.8 & $-67.3$ &  $-71.5$ \\ 
& n-butyl-ammonium & $-64.4$ & $-6.8$ & $-71.2$ & 4.4 & $-66.8$  & \\[1mm]
%\multicolumn{ 1}{l}{histidine} & imidazolium    & 
% $-54.1$ & $-6.8$ & $-60.9$ & $2.4$ & $-58.5$ &  & $-63.1$ \\
%                & methly-imidazolium   & 
% $-51.0$ & $-6.8$ & $-57.8$ & $3.7$ & $-54.1$ &  & $-64.1$   \\
\multicolumn{ 1}{l}{aspartate} 
& acetate  & $-82.9$ & $-6.8$ & $-89.7$ & 3.6 & $-86.1$ & $-77.6$\\
& acetate (OPLS-AA)$^{b}$ 
           & $-76.8$ & $-6.8$ & $-83.6$ & 1.9 & $-81.7$  &   \\[1mm]
\multicolumn{ 1}{l}{glutamate} 
&propanoate& $-82.3$ & $-6.8$ & $-89.2$ & 4.1 & $-85.1$ & $-76.2$\\ \hline
\end{tabular}}
%\centering
{\footnotesize $^{a}$All experimental values are from Kelly et al.~\cite{Kelly2006}
except for guanidinium which is taken from Reif et al.~\cite{Reif2012} The 
guanidinium value is adjusted to make it consistent with the Kelly et al. 
set~\cite{Kelly2006}, i.e., 1.9~kcal/mol is subtracted for conversion of the state 
from 1~bar to 1~M in the gas phase and 1.1~kcal/mol is subtracted to shift the 
$\Delta G^{\circ} [{\rm H}^+]$  value to that of Tissandier et al.~\cite{Tissandier1998}. \\
$^{b}$Acetate parameters from Ref.~\cite{Jensen2008}
(q,$\sigma$,$\epsilon$): 
O=($-0.67e$, 2.96~\angs, 0.21~kcal/mol), 
C=($0.44e$, 3.75~\angs, 0.105~kcal/mol).}
\end{table}

The above observations about the charges on the side chains of the amino acids Arg, Lys, Asp, and Glu 
are consistent with the observed behaviour of the side-chain interactions during dissociation of peptide 
ligands from proteins~\cite{Rashid2014,Rashid2012}. The N--O distances between two side chains, monitored 
during potential of mean force calculations, reveal that Arg--Asp and Arg--Glu contacts can persist beyond 
10~\angs\ but such an unexpected behaviour does not occur for Lys contacts. The persistence of the former 
contacts can be explained by the overcharging of the O atoms in the Asp and Glu side chains which, 
therefore can stick to the Arg side chain longer than expected. The undercharging of the Lys side chain 
apparently counters the stickiness of the Asp and Glu side chains so that such a persistence of contacts 
is not seen in Lys interactions. 

\section{Conclusion}
A spherical droplet provides the simplest and physically most transparent system for calculation of 
solvation free energies of ions. We have shown that artefacts arising from the boundary with vacuum 
become negligibly small for droplet sizes of $R=21$~\angs\ so that one can obtain uniquely convergent 
solvation free energies independent of the parameters used for confining water in the droplet or 
restraining the ion at the centre. The computational effort required is rather small for a single 
calculation, and the method can be easily upscaled for optimisation of force fields using GPUs and 
clusters.

Application of the method to the calculation of solvation free energies of ions and the side chain analogues 
of charged amino acids have revealed discrepancies with experiments in several cases. A sizeable portion 
of the discrepancies is caused by the older experimental values used in fitting the force fields, which 
have been shifted in the new compilations by the choice of $\Delta G^{\circ} [{\rm H}^+]$~\cite{Tissandier1998,Kelly2006}. 
The free energy calculations can be performed with chemical accuracy nowadays, but this would be of little 
value if the results are affected by inaccuracies in force fields. So it is important to optimise force 
fields using the latest solvation free energies. The uniquely convergent solvation free energies obtained 
using our method could help to make this process more robust.
%=======================================================================================%