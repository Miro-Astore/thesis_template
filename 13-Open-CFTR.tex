%=======================================================================================%
\chapter{Resolving a Conducting Conformation of CFTR Using Free Energy Calculations}
\label{chap:opening}
\chapquote{No hats for soup} {-Benjamin John Goodwin}
ABSTRACT 
Understanding how CFTR conducts ions is critical to ongoing drug discovery efforts to treat Cystic Fibrosis. Existing structures of CFTR  raise unresolved qustions as to how CFTR conducts ions as they exhibit a constriction smaller than the ions themselves. This indicates that there must be some level of conformational change for ions to pass through this structure. Here we present innovative simulation techniques combining simple numerical techniques and new advances in free energy calculations to resolve the full conduction pathway in CFTR. We hope that the techniques and philosophy in this chapter will be refined and applied to other complex protein systems of physiological relevance. 
\newline

\section{Discussion}

The present study diverges in an important way from existing free energy calculations which investigate protein conformational changes in the literature. Present studies have focussed on recreating intermediate free energy landscapes between \textit{known} endpoints \cite{lev2020, bergh2021}. Such approaches are critical to the developement of molecular techniques in order to understand the energetics and kinetics of molecular machines. However, these studies are inherently limited to the availability of high quality experimental 3d structures. Protein forcefields may have their issues but they are now of sufficient quality that they can now be used, alongside considerable computer power to investigate parts of the conformational landscape which have critical functional roles but are not covered by experimental structures. This is akin to the developement of an unsupervised vs. supervised machine learning algorithms. Each approach is powerful but has its own domain of applicability and drawbacks. 

As shown by the results in the present study the difficulty in converging a free energy landscape with collective variables derived \textit {ab initio} from long classical MD simulations can be difficult as the CVs are very likely going to be suboptimal. Machine learning techniques, more sophisticated than the simple PCA algorithm used here would likely do a much better job of choosing quickly converging CVs. 
