%=======================================================================================%
\chapter{Escape of \Na\ from Na3 in \GltPh}
\label{chap:unbind}
ABSTRACT \newline
Glutamate/Aspartate transport through the mammalian excitatory amino acid transporters 
is coupled to the co-transport of three \Na, one \Hi\ and the counter-transport of \K. 
The archaeal homologue \GltPh\ is coupled to only the co-transport of three \Na. The 
first two \Na\ sites (Na1,Na2) were resolved from both the outward and inward the crystal 
structure. The third site (Na3) was determined through a combination of mutagenesis 
experiments and molecular dynamics simulations and later verified in the crystal structure 
of \GltTk. Previous studies using path-independent free energy methods have suggested 
unbinding of the last \Na\ from Na3 is the rate-limiting step. Here we use path-dependent 
methods to determine the path taken and energy required for \Na\ to leave the Na3 site to 
the bulk region to the bulk region. A detailed characterisation of structural changes of 
the binding pocket as \Na\ leaves the binding site are given. In addition, we estimate the 
release time of \Na\ using Smoluchowski mean-first-passage-time theory. Our observations 
of structural changes made along the path and estimated time of release are consistent 
with known experimental data. The results presented here indicate that the release of 
\Na\ in the third binding site is a slow process but not the rate-limiting step of the 
transport cycle. This investigation may shed light on the ligand release mechanism in 
the mammalian transporters.

\newpage
\section{Introduction}
The mammalian glutamate transporters, also known as the excitatory amino acid transporters 
(EAATs), are responsible for clearing excess glutamate released at the synapses. Disruption 
of EAATs can lead to increased concentrations of glutamate, causing excitotoxicity of receptors. 
Such effects have been associated with many pathological conditions, including Alzheimer's 
disease, cerebral ischemia and amyotrophic lateral sclerosis~\cite{Danbolt2001}. EAATs 
function by cycling between the extracellular and intracellular medium. In each cycle, EAATs 
bind and transport the substrate (Asp or Glu) across the membrane by coupling three \Na\ and 
one \Hi\ ions~\cite{Zerangue1996}. After the release of the ligands, one \K\ ion is 
counter-transported to complete the cycle. While much has been learned about the functional 
properties of glutamate transporters from mutagenesis experiments, a mechanistic understanding 
of the transport process was not possible in the absence of a crystal structure. 

The first crystal structure of aspartate/glutamate transporters was that of \GltPh\ from 
Pyrococcus horikoshii—an archaeal homologue of EAATs~\cite{Yernool2004}. In successive 
iterations of the \GltPh\ crystal structure, first, the binding sites for the substrate and two 
\Na\ ions (called Na1 and Na2) were resolved in the outward-facing (OF) state~\cite{Boudker2007}, 
followed by the determination of the inward-facing (IF)~\cite{Reyes2009}, and intermediate 
conformations~\cite{Verdon2012}. Physiological studies of \GltPh\ revealed that it transported 
Asp (and not Glu) by coupling three \Na\ ions, but without the co-transport of a \Hi\ ion and 
the counter-transport of a \K\ ion~\cite{Ryan2009,Groeneveld2010}. The third \Na\ site (called 
Na3) could not be resolved in the crystal structures. Several Na3 site have been proposed from 
electrostatic calculations~\cite{Holley2009} and molecular dynamics (MD) 
simulations~\cite{Bastug2012,Larsson2010,Huang2010}. The \Na\ coordination shell proposed in 
the last reference (T92, S93, N310, and D312 side chains and Y89 backbone) was consistent with 
the available mutagenesis data, and tests via the T92A and S93A mutations provided further 
support for this site is the Na3 site~\cite{Bastug2012}. This was later verified with the 
recent crystal structure of \GltTk\ that shows \Na\ bound to the Na3 site proposed earlier 
\cite{Guskov2016}.

Although the overall sequence identity between \GltPh\ and EAATs is low (37\%), the homology 
for the binding pocket is close to 60\%. In addition, the residues involved in ligand binding 
in \GltPh\ are conserved in EAATs~\cite{Arriza1997,Vandenberg2013,Yernool2004,Boudker2007}. 
Thus, \GltPh\ provides a good starting model for mechanistic studies of the transport mechanism 
in glutamate transporters. Several computational investigations of \GltPh\ have been performed 
so far which include: MD simulations of ligand binding and gating in the OF 
\cite{Huang2008,Huang2010,Shrivastava2008} and IF states~\cite{Zomot2013}, free energy 
calculations of ion and substrate binding in the OF~\cite{Larsson2010,Heinzelmann2011} and IF 
states \cite{Heinzelmann2013}, metadynamics simulations of substrate binding and release 
\cite{Grazioso2012}, and study of the transition from the OF to IF state using the anisotropic 
network models in combination with MD simulations~\cite{Jiang2011,Das2014}. A homology model 
for EAAT3 was recently constructed using the \GltPh\ structure, which provided further insights 
on the \K\ binding site~\cite{Heinzelmann2014} as well as elucidating the mechanism of \Hi\ 
transport~\cite{Heinzelmann2014a}.

Free energy calculations performed using the crystal structures of \GltPh\ indicate that the 
binding order of ligands is Na3, Na1, Asp, and Na2 in the OF state~\cite{Heinzelmann2011}, and 
they are released in the reverse order in the IF state~\cite{Heinzelmann2013}. While the predicted 
binding order is consistent with the recent experimental studies of \Na\ and Asp 
binding~\cite{Reyes2013b,Ewers2013,Hanelt2015}, there are sizeable discrepancies in the binding 
free energies. These discrepancies are addressed in the \chapref{chap:bind} where the issue lies 
in not considering an intermediate state (Na1\prim) in the free energy calculations. The transition 
of Na1\prim\ to Na3 requires conformational changes in the protein. Experimentally, substantial 
conformational changes are observed in the protein during ligand binding as large activation 
energies are obtained from calorimetric studies of \GltPh\ 
\cite{Reyes2013b,Ewers2013,Hanelt2015}. More direct experimental evidence for such conformational 
changes is provided by the comparison of the recent apo structures of \GltPh\ 
\cite{Verdon2014} and \GltTk~\cite{Jensen2013} with the fully bound \GltPh\ structures 
\cite{Boudker2007,Reyes2009}. The path-independent free energy methods are 
clearly inadequate for description of ligand binding where conformational changes occur, and one 
has to appeal to path-dependent methods to capture the effect of such changes on the free energy 
of a ligand along the reaction coordinate.

In this work, we attempt to describe the conformational changes that occur in the binding pocket 
during the release of the last \Na\ ion from the Na3 site and provide an estimate for its release 
time. We do not consider the other ligands because they are observed to unbind during microseconds 
simulations, indicating that their release time is fast~\cite{Zomot2013}. Using the crystal structure 
of the IF state of \GltPh~\cite{Reyes2009}, we determine the path taken by the \Na\ ion from the Na3 
site to the bulk region. We then perform umbrella sampling MD simulations to calculate the free-energy 
profile and the diffusion coefficient of the ion along this path. The free-energy profile and diffusion 
coefficient results are used in the mean first-passage solution of the Smoluchowski equation to get 
an estimate for the release time of the \Na\ ion. We also use the trajectory data from the umbrella 
sampling MD simulations to quantify the conformational changes that occur in the protein as the \Na\ 
ion moves from the Na3 site to bulk. Finally, we discuss the implications of our results for the 
transport mechanism in EAATs, which is much faster than that of \GltPh.

\section{Method}
\subsection{Model System and Simulation Details}
% System and MD
In this study, we use the crystal structure of \GltPh\ in the inward-facing closed conformation 
(PDB ID: 3KBC). The crystal structure consists of the protomer with two \Na\ and Asp bound. The 
third \Na\ was added to the Na3 site observed in \GltTk~\cite{Guskov2016}. We embed the \GltPh\ 
trimer in a 1-palmitoyl-2-oleoyl-phosphatidylethanolamine (POPE) phospholipid bilayer using the 
software VMD~\cite{Humphrey1996}. We then solvate this protein-lipid complex with 16204 water 
molecules along 34 \Na\ and 43 \Cl\ neutralising ions. After following the same standard 
equilibration protocol as in~\cite{Heinzelmann2013} we then remove the Na2, Asp and Na1 from the 
monomers simulating for 5 ns after the removal of each ligand. We further equilibrate the system 
for 50 ns without any restraints making sure the gate between HP1 and HP2 is open, and water 
molecules fill the binding pocket.

All MD simulations are performed using NAMD package (version 2.10)~\cite{Phillips2005} with the 
CHARMM force field~\cite{Klauda2012}. We use the NPT ensemble keeping the temperature constant 
at 300 K using the Langevin thermostat with a damping factor of 5 ps$^{-1}$. The pressure is 
maintained at 1 atm using the Langevin piston method with a damping factor of 20 ps$^{-1}$ 
\cite{Feller1995}. We utilise periodic boundary conditions, and electrostatic interactions are 
calculated using the particle-mesh Ewald (PME) method~\cite{Darden1993} without truncation. 
Non-bonded interactions are truncated at 12 \angs\ and replaced with a smooth switching function 
starting from 10 \angs. In all simulations, a time step of 2 fs is employed for the integrator.

\subsection{Umbrella Sampling and Free Energy Calculations}
% Reaction Coordinate Explanation
In order to perform path-dependent free energy calculations, we need to 
establish an appropriate reaction coordinate (RC) for the \Na\ ion from the Na3 
site to bulk. In most systems studied, the RC follows a straight line and can be 
aligned with one of the Cartesian axes, which simplifies the construction of 
the window positions for umbrella sampling. However, for the simulation system 
at hand, the path for the \Na\ ion follows a curved trajectory, and such a 
simple choice is not possible. To facilitate the construction of the umbrella 
windows, we exploit the presence of two intermediate binding sites along the RC to 
be denoted by Na1\prim\ and Na1\dprim\ (details of these sites are discussed in 
Results). The path between any two neighbouring sites is approximately linear, 
thus they provide an initial path to build the actual RC. Three such vectors are 
connecting Na3--Na1\prim, Na1\prim--Na1\dprim, and Na1\dprim--bulk. To 
find the RC, we first perform steered MD (SMD) simulations along the direction 
of the Na3--Na1\prim\ vector using a stiff-spring constant of 15~\spring\ and a 
relatively slow speed of 1~\angs/ns. A second SMD simulation with different 
initial 
conditions is observed to yield essentially the same path. Therefore, the 
average 
of the two trajectories is taken as the RC (more SMD trajectories would be 
included in the average if the paths were not similar).  The same procedure is 
used for the vectors between Na1\prim--Na1\dprim\ and Na1\dprim--bulk to obtain 
the complete RC. In the following, we will denote the curvilinear RC with 
$\zeta$ and its projection on the vectors between the binding sites by $z$. For 
convenience, umbrella potentials are constructed at 0.5~\angs\ intervals along the 
$z$-coordinate at points $z_i$, and applied at the corresponding points 
$\zeta_i$ along the RC. That is, each window experiences a harmonic potential in 
the form
\begin{equation}
 U = \frac{1}{2} k(\zeta - \zeta_i)^2,
 \label{na3:eq1}
\end{equation}
where $\zeta$ is obtained from the projection of the position of the \Na\ ion 
on to the RC.  A spring constant of 20~\spring\ is used in the direction of the 
RC (defined by the tangent to the RC at $\zeta_i$), but no restraints are 
applied to the ion in the orthogonal plane. The reason for using a relatively 
large spring constant is to facilitate the calculation of the 
diffusion coefficient of the ion (see the section below for more details). In 
order to keep the reference to the RC absolute, we apply small restraints to 
the C$\alpha$ atoms of the protein residues outside the binding pocket with a 
spring constant of 0.1~\spring. The decision to exclude the C$\alpha$ atoms in 
the binding pocket from restraining follows from test umbrella sampling calculations, 
which have shown that the free-energy profile is affected when these residues 
are restrained. The trajectories obtained from the umbrella sampling MD simulations 
in each window are unbiased and combined using the weighted histogram analysis method 
(WHAM) \cite{Kumar1992} to construct the free-energy profile between two binding sites. 
For the Na1\dprim\ $\rightarrow$ bulk transition, we apply a flat-bottom funnel potential 
to reduce the phase space in the bulk~\cite{Limongelli2013}. The funnel parameters 
employed are 12~\angs, 0.6 rad, 1~\angs\  and 1~\spring\ for $R_{\text{cyl}}$, 
$Z_{\text{cc}}$, $\alpha$ and $k_{\text{cyl}}$ respectively. The funnel potential 
is implemented as a \verb+tclForces+ script in NAMD (see \appref{apx:funnel} for 
the script).

\subsection{Diffusion Coefficient and Mean First-passage Time}
% Smoluchowski vs Kramer's
In a previous work, Kramer's rate theory was used to estimate the release time of the \Na\ ion 
from the Na3 binding site, assuming that the escape path can be represented by a single potential 
well~\cite{Heinzelmann2013}. As already alluded, the escape path is more complicated than a single 
well, so a more sophisticated treatment of the release time is required. Here we employ a method 
that uses the results from the path-dependent umbrella sampling simulations as input. The time scale 
for the release time of \Na\ is much larger compared to the  relaxation time of the ion's velocity 
correlations. This means that the inertial effects can be neglected and the motion of \Na\ inside 
\GltPh\ can be described by the Smoluchowski equation~\cite{Izrailev}. The mean first-passage time 
solution of the Smoluchowski equation is given by~\cite{Szabo1980}
\begin{equation}
 \tau(a\rightarrow b) = \int_{a}^{b} d\zeta 
 \frac{e^{W(\zeta)/\kT}}{D(\zeta)} 
 \times \int_{a}^{\zeta} d\zeta' e^{-W(\zeta')/\kT}.
 \label{na3:eq2}
\end{equation}
Here $W(\zeta)$ is the free-energy along the RC, $D(\zeta)$ is the path-dependent diffusion coefficient, 
and $a$ and $b$ denote the initial and final states, respectively, which we choose as the bottom 
of the well and top of the barrier in the free-energy profile. 

The diffusion coefficient is usually calculated using the velocity autocorrelation function (Green-Kubo 
relation), and the value obtained is isotropic. For path-dependent calculations, Woolf and Roux 
derived an expression for the diffusion coefficient by taking the Laplace transformation of the 
velocity autocorrelation function~\cite{Woolf1994}. The diffusion coefficient is obtained via a 
numerical extrapolation process, which is not straightforward. A more convenient expression was 
derived by Hummer, who showed that the position, instead of the velocity autocorrelation function, 
can be used in the calculation of the diffusion coefficient~\cite{Hummer2005}
\begin{equation}
 D(\zeta_i) = \frac{var(\zeta_i)}{\tau_i},
 \label{na3:eq3}
\end{equation}
where $var(\zeta_i)= \langle \delta\zeta^2 \rangle_i=\langle \langle\zeta^2\rangle - \langle\zeta
\rangle^2 \rangle_i$ is the variance of the ion position along the RC at the $i^{\rm th}$ window,
and $\tau_i$ is the characteristic time of the normalised autocorrelation function of $\zeta$ at 
the $i^{\rm th}$ window 
\begin{equation}
 \tau_i = \frac{\int_{0}^{\infty} \langle \delta \zeta(t).
 \delta \zeta(0) \rangle_i dt} {var(\zeta_i)}.
 \label{na3:eq4}
\end{equation} 
This expression is valid provided the system behaves like an overdamped harmonic oscillator. This 
behaviour can be enforced on the ion by using a sufficiently strong spring constant in the direction 
of the RC. We find that $k=20$~\spring\ is sufficient for this purpose.  For smaller values of $k$, 
the autocorrelation function does not behave like an exponentially decaying function. The integration 
of the position autocorrelation function is done up to 1~ps using a total of 4~ns trajectory data for 
the ensemble average (\appref{apx:diff} list the Fortran source code).

\section{Results and Discussion}
\subsection{Escape Path of \Na\ from the Na3 Site}
% Initial path search
In order to search for potential binding sites, the \Na\ ion is steered from the Na3 site towards 
the Na1 site in SMD simulations. Several SMD simulations are performed with different initial 
conditions to ensure adequate sampling. At the end of the steering, we allow the system to relax 
by performing 10 ns MD simulation. At the end of the equilibrium simulation, we find a stable 
site between the Na3 and Na1 sites, which will be referred to as Na1\prim. SMD simulations between the 
Na3 and the Na1\prim\ sites are repeated to make sure that the proposed Na1\prim\ site is unique. 
The Na1\prim\ site is coordinated by both the D312 and D405 side chains (see \tabref{na3:tab1} 
for the coordinating residues and the average \Na--O distances). A similar intermediate state was 
also observed in MD simulations of the OF conformation of \GltPh{}~\cite{Huang2010}. We note that, 
as in the Na3 site, there are no water molecules in the coordination shell of \Na\ in the Na1\prim\ 
site. 

% Coordinating residues and distances
\begin{table}[b!] 
\caption{\label{na3:tab1}Residues coordinating the \Na\ ion in the Na3, Na1\prim, and 
Na1\dprim\ sites, and at the transition states (TS) in between.$^{a}$}  
\begin{center}
\resizebox{\textwidth}{!}{
\begin{tabular}{lcccccc}
\hline
Helix-Residue & Na3 & TS1  & Na1\prim\  &  TS2  & Na1\dprim\ & TS3 \\ \hline
TM3--Y89 (O)   & 2.3 $\pm$ 0.1 & 5.2 $\pm$ 0.2 & - & - & - & - \\ 
TM3--T92 (OH)  & 2.4 $\pm$ 0.1 & 3.8 $\pm$ 0.2 & - & - & - & - \\ 
TM3--S93 (OH)  & 2.4 $\pm$ 0.1 & 5.0 $\pm$ 0.3 & - & - & - & - \\ 
TM7--D312 (O1) & 3.5 $\pm$ 0.2 & 2.4 $\pm$ 0.1 & 2.3 $\pm$ 0.1 & 2.2 $\pm$ 0.1 & - & - \\ 
TM7--D312 (O2) & 2.1 $\pm$ 0.1 & 2.6 $\pm$ 0.2 & 2.3 $\pm$ 0.1 & 3.2 $\pm$ 0.5 & - & - \\
TM7--N310 (O$_{\delta}$) & 2.2 $\pm$ 0.1 & 2.2 $\pm$ 0.1 & 2.3 $\pm$ 0.1 & 2.7 $\pm$ 0.3 & 2.3 $\pm$ 0.1 & 8.6 $\pm$ 0.4 \\ 
TM7--N310 (O)  & 6.2 $\pm$ 0.2 & 5.0 $\pm$ 0.3 & 6.3 $\pm$ 0.2 & 5.9 $\pm$ 0.3 & 5.2 $\pm$ 0.6 & 7.0 $\pm$ 0.8 \\ 
TM8--N401 (O)  & - & 4.8 $\pm$ 0.2 & 2.4 $\pm$ 0.1 & 2.4 $\pm$ 0.2 & 2.7 $\pm$ 0.4 & 8.7 $\pm$ 1.0 \\ 
TM8--D405 (O1) & - & 4.4 $\pm$ 0.2 & 3.9 $\pm$ 0.2 & 3.0 $\pm$ 0.4 & 2.2 $\pm$ 0.1 & 7.4 $\pm$ 1.1 \\
TM8--D405 (O2) & - & 5.9 $\pm$ 0.2 & 2.2 $\pm$ 0.1 & 2.2 $\pm$ 0.1 & 2.3 $\pm$ 0.1 & 7.0 $\pm$ 1.3 \\ 
TM7--G306 (O)  & - & - & - & 4.1 $\pm$ 0.5 & 2.5 $\pm$ 0.3 & 4.9 $\pm$ 1.6 \\ 
TM7--A307 (O)  & - & - & - &  & 5.7 $\pm$ 0.4 & 3.6 $\pm$ 1.2 \\ 
H2O (1) & - & - & - & - & 2.4 $\pm$ 0.3 & 2.4 $\pm$ 0.2 \\ 
H2O (2) & - & - & - & - & - & 2.3 $\pm$ 0.1 \\
H2O (3) & - & - & - & - & - & 2.4 $\pm$ 0.3 \\
H2O (4) & - & - & - & - & - & 2.6 $\pm$ 0.5 \\
H2O (5) & - & - & - & - & - & 2.7 $\pm$ 0.5 \\
\hline
\end{tabular}}
\end{center}
\footnotesize $^{a}$The average \Na--O distances calculated from 2 ns MD simulations are 
presented in columns 2--7 (in \angs). The Na1 coordination shell differs from that of 
Na1\dprim\ by the replacement of N310 (O$_{\delta}$) with N310 (O).
\end{table} 

Superposing the current configuration with a system containing two \Na\ ions bound to the Na1 and 
Na3 sites, we observe about 3~\angs\ distance between the Na1\prim\ and Na1 sites. This suggests 
that there may be another binding site in the vicinity of the Na1 site. Again we perform SMD 
simulations, steering the \Na\ ion from the Na1\prim\ site towards the Na1 site. After equilibration, 
a stable binding site is found at the location of the Na1 site, which will be called Na1\dprim. The 
coordination shell of Na1\dprim\ is very similar to that of Na1 (see \tabref{na3:tab1})---the 
only difference is that the carbonyl oxygen of N310 is replaced by the side-chain oxygen of N310. 
In this site, D312 no longer coordinates \Na, and the coordination shell is completed by the G306 
carbonyl oxygen and a water molecule. To see if the same coordination shell as in the Na1 site can 
be achieved, we generate a new configuration starting with two \Na\ ions present in Na1 and Na3 
sites. After removing the \Na\ ion from the Na3 site and equilibrating for 20~ns, we see the \Na\ 
ion moving from the Na1 site to the Na1\prim\ site instead. Thus it appears that the Na1\dprim\ 
configuration, where \Na\ is coordinated by N310 (O$_{\delta}$), is only accessible when \Na\ is 
steered from the Na3/Na1\prim\ sites to the Na1 position.

From the Na1\dprim\ site, there is only one obvious path for \Na\ to take. The Na1\dprim\ site is 
solvent-exposed, and we expect it to be the last site before \Na\ comes to a bulk-like environment 
outside the binding pocket. We steer the \Na\ ion from the Na1\dprim\ site towards the HP1-HP2 gate. 
Along the SMD trajectory, \Na\ has a brief contact with the A307 (O) along with G354 (O) and S279 (OH), 
which are residues part of HP1-HP2, but does not interact with any other residues in the binding pocket. 
Hence, we deduce that Na1\dprim\ is the last binding site before bulk. Summarising the steering 
simulations, we predict the last \Na\ ion to leave \GltPh\ through the following path: Na3 $\rightarrow$ 
Na1\prim\ $\rightarrow$ Na1\dprim $\rightarrow$ Bulk. The full transition path of the \Na\ ion from the 
Na3 site to bulk and the \Na\ binding sites are shown in \figref{na3:fig1}. 

% Figure of Transition Paths
\begin{figure}[b!]
 \centering
  \includegraphics[width=0.6\textwidth]{Figures/Na3-Paper/fig1.jpg}
 \caption{Full transition path of the \Na\ ion from  the  Na3 site to bulk 
(black line). The binding sites are indicated with yellow balls. The snapshot 
shows the positions of the key residues when the \Na\ ion is in the Na3 site.}
 \label{na3:fig1}
\end{figure}

It is clear from \tabref{na3:tab1} and \figref{na3:fig1} that the residues N310, D312, 
and to a lesser degree, D405 must undergo substantial conformational changes during the 
release of the \Na\ in order for their side-chain oxygen atoms to keep coordinating it. 
To help visualising these changes, we show snapshots of the three transitions, 
Na3 $\rightarrow$ Na1\prim, Na1\prim\ $\rightarrow$ Na1\dprim, and Na1\dprim\ $\rightarrow$ 
Bulk, in \figref{na3:fig2}. In each transition, the final conformations of the key residues 
(N310, D312, and D405) are superimposed on the initial ones using different colours, to 
indicate the extent of the conformational changes. The role of these changes in facilitating 
the release of the \Na\ ion from the Na3 site will be further discussed and quantified when 
we analyse the trajectory data obtained from the umbrella sampling simulations. It is 
important to note that the path described above may not be the lowest free energy path 
and that there may be other paths. However, based on the trial SMD simulations, this is 
the most likely path that is available.

% Figure of Transition Paths
\begin{figure}[b!]
 \centering
  \includegraphics[width=0.48\textwidth]{Figures/Na3-Paper/fig2.jpg}
 \caption{Transition path of the \Na\ ion: (A) Na3 $\rightarrow$ Na1\prim, (B) 
Na1\prim\ $\rightarrow$ Na1\dprim, and (C) Na1\dprim\ $\rightarrow$ Bulk. Only 
the important residues (N310, D312 and D405) are shown for clarity. Red, blue, 
and black balls represent the O, N, and C$\alpha$ atoms of these 
residues. Conformations of these residues in the initial and final states are 
shown in blue and orange, respectively. Yellow balls indicate the average 
position of the \Na\ ion in each umbrella window. Interactions of N310 (N$_{\delta}$) 
with the D312 and D405 side chains are indicated by dotted lines.}
 \label{na3:fig2}
\end{figure}

\subsection{Free Energy from Umbrella Sampling Simulations and Conformational Changes}
Using the RC described in the methods, we perform umbrella sampling simulations along the 
three transition paths, A) Na3 $\rightarrow$ Na1\prim, B) Na1\prim\ $\rightarrow$ Na1\dprim, 
and C) Na1\dprim\ $\rightarrow$ Bulk. For each transition path, a free-energy profile is 
constructed from the RC of the \Na\ ion using WHAM (\figref{na3:fig3}, top). Because a large 
spring constant is used, the overlap of the RC distributions between the neighbouring windows 
is observed to drop below the critical value ($<$5\%) in several places. Extra windows are 
inserted in such places to avoid any simulation artefacts in the free-energy profile due to poor 
sampling. This was a problem, especially near the transition states, where window separations as 
small as 0.1~\angs\ had to be used. The windows are obtained by performing SMD simulations from 
the closest umbrella window available. Evidence for the convergence of the free-energy profiles 
is presented in \figref{na3:figs1} in the Supporting Material. 

The trajectory data from the umbrella sampling simulations are also used to track the changes in 
the coordination shell of the \Na\ ion as it moves along each transition path (\figref{na3:fig3}, 
middle), and to calculate its diffusion coefficient using Eq.~\eqref{na3:eq3} (\figref{na3:fig3}, 
bottom). The side chains oxygens of N310 and D312 from TM7, and N401 and D405 from TM8 remain in 
the coordination shell of the \Na\ ion over substantial parts of the transition paths. It is of 
interest to find out whether the backbone atoms near these residues also move to help coordinating 
the \Na\ ion besides the side chains. For this purpose, we have calculated the RMSD of the C$\alpha$ 
atoms of the N310--T314 residues in TM7 (NMDGT motif), and the N401--D405 residues in TM8 using 
the configuration when \Na\ is in Na3 as reference (\figref{na3:fig4}). 

Below we discuss the critical events in each transition and correlate the energetic features of 
the free-energy profile with the conformational changes that occur in the protein using 
\figrefni{na3:fig2}{\ref{na3:fig4}}.

% free-energy profile and diffusion coefficient 
\begin{figure}[t!]
  \includegraphics[width=1.0\textwidth]{Figures/Na3-Paper/fig3.png}
\caption{Free-energy profiles (top), the \Na--O distances (middle), and the relative diffusion 
coefficients (bottom)  for the transitions 
(A) Na3 $\rightarrow$ Na1\prim, (B) Na1\prim\ $\rightarrow$ Na1\dprim,
and (C) Na1\dprim\ $\rightarrow$ Bulk. The bulk diffusion coefficient of \Na\ 
ions is taken as $D_0=1.33 \times 10^{-9}$~m$^2$/s~\cite{Harned1958}.}
\label{na3:fig3}
\end{figure}

A) Na3 $\rightarrow$ Na1\prim\ (\figrefi{na3:fig3}{A}): The two sites are separated by about 
5.5~\angs, and the transition state (TS1) is near $z=3.1$~\angs. After the \Na\ ion leaves the Na3 
site, first S93 (OH) departs from its coordination shell followed by Y89 (O) and then T92 (OH). This 
is partly compensated by the full integration of D312 (O1) in the shell. Both the N310 and D312 side 
chains faithfully track the \Na\ ion in this region. The ion remains under-coordinated while dragging 
the N310 and D312 side chains, which results in a steeply rising free-energy profile up to TS1. The 
height of the energy barrier at TS1 is 17.3~kcal/mol, which is an extremely high barrier for an ion 
to cross. Entry of N401 (O) to the coordination shell at TS1 provides some relief to the free-energy 
profile. Finally, with the entry of D405 (O1) to the coordination shell, the Na1\prim\ binding site is 
formed. Inspection of \tabref{na3:tab1} shows that the quality of the coordination shell of \Na\ at 
Na1\prim\ is at least as good as that at Na3, yet the \Na\ free energy at Na1\prim\ is 6.9~kcal/mol 
higher than that at Na3. We attribute this to the conformational changes that occur both in the N310 
and D312 side chains (\figrefi{na3:fig2}{A}), and in the backbone of the NMDGT motif (\figrefi{na3:fig4}{A}) 
during the transition. The RMSDs of the C$\alpha$ atoms in NMDGT change little between Na3 and TS1 
but exhibit larger changes between TS1 and Na1\prim. 

The diffusion coefficient profile of \Na\ along this path remains below the bulk value (i.e. 
$1.33 \times 10^{-9}$~m$^2$/s~\cite{Harned1958}) even at the transition state. This is because the 
ion is well coordinated by oxygen atoms through the transition (\figrefi{na3:fig3}{A}). If the ion is 
not well coordinated, then the environment will result in a smaller friction coefficient for the ion 
and the diffusion profile increases. We next combine the free-energy profile and diffusion coefficient 
results in Eq.~\eqref{na3:eq2} to estimate the escape time for the Na3 $\rightarrow$ Na1\prim\ transition, 
which yields about 7~sec. This is a very slow process for an ion but is a small fraction (4\%) of 
the observed turnover time of $\sim$3~min in \GltPh{}~\cite{Ryan2009}. 

% RMSD
\begin{figure}[t!]
\centering
 \includegraphics[width=1.0\textwidth]{Figures/Na3-Paper/fig4.jpg}
 \caption{RMSD of the C$\alpha$ atoms of residues in TM7 (A) and TM8 (B). The 
          reference frame is chosen as the state with \Na\ in the Na3 site. RMSDs 
          are calculated at all \Na\ binding sites, and the transition states in 
          between.}
\label{na3:fig4}
\end{figure}

\begin{figure}[b!]
\centering
 \includegraphics[width=0.6\textwidth]{Figures/Na3-Paper/fig5.png}
 \caption{Effect of the restraining of the D312 residue on the Na3 $\rightarrow$ 
          Na1\prim\ free-energy profile. The restraint-free profile from \figrefi{na3:fig3}{A} 
          (black) is compared to the profile obtained while D312 is restrained (red).}
\label{na3:fig5}
\end{figure}

An interesting question here is the chaperone role played by the D312 side chain, and how much this 
helps the \Na\ ion to cross the Na3 $\rightarrow$ Na1\prim\ barrier. To address this question, we 
have constructed another free-energy profile for this transition while keeping the D312 residue 
restrained in its initial position at the Na3 site. As shown in \figref{na3:fig5}, the energy barrier 
faced by the \Na\ ion is nearly doubled when the D312 residue is restrained. To see the effect of the 
higher barrier on the escape time, we repeat the above calculation using the restrained free-energy 
profile in \figref{na3:fig5}. We find that the escape time has increased from 7~sec to $10^{13}$~min 
or 18 million years. Thus chaperoning by the D312 side chain plays an essential role in facilitating 
the escape of the \Na\ ion from the Na3 site. While the N310 side chain also tracks the \Na\ ion, 
this is not solely due to the N310 (O$_{\delta}$)--\Na\ interaction. As shown in 
\figrefn{na3:fig2}{A{\color{black}--}B}, N310 (N$_{\delta}$) forms an ionic bond with D405 (O1), 
which provides an extra incentive for the movement of the N310 side chain. Presumably, this interaction 
is also responsible for the exclusion of D405 (O1) from the coordination shell of \Na\ at the Na1\prim\ 
site (\tabref{na3:tab1}).

\begin{figure}[b!]
\centering
 \includegraphics[width=0.5\textwidth]{Figures/Na3-Paper/fig6.jpg}
 \caption{Ratchet-like behaviour of N310 as \Na\ moves from the Na1\prim\ site 
          (A) to the Na1\dprim\ site (B). N310 (N$_{\delta}$) switches its H-bond 
          from D405 in (A) to D312 in (B), which sets up the H-bond network that 
          prevents \Na\ from falling back to Na1\prim. H-bonds are indicated with 
          dashed lines.}
\label{na3:fig6}
\end{figure}

B) Na1\prim\ $\rightarrow$ Na1\dprim\ (\figrefi{na3:fig3}{B}): The sites are separated by 2~\angs, 
and TS2 is at $z=1.2$~\angs. The \Na\ ion remains well-coordinated throughout this region. In 
particular, at TS2 both D312 and D405 side chains coordinate the ion. After TS2, D312 completely 
decouples from the ion with its backbone and side chain relaxing back while both side-chain oxygens 
of D405 firmly couple to the ion (\figrefi{na3:fig2}{B}). But functionally the most significant 
event in this transition is the conformational change exhibited by N310---position of its backbone 
undergoes a large shift and its side chain flips (\figrefi{na3:fig2}{B}). It is 
seen from \figref{na3:fig4} that the stress created on the NMDGT motif by the binding of the \Na\ 
ion to the Na1\prim\ site becomes maximal at TS2 and is relieved only after the ion binds to Na1\dprim. 
Flipping of the N310 backbone is not driven by ion coordination, which actually gets a bit worse at 
TS2. Rather, it enables N310 (N$_{\delta}$) to switch its H-bond partner from D405 (O1), which is 
needed for the coordination of \Na, to D312 (O1) which has become free after TS2 (\figrefi{na3:fig2}{B}). 
In this position, N310 (N$_{\delta}$) also makes an H-bond with N401 (O), thus setting up the H-bond 
network, D312 (O1)--N310 (N$_{\delta}$)--N401 (O)--D405 (N) (\figrefn{na3:fig6}{A{\color{black}--}B}). 
The snapshot in \figrefi{na3:fig6}{B} suggests that this H-bond network may prevent the backward motion 
of the \Na\ ion from Na1\dprim\ to Na1\prim. 

Inspection of the free-energy profile in \figref{na3:fig3} shows that this is a real possibility for 
the ion at Na1\dprim. For \Na\ at Na1\prim, the forward and backward barriers are 4.9 and 10.4~kcal/mol, 
respectively, so the forward motion is much more likely. But for \Na\ at Na1\dprim, the forward and 
backward barriers are 14.3 and 2.4~kcal/mol respectively, which clearly prefers the backward motion. 
However, due to the formation of the H-bond network after the ion goes over TS2, the Na1\prim\ 
$\rightarrow$ Na1\dprim\ transition is not reversible, and the free-energy profile in \figrefi{na3:fig3}{B} 
is valid only in the forward direction. To show what happens in the reverse direction and demonstrate the 
effect of the irreversibility quantitatively, we have constructed another free-energy profile for the 
reverse transition Na1\dprim\ $\rightarrow$ Na1\prim. As shown in \figref{na3:fig7}, there is a rising 
free-energy profile in the reverse direction well beyond TS2. Thus the \Na\ ion cannot breach the H-bond 
network, and it has to move forward to bulk. A clearer illustration of the ratchet-like behaviour of N310 
during the Na1\prim\ $\rightarrow$ Na1\dprim\ transition is provided in \figref{na3:fig6}. The ratchet 
function of N310 is as important in facilitating the release of the \Na\ ion as its chaperoning by D312 
during the Na3 $\rightarrow$ Na1\prim\ transition. However, it is important to note that the ratchet 
process may be a result of micro-irreversibility due to limited sampling. Long-time MD simulations 
in the scale of microseconds may reveal a different picture.

\begin{figure}[t!]
\centering
 \includegraphics[width=0.6\textwidth]{Figures/Na3-Paper/fig7.png}
 \caption{Comparison of the free-energy profiles for the forward (black) and reverse (red) 
          transitions between the Na1\prim\ and Na1\dprim\ sites. The reverse 
          energy barrier exceeds 16 kcal/mol, which is well over the barrier 
          for the Na1\dprim\ $\rightarrow$ Bulk transition.}
\label{na3:fig7}
\end{figure}

A similar H-bond network but without the involvement of D312 was also observed in MD simulations 
of the OF conformation of \GltPh~\cite{Huang2010}. It was noted that this H-bond network prevented 
hydration of the Na3 site, but the presence of a \Na\ ion  at the Na1 site was sufficient to break 
the H-bond network and access the Na3 site~\cite{Huang2010}. As mentioned earlier (\tabref{na3:tab1}), 
\Na\ is coordinated by N310 (O) at the Na1 site, and not by the N310 side chain, which makes only 
one H-bond with N401 (O). Thus it is relatively easy for \Na\ to break this single H-bond as observed 
in MD simulations. In contrast, at the Na1\dprim\ site of the IF conformation, the N310 side chain is 
involved in two H-bonds besides coordinating \Na\ (\figrefi{na3:fig6}{B}). As demonstrated by the 
reverse free-energy profile (\figref{na3:fig7}), it is much harder to break this H-bond network, which 
is fortified by the addition of the D312 side chain.

In other respects, the Na1\prim\ $\rightarrow$ Na1\dprim\ transition is rather ordinary. The 
diffusion coefficient of \Na\ remains near the bulk values, and the energy barrier at TS2 
(4.9~kcal/mol) is rather low. The transition time estimated using the free-energy profile and 
diffusion coefficient results in \figrefi{na3:fig3}{B} yields about 10~ns. This is nine orders of 
magnitude smaller than the escape time from the Na3 site and hence completely negligible 
compared to it.

C) Na1\dprim\ $\rightarrow$ Bulk (\figrefi{na3:fig3}{C}): Here, the transition state on the free-energy 
profile extends to about 4.5~\angs. As the \Na\ ion moves towards bulk, N401 (O) is the first to depart 
its coordination shell. The N310 and D405 side chain oxygens track \Na\ for about 4~\angs\ and 
start decoupling after that. During the initial pulling stage from N310 and D405, a second water 
molecule enters the coordination shell of \Na. We note that the TM8 backbone remains relatively 
rigid during the release of \Na\ (\figref{na3:fig4}), which partly explains the limited chaperoning 
role played by D405 compared to D312. The under-coordination of \Na\ while pulling the N310 and D405 
side chain is responsible for the steep rise in the profile in the region $z=0$--5~\angs. After decoupling 
of N310 and D405 (i.e. from $z=5$~\angs) the ion is coordinated by about 5 water molecules as shown 
in \figref{na3:fig8}. At this point, \Na\ is in the third transition state TS3 and is coordinated by 
one shell of water molecules. It is important to note that the \Na\ ion is not out the bulk medium yet 
as it is still inside the binding pocket even though it is coordinated with 5 water molecules. The 
number of water molecules coordinating \Na\ decreases to three as it passes through the HP1-HP2 gate. 
From 6--10~\angs\ the \Na\ weakly interacts with A307 (O), G354 (O) and S279 (OH) residues, which is 
responsible for the dip in the free-energy profile from the transition state and reduced number of water 
molecules. From 10~\angs\ onwards \Na\ detaches itself from any protein oxygens with S279 (OH) being 
the last and is coordinated by about 6 water molecules, \figref{na3:fig8}. As shown in the free-energy 
profile beyond 10~\angs\ the energy is flat signalling that the ion has reached a bulk-like medium. 

\begin{figure}[t!]
\centering
 \includegraphics[width=0.6\textwidth]{Figures/Na3-Paper/fig8.jpg}
 \caption{The average number of water molecules entering the coordination shell of \Na\ 
          during the Na1\dprim\ $\rightarrow$ Bulk transition. The results are obtained 
          from the umbrella sampling windows using a cutoff radius of 3~\angs\ for 
          the \Na--O distance.}
\label{na3:fig8}
\end{figure}

The peak value of the free-energy profile is 14.3~kcal/mol, which is a substantial energy barrier for 
the ion to surmount and exit into the bulk. At the bulk (the flat region in the plot) the energy is 11.1~kcal/mol. 
The diffusion coefficient of the ion fluctuates around half the bulk value and approaches the bulk value 
near the flat region of the profile reflecting the exit and entry of various protein and water oxygens 
to the coordination shell of the ion during the extended transition region. The escape time obtained 
from the free-energy profile and diffusion coefficients in \figrefi{na3:fig3}{C} is about 0.5~s. While 
this is slower than the Na1\prim\ $\rightarrow$ Na1\dprim\ transition, it is only one order of magnitude 
faster compared to the escape time from the Na3 site. Thus the slowest step for the release of \Na\ from 
Na3 is the Na3 $\rightarrow$ Na1\prim\ transition.

\subsection{Implications for \GltPh}
The detailed discussion of the release of the \Na\ ion from the Na3 site to bulk shows that the 
protein does not just provide a passive conduit but is actively involved in the release of the ion 
through the chaperon and ratchet functions of the D312 and N310 residues in the NMDGT motif. This is 
consistent with the calorimetric studies~\cite{Reyes2013b,Hanelt2015,Ewers2013} and comparison of the 
apo and bound structures of \GltPh~\cite{Verdon2014}, which indicate that substantial conformational 
changes occur in the protein during \Na\ binding. An important consequence of these observations is 
that binding free energy calculated using path-independent methods with unvarying \GltPh\ structures 
cannot provide an accurate description of the ligand binding/unbinding process. For example, the energy 
for the translocation of \Na\ from the Na1\dprim\ site differs between the two methods. The free-energy 
profile for this transition yields an energy difference of $-$11.1 kca/mol (\figrefi{na3:fig3}{C}) while 
TI calculations give $-$17.5 kcal/mol (\figref{na3:figs2}). This difference is due to the conformational 
changes as a result of doing work on the protein as the ion unbinds. In this case, the chaperoning of 
the \Na\ ion by the N310 and D405 side chain up to TS3 helps reduce the energy barrier in the profile 
and hence the translocation energy. Therefore, to obtain the correct value with path-independent 
methods, the conformational energy needs to be taken into account. As mentioned previously, the 
estimated escape time of \Na\ is around 7~sec, yet the measured turnover rate is 3~min~\cite{Ryan2009}. 
Thus even though the release process is it is not be the rate-limiting step in the transport cycle. 
The rate-limiting step in the transport cycle is most likely the conformational transition of the 
protein from the OF to IF state. Work has been done on this transition using the anisotropic network 
model (ANM) with coarse-grained models of \GltPh~\cite{Das2014}. However, an estimate of the 
transition time has not been calculated, and further studies are needed.

\subsection{Implications for EAATs}
Transport rates in EAATs are much faster than \GltPh, e.g., the turnover time for EAAT3 is about 
0.01~s~\cite{Grewer2000}, which is 2000 times faster than \GltPh. Because the residues involved 
in ligand binding are conserved between \GltPh\ and EAATs, one has to consider other factors 
for the speed up. Computational studies on a homology model of EAAT3 have shown that a \K\ ion 
can bind to a site near the Na1\prim--Na1\dprim\ sites~\cite{Heinzelmann2014,Heinzelmann2014a}. 
As discussed in the Na3 $\rightarrow$ Na1\prim\ transition, the tunnel connecting the two sites 
is too narrow to allow any exchange of the \Na\ and \K\ ions. This also appears to be the reason 
why \Na\ at the Na3 site could not be exchanged with \Tl\ in \GltPh\ crystal structures 
\cite{Boudker2007,Reyes2009}. Inspection of the EAAT1 crystal structure \cite{Canul-Tec2017} shows 
that this tunnel is larger than that in \GltPh. Thus the barrier of Na3 $\rightarrow$ Na1\prim\ 
is expected to decrease in EAATs. In addition, the exchange of \K\ and \Na\ will further reduce 
the transport cycle time, which could explain the extremely fast process compared to \GltPh. 
The exchange process of \Na\ and \K\ requires further experimental and computational studies to 
clarify this hypothesis. 

\section{Conclusion}
Compelled by the recent experimental findings that conformational changes accompany binding of \Na\ 
ions in \GltPh, we have performed umbrella sampling calculations for the release of the last \Na\ 
ion from the Na3 site to bulk in the IF state of \GltPh. Identifying the rate-limiting steps in \GltPh\ 
is expected to provide important clues on how the coupling of \Hi\ and \Na\ ions in EAATs could speed up the 
transport rate by up to three orders of magnitude. Our estimate for the release time of the last \Na\ 
ion, obtained from the free-energy profile and diffusion coefficient results, is about 7~sec. This is 
a small fraction of the observed turnover time of 3 min (about 4\%), which suggests that the release 
of the last \Na\ ion is a very slow process but is not the rate-limiting step in \GltPh. The transition 
from the OF to IF is expected to be a slower process than the release of \Na.

The umbrella sampling simulations, performed along the three transition paths identified between 
the Na3 site and bulk, have revealed that the D312 and N310 residues in the NMDGT motif undergo 
substantial conformational changes, and thereby play crucial roles in facilitating the release of 
the \Na\ ion. As shown in  \figref{na3:fig5}, it would be impossible for the \Na\ ion to escape from 
the Na3 site without the chaperoning of the D312 side chain. Similarly, the N310 side chain chaperones 
the \Na\ ion from the Na3 site all the way to the protein/bulk interface. Another contribution of N310 
to the release of the \Na\ ion is its ratchet function during the Na1\prim\ $\rightarrow$ Na1\dprim\ 
transition, which prevents the ion from going back in the short timescale available in the current 
MD simulations (\figrefs{na3:fig6} and \ref{na3:fig7}). Much longer simulations in the microseconds 
timescale may demonstrate the reversibility of the N310 side chain. Also, the path investigated may or 
may not be the lowest free energy path. Further investigation is required, and more advanced 
computational method like the string-method with swarms of trajectories~\cite{Gan2009} may be needed.

\pagebreak
\begin{subappendices}
\counterwithin{figure}{section}
{
\hypersetup{linkcolor=black}
\section{Appendix for Chapter~\ref*{chap:unbind}}
}


% Free-energy profile Convergence
\begin{figure}[b!]
\centering
 \includegraphics[width=8.5cm]{Figures/Na3-Paper/fig1A.png}
 \caption{The convergence of the free-energy profile shown in \figref{na3:fig3} from block 
          data analysis. In each case, 3 ns of production is collected after equilibration, 
          which took up to 2 ns (not shown). The free-energy profile, obtained from 1 ns 
          blocks, is seen to fluctuate around the baseline profile obtained from the total 
          production data.}
\label{na3:figs1}
\end{figure}

% TI profile for Na1'' to Bulk
\begin{figure}[h!]
 \centering
 \includegraphics[width=0.6\textwidth]{Figures/Na3-Paper/fig2A.jpg}
  \caption{The convergence of TI calculations for translocating the \Na\ ion from 
           bulk to the Na1\dprim\ site. Calculations are performed using the 
           protocols in Refs.~\cite{Heinzelmann2011,Heinzelmann2013}. The difference 
           between the forward and backward TI results is less than a kcal/mol, 
           indicating negligible hysteresis. The average of the forward and backward 
           results yields $-17.5$~kcal/mol for the translocation free energy of 
           \Na\ from bulk to the Na1\dprim\ site.}
\label{na3:figs2}
\end{figure}

\end{subappendices}

\counterwithin{figure}{chapter}
%=======================================================================================%