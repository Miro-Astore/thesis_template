\definecolor{codegreen}{rgb}{0,0.6,0}
\definecolor{codegray}{rgb}{0.5,0.5,0.5}
\definecolor{codepurple}{rgb}{0.58,0,0.82}
\definecolor{codeorange}{rgb}{0.98,0.6,0.01}
\definecolor{codeorange2}{rgb}{1.0,0.5,0.0}
\definecolor{backcolour}{rgb}{0.95,0.95,0.92}

\lstdefinestyle{customTCL}{
    backgroundcolor=\color{backcolour},   
    commentstyle=\color{codegreen},
    keywordstyle=\color{blue},
    numberstyle=\tiny\color{codegray},
    stringstyle=\color{codeorange},
    basicstyle=\footnotesize,
    breakatwhitespace=false,
    language=tcl, 
    breaklines=true,                 
    captionpos=b,                    
    keepspaces=true,                 
    numbers=left,                    
    numbersep=5pt,                  
    showspaces=false,                
    showstringspaces=false,
    showtabs=false,                  
    tabsize=4
}

\lstdefinestyle{customF90}{
    backgroundcolor=\color{backcolour},   
    commentstyle=\color{codegreen},
    keywordstyle=\color{blue},
    numberstyle=\tiny\color{codegray},
    stringstyle=\color{codeorange},
    basicstyle=\footnotesize,
    breakatwhitespace=false,
    language=fortran, 
    morekeywords={*,getarg},
    breaklines=true,                 
    captionpos=b,                    
    keepspaces=true,                 
    numbers=left,                    
    numbersep=5pt,                  
    showspaces=false,                
    showstringspaces=false,
    showtabs=false,                  
    tabsize=4
}

\lstdefinestyle{customPython}{
    backgroundcolor=\color{backcolour},   
    commentstyle=\color{blue},
    keywordstyle=\color{codeorange2},
    numberstyle=\tiny\color{codegray},
    stringstyle=\color{codegreen},
    basicstyle=\footnotesize,
    breakatwhitespace=false,
    language=Python, 
    breaklines=true,                 
    captionpos=b,                    
    keepspaces=true,                 
    numbers=left,                    
    numbersep=5pt,                  
    showspaces=false,                
    showstringspaces=false,
    showtabs=false,                  
    tabsize=4
}

\newpage
%\addcontentsline{toc}{chapter}{\hyperref[apx:charges]{Appendix}}
\appendix

An appendix of uncollected thoughts. 

\begin{itemize}
\item Considering that the passive immune system consists of more inanimate layers of the body such as the skin would it not also make sees that social elements of our behaviour form part of our immune system as well? Think about the visceral reaction we have toward fecal matter or someone vomiting. These are neural queues to change our behaviour. Our response to the pandemic---social distancing and vaccine development could then be viewed as part of our adaptive immunity. 

\item V(D)J recombination is a process where the body somatically shuffles parts other genome in order to create variable antibodies which target pathogenic substrates with high affinity and specificity. It's amazing and potentially has important implications for protein physics and protein bioinformatics.

\item There are considerable efforts to develop drug discovery mechanisms for specific cell types \cite{yu2020} as this has the potential to reduce the off target side effects and increase therapeutic efficacy. 

\item The megaplate experiment\cite{baym2016} is an illustrative example of a Strange Loop \cite{hofstadter2007}. 

\item Molecular dynamics is well situated to the molecular basis for disease. In an organism (even if genetic engineering were to progress substantially) the genome is effectively fixed. Meaning states we consider to be diseased states are only a small perturbation away from the statistical norm. So we can use our physical understanding of the base state as a way to understand the disease (note that the disease state also tells us a lot about the healthy state too). By contrast fields like synthetic biology take a more high through-put approach, and current molecular dynamics techniques and computational engines can simply not keep up with the number of experiments carried out in these fields.  

\item I just noticed that the words organ, organism and organisation all contain the Greek root ``organon" meaning ``tool". That's a weird coincidence. Someone should look into that. 
\end{itemize}
