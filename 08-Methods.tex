%=======================================================================================%
\chapter{From Protons to Proteins: Methods to simulate the inside of a cell.}
\numberwithin{equation}{chapter}
\label{chap:methods}

\section{Quantum Mechanics is Not Tractable at the Scale of Biology.}
Living things are made of atoms and atoms themselves are composed of many particles. The motions of atoms and their constituent particles are governed by quantum mechanics. Unfortunately, performing simulations for the number of atoms involved in proteins and other cellular components at quantum mechanical accuracy is impossible. Hence, we will show how to take the fundamental formulation of atomic interactions in the Schr\"{o}dinger wave equation and apply approximations in order to produce a model which is capable of simulating macromolecular systems at biologically relevant timescales. 

We will gradually integrate upwards, beginning with the interactions in a single atom we will work our way up to a complex macromolecular system with lipids, water, salts and of course, proteins. Ultimately this section rationalises the treatment of atoms as point charges in classical molecular dynamics simulations.

\subsection{A full quantum mechanical treatment}
Since we are dealing with atoms which are governed by quantum mechanics we must begin our journey upwards with the time dependent form of the Schr\"{o}dinger wave equation. 

\begin{equation}
i\hbar \frac {\partial}{\partial t} \Psi (\textbf{x},t) = \big[ -\frac{\hbar ^2}{2m}\nabla^2 + V (\textbf{x}, t) \big] \Psi (\textbf{x},t) 
\label {schordinger_time_dependent}
\end{equation}

In quantum systems we treat all particles as waves hence the use of the wave function $\Psi (\textbf{x},t)$. The complex amplitude of the wave function $|\Psi (\textbf {x}, t)|^2$ tells us the likelihood of detecting the particle at time $t$ and at place $\textbf{x}$. The term in the brackets correspond to $-\frac{\hbar ^2}{2m}\nabla^2 $ the kinetic energy of the particle with mass $m$ while $V (\textbf{x}, t)$ is the potential energy of the system. Given that the left hand term $i\hbar \frac {\partial}{\partial t} \Psi (\textbf{x},t)$ contains a gradient with respect to time, it governs how the wave function will evolve in time.

When the external potential $V$ has no explicit dependence on time, this equation reduces to the familiar time independent form. 

\begin{equation}
	E \Psi (\textbf{x}, t) = \big[ -\frac{\hbar ^2}{2m}\nabla^2 + V (\textbf{x}) \big] \Psi (\textbf{x}, t) = H \Psi(\textbf{x}, t) 
 \end{equation}

Note that the wave function $\Psi (\textbf {x}, t)$ is still allowed to evolve in time. 

In an atom there are two types of particles, nuclei which we will denote with the subscript $i$ and electrons denoted by $e$. In order to treat these elements separately we decompose the Hamiltonian of the system into a few components. 

\begin {equation}
H = T_n + T_e + V_{n-n} + V_{n_e} + V_{e-e}
\end{equation}

Where $T_n$ and $T_e$ denote the kinetic energy of the nucleus and electrons respectively. While $V_{n-n}, V_{n-e}, V_{e-e}$ denote the potential energy for interactions between nuclei, between electrons and nuclei and between electrons respectively.

Since the potential terms all describe charged species, they follow Coulomb's law and have the form.

\begin{equation}
	V_{n-n} = \sum_{i>j} \frac{q_e^2 z_i z_j }{|\textbf{R}_i-\textbf{R}_j|},\quad V_{n-e} = \sum_{i,l} \frac{q_e^2 z_i }{|\textbf{r}_l-\textbf{R}_i|},\quad  V_{e-e}  = \sum_{l>k} \frac{q_e^2 }{|\textbf{r}_l-\textbf{r}_k|}
\end{equation}

Here the $z_i$ represent the charge atomic number (and thus the charge) of the $i$th nucleus and $q_e$ is the unit charge of the electron. The reason for the separate coordinates $R_i$ and $r_l$ is to separate out the treatment of nuclei and electrons which will be important once we apply the Born-Oppenheimer approximation.

Meanwhile, the kinetic energy terms are quite simple. 

\begin {equation}
T_n = - \sum_i \frac{\hbar^2}{2M_i} \nabla_i ^2,\quad  T_e = - \sum_l \frac{\hbar^2}{2m_l} \nabla_l ^2
\end {equation}

$M_i$ represents the mass of the $i$th nucleon and $m_l$ represents the mass of the $l$th electron. The separate subscripts $i$ and $l$ are due to the different coordinates which we use to denote the positions of the nuclei and the electrons. The reason for this will become clear when we apply the Born-Oppenheimer approximation to separate the wave functions and solve them separately.

Here, the $M_i$ are the masses of the nuclei and the operator $\nabla = \frac{\partial}{\partial x} + \frac{\partial }{\partial y} + \frac{\partial}{\partial z} $


\subsection{The Born-Oppenheimer approximation.}
We now make use of the Born-Oppenheimer approximation \cite{Born1927}. This approximation is motivated by the observation that electrons are 3-4 orders of magnitude lighter than the nucleus, so the kinetic energy in the electrons is negligible and we can assume that the electrons will respond instantaneously to any changes in the wave function of the nucleus. Thus, we can disregard $T_e$, $U_{n-e}$ and $U_{e-e}$

This allows us to split the total wave function into two parts using a direct product. One term deals with the nuclei and one with the electrons in the system. 

\begin {equation}
\Psi(R_i,t) = \psi_e (r_l,R_i) \psi_n(R_i,t)
\end {equation}

\subsection{Simulating Molecules Without any Quantum Mechamics}
The Born-Oppenheimer approximation can be followed further to derive Hartree-Fock methods which allow calculations of the organisation of electron clouds around small molecules. This lets us derive the energy profile of certain degrees of freedom within the molecule such as the energetics of stretching out a bond or twisting a dihedral angle. In molecular dynamics we try to match these energetic functions using an effective potential. 
The CHARMM effective potential employed in this work is common in all-atom molecular dynamics. Similar functions are used in other forcefields such as AMBER, GROMOS and OPLS with different parameters and design philosophies.

We split up the molecular potential into several components dealing with the energies from covalent bonds, including bond stretching, twisting and pinching. As well as energies associated with the forces that atoms exert on eachother hwne they are not bonded together. Namely and Coulomb forces due to electric charges on the atom and attractive Van Der Walls interactions and repulsion due to Pauli Exclusion the latter two forces are combined into one term we will analyse in detail $U_{LJ}$.
\begin{equation}
	U_{CHARMM} = \underbrace{U_{LJ} + U_{Coulomb}}_{U_{non-bonded}} + \underbrace{U_{bonds} + U_{angles} + U_{dihedrals} + U_{impropers}}_{U_{bonded}}
\end{equation}

Interestingly, the bonded terms may all reasonably be approximated by harmonic springs. 

\begin{equation}
	U_{bonds} = \sum_{i>j} k_{bonds} |(\textbf{r}_i - \textbf{r}_i) - r_0 |^2 , U_{angles} = \sum_

\end{equation}

