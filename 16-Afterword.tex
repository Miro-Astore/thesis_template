\chapter{Afterword}
\label{chap:Afterword}
\chapquote {}{}

 Thanks for coming along. I hope that towel came in handy \cite{adamd1979}. If you've made it this far I probably have you as a captive audience so I'm going to keep you just a bit longer. There are some things I'm going to tell you which are at the same time political, moral, practical and philosophical. The most important thing I will stress is the collaborative nature of science.

Just look at the reference list that follows, there are hundreds of publications listing thousands of authors. This likely millions, perhaps even billions of dollars of investment over the course of more than 100 years. However meager, my contributions to the study of CF and biophysics were only possible because of the patient and careful work of these people, from mathematicians to epidemiologists. Even the breakthroughs of the mythical lone genus aren't useful to anybody until implemented into a technology by thousands of workers. 

Throughout my studies I've needed to consult experts from all over science, who's brains have also been shaped by their own community of thousands. Every time I spoke with a friend in computer science I'd come away with a new trick to churn out simulations faster, every time I spoke to an astrophysicist I'd begin thinking in abstract hyperspaces which inspired me to push CFTR in a new direction (literally), when I spoke to cell biologists I'd come away motivated by the crazy things I hope to one day do help do to cells and every time I spoke to a biochemist I would something about proteins or CFTR which I could use for my simulations. I hope at the same time that some of my colleagues have also taken some inspiration from an increasingly unhinged physicist with a handful of GPUs. 

In these ways it has been invaluable to me to be able to stroll down the strangely long hallway of the school of physics to ask Zachary Picker for help with quantum mechanics or send an email across the world to ask Isabelle Callebaut what she thought about my dilated CFTR structure. Without such a diverse set of people I doubt I would have produced what I think is probably one of the few PhD dissertations which makes mention of both Schrodinger's wave equation and rectal biopsies. 

I want to make a quick note on the use of computers. I think, like pen and paper for the traditional mathematician, computers should be used as tools to think for the biologist. We have excellent theoretical models in biology but they are too complicated for the human brain to crank the handle on. I have found that automating parts of a workflow and cultivating a maintain library is like maintaining a tool belt. It makes you more useful. Plug into the matrix. It's fun.

They might have taken a long time but I hope I have shown you just how useful this kind of basic science can be . We're an adaptable species but we don't yet have any method of solving technical problems except this kind of patient investment

If my visa isn't denied I'm really looking forward to doing more of this stuff on TRP channels, maybe I'll make super chillies or something. .

In any case, thanks for sticking through X thousand words. I hope you learned a thing or two. .

Best, Miro
