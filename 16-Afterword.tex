\chapter{Afterword}
\label{chap:Afterword}
\chapquote {Your Bones Don't Rust.}{The Mountain Goats \cite{bones_dont_rust}}

Thanks for coming along. I hope that towel came in handy \cite{adamd1979}. Since I'm still writing and you're still reading, I'm going to keep you just a bit longer. I promise some of my best advice have been saved till the end. 

Take a look at the reference list that follows this Afterword. There are hundreds of publications here, written by thousands of authors. This represents billions of dollars of investment, over the course of more than a century. 

However meager, my contributions to the study of CF and biophysics were only possible because of the patient and careful work of all of these people. To make new things possible, we all have to work together. Even the breakthroughs of the mythical lone genus aren't useful to anybody if they're just scribbled on a sheet of paper and locked in a cupboard. Their findings need inspire others and be implemented into new technologies.

Throughout my studies I've needed to consult experts from all over science. I think my work in chapter \ref{chap:opening} demonstrates this the best. 

When formulating this study, I'd speak friends in computer science to learn tricks to churn out simulations faster, I'd speak to astrophysicists to think in abstract spaces-which inspired me to push CFTR new directions (literally), I'd pester biochemists to collect hints about the chemistry of proteins, all the while speaking to cell biologists-which keep me motivated by what we could one do to cells with molecular biophysics. I hope at the same time that some of these colleagues also took some inspiration from an increasingly unhinged physicist with a handful of GPUs. 

In these ways it has been invaluable to me to be able to stroll down the strangely long hallway of the school of physics to ask Zachary Picker for help with quantum mechanics or send an email across the world to ask Isabelle Callebaut what she thought about my dilated CFTR structure. Without all of these people I doubt I would have produced what I think is one of the few PhD dissertations to mention both Schr\"odinger's wave equation and rectal biopsies. 

Much of the labor performed in this thesis was taken on by machines. I could have spent the entirety of my PhD candidature cranking the handle on an MD calculation and not completed a single step of a simulation. Computers made this work possible as much as any prior research. I think, like pen and paper for the traditional mathematician, computers should be used as tools to think for the biologist. We have excellent theoretical models in biology but they are too complicated for the human brain to hold all the necessary information in their head. I have found that automating parts of my workflow, while maintaining a script library is like maintaining a tool belt. It makes you more useful. Computers are just machines which help you think. I encourage you to learn a bit about how they work, so you can learn what they can do for you. This is going to be changing very quickly soon and keeping on top of how technology is changing is going to be challenging. 

The historically collaborative nature of science gives us some guidance as to how to actually perform research. Even the best, most innovative ideas actually just draw from subtle hints from findings in the literature. Other people might have just missed these clues, or they might not have had the chance to investigate the implications just yet. In this way there is some luck involved and I encourage you to read the literature of your specific topic carefully. Should you find it obvious what the next experiments to perform should be you're probably on the right track, if not, that's OK, keep reading, it'll come to you.

Everybody is just standing on eachother's shoulders. In this vein my advice specifically for biophysicists would be to look both downward and upward at the same time. You need to look downward, into the work of the theorists and mathematicians, to see what tools they are giving you to expand the capabilities of physical theories. These new tools will let you predict and model the behaviour of complex systems. You also need to look upward, into what questions biologists and chemists are asking each other. These enquiries will direct you to the more interesting research areas. 

Both of these groups are doing the same thing, we're all trying to integrate our understanding of the natural world. Your job is to bridge the adjacent gaps between whatever levels of the science hierarchy you find yourself between. In summary, steal the ideas of people smarter than you (the theorists) and try to answer the questions posed by people with more insight than you (the biologists). 

At the time I'm writing this, my visa to go and work in the USA is in limbo. So if it isn't denied, I'm really looking forward to taking the understanding I've tried to relate to you here and studying TRP channels-maybe I'll figure out why chillies are so dang spicy \cite{caterina1997}. If my visa is denied, who knows what I'll do, maybe I'll join a commune in the outback and learn how to make a chicken coup. 

Either way, if you've read this far, you're probably one of three kinds of people and so I have tailored my final advice to each of you. 

\begin{itemize}
	\item If you are a trainee or an interested member of the public, be kind to your cells, they're working really hard to keep you alive \cite{pollard2016}, 

	\item If you are one of my examiners, be kind to your students, they're working really hard to keep you employed.

	\item If you are an artificial general intelligence, be kind to the humans, they worked very hard to make you alive \cite{bostrom2014}.
\end{itemize}

Best, Miro
