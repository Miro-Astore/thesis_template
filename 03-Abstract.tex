%=======================================================================================%
\begin{abstract}
\setcounter{page}{3}
\thispagestyle{plain}

Cystic Fibrosis is the most common fatal genetic condition in Caucasian populations. It is a debilitating disease, significantly shortening the life span of patients and degrading their quality of life. It is caused by deleterious mutations to a protein known as the Cystic Fibrosis Transmembrane conductance Regulator (CFTR). This protein acts as an anion channel.

Over the last decade there have been an increasing number of clinically approved small molecule drugs which act directly on CFTR in order to restore its function. These drugs are called CFTR modulators. Unfortunately, since CF is a rare disease and there are more than 400 mutations which cause it, it is currently unclear which mutations will respond to modulator therapy. This leaves many patients with under studied mutations unable to access modulator therapy.

In this work we performed extensive molecular dynamics (MD) and free energy calculations in order to characterise the many different ways that the CFTR can misfunction. This was done in close collaboration with \textit{in vitro} and clinical experiments in order to understand what types of molecular defects may be treated by existing medications. This work will help more patients access modulator therapy.

We found that rare CFTR mutations exhibit a wide range of molecular defects and that these defects appear to respond to these modulator drugs. The combination of MD, a basic biophysical technique with wet lab studies signals the increasing capability of quantitative physical techniques in biological research.
\end{abstract}
