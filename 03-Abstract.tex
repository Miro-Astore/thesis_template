%=======================================================================================%
\begin{abstract}
\setcounter{page}{3}
\thispagestyle{plain}

%\chapquote{Cells have a mind of their own} {-Shafagh Waters (personal communication)}
Sophisticated quantitative and physical thinking is revolutionising biology. The refinement of theoretical models in physical chemistry, alongside the growing capabilities computational engines and biochemical techniques are giving unprecedented insight into the structure and dynamics of living things. Here we will present a small example of this paradigm shift, by demonstrating how a physical mindset can contribute to the study of a disease---Cystic Fibrosis.

Cystic Fibrosis is the most common fatal genetic condition in Caucasians. It is a debilitating disease, significantly shortening the life span of patients and degrading their quality of life. It is caused by deleterious mutations to a protein known as the Cystic Fibrosis Transmembrane conductance Regulator (CFTR). This protein acts as an anion channel---in order to balance the levels of salts across epithelial membranes, it must conduct chloride and bicarbonate.

Over the last decade there have been an increasing number of clinically approved small molecule drugs, called CFTR modulators, which act directly on CFTR in order to restore its function. Unfortunately, since CF is a rare disease and there are more than 400 mutations which cause it, there is insufficient clinical data to determine which patients will respond to modulator therapy. This leaves many patients with rare mutations unable to access these life changing drugs.

In this work, we used extensive molecular dynamics (MD) and free energy calculations to study how CFTR works and characterise the numerous ways it can misfunction. This was done in close collaboration with \textit{in vitro} and clinical experiments, in order to understand what types of molecular defects may be treated by existing modulators with a process known as theratyping. 

Our findings indicate that each CFTR mutation is largely unique, with each one exhibiting its own set of molecular interactions which cause pathogenesis. What is then remarkable, is that these different mutations all appear to respond to CFTR modulators---both \textit{in vitro} and in some cases, in the clinic as well. This strongly indicates the possibility that a larger population of of those afflicted with rare forms CF may respond to the right choice of modulator therapy. This physical understanding of the underlying cause of CF leads us to suggest that patients carrying missense mutations should be systematically theratyped, and the choice of modulators could be informed by a molecular understanding of the misfunction in CFTR.

%In order to understand the effects of rare mutations, it was important to also shed light on the function of the healthy CFTR gene. Our simulations were able to add lots of missing information to the existing atomic structure of CFTR, such as adding a poorly resolved protein loop and predicting a conducting conformation of the CFTR protein. This further demonstrates that the complimentary power of MD  and free energy calculations to experimental biological techniques.

%In order to in order to direct further enquiries into the molecular studies of Cystic Fibrosis, we have formulated a physical model for the rescue of CFTR by small molecules. This combination of MD, a basic biophysical technique with wet lab studies signals the increasing capability of quantitative physical techniques in biological research. 

\end{abstract}
