\chapter{Resolving a Conducting Conformation of CFTR Using Free Energy Calculations}
\label{chap:opening}
\chapquote{You wanna fight?} {- Doctor Zachary Picker. Layperson. (personal communication)}

\section*{\centering Abstract} 
The misfunction of the CFTR gene causes Cystic Fibrosis. This protein conducts both chloride and bicarbonate. Understanding how CFTR conducts ions is critical to ongoing drug discovery efforts to treat Cystic Fibrosis. Existing structures of CFTR raise unresolved qustions as to how CFTR conducts ions as they exhibit a constriction smaller than the ions themselves. This indicates that there must be some level of conformational change for ions to pass through this structure. Here we present innovative simulation techniques combining principal component analysis of 8 microseconds of protein simulations and new advances in free energy calculations to resolve the full conduction pathway in human CFTR. We also propose experimental single ion channel electrophysiology techniques to experimetnally test whether this conformation is physiologically important.  

The presented study differs in an important way from existing studies which explore protein conformational landscape. Existing studies have previously investigated the conformational space \textit{between} solved protein conformations. The presented study demonstrates that with careful application of modelling it  is now possible to go \textit{beyond} the conformational neighbourhood of existing protein structures.

The findings of this study demonstrate that computational power and protein forcefields are now sufficiently developed to take the advances of the cryo-EM ``resolution revolution" and elucidate even more of the protein conformational landscape. This heralds a new era of computational structural biolgoy and biophysics.


\section{Introduction}

Ion channels are critical clinical targets. Their regulation of cellular membrane potential makes them attractive targets for many lcinical therapies. The mechanism and conditions under which they open and close is a closely studied topic due to the ability of small molecule drugs to regulate this transition and the pathology that a misfuncitonal channel can cause, either from mutation or environmental changes. Perhaps the most well-known channelopathie is the misfunction of the anion channel, the Cystic Fibrosis Transmembrane conudtance Regulator (CFTR) which causes Cystic Fibrosis (CF) \cite{riordan1989,gadsby2006}. 

Various mutations may cause this ion channel to misfunction but the advent of new modulators heralds a new era in the treatment of this life-limiting disease. Due to the importance of the conductance of this channel in disease pathology, a molecular understanding behind the basis of its conductivity will inform the rational design of novel therapeutics to better patient outcomes. Unfortunately, current protein structures of this protein are insufficiently dilated in order to conducta nions, leaving unresolved questions behind the basis fo its conductivity. 

The cahnnel exhibits a slectivity filter which must be both narrow enough to partially dehydrate chloride and also large enough to allow the passage of large anions such as bicarbonate and glutathione \cite{}. 

\subsection{The outer pore in the Cryo-EM structure of phosphorylated human CFTR is not sufficiently open to conduct anions}

\section{Results}


\subsection{Using metadynamics and long simulations we can dilate the pore to discover a conducting conformation}

\subsection{Proving the discovered conformation is capable of ion conduction with umbrella sampling}

\subsection{This open conformation appears to be stabilised by a salt bridge not present in the cryo-EM structure.}

\label{salt_bridge}
It appears as though the the proposed open structure of CFTR is stabilised by a salt bridge that only forms in the dilated conformation. This presents an opportunity to test the predictions of these simulations using electrophysiology. 

The propsed conformation is also largeyl consistent withi TM8 demonstrate that Y914 and Y917 are solvent exposed, pore lining helices, position 914 linked to position 102 and 334 when they were each replaced by cysteines. Interestingly, the distance in our proposed open structure is similar to that found in 6MSM, consistent with these experiments \cite{negoda2019}. 

Mutations at R334 also exhibit a gating defect, with infrequent transitions to the fully open state \cite{fuller2005, cui2013a}.

\section{Conclusion}
With innovative application of free energy techniques we have been able to study an open, conducting conformation of CFTR. The conducting state of the channel has critical importance to drug discovery efforts for potentiators. If a small molecule drug could be developed to favor this state it would be a highly effective drug, as potentiators have demonstrated life changing clinical efficacy.

\section{Methods Details}

The methodology used in this chapter is largely consistent with the other studies in this thesis with some key differnces. The system was also constructed by embedding the extended CFTR model into a POPC membrane and then the whole these biomolecules were immersed in a neutralising 0.15mol/L KCl solution. For all calculations in this chapter the C-terminus was trunkaed at amino acid 1450 in order to make the simluation box smaller and save computer time.

