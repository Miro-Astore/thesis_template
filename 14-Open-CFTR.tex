\chapter{Resolving a Conducting Conformation of CFTR Using Free Energy Calculations}
\label{chap:opening}
\chapquote{You wanna fight?} {- Doctor Zachary Picker. Layperson. (personal communication)}

\section*{\centering Abstract} 
The misfunction of the CFTR gene causes Cystic Fibrosis. This protein conducts both chloride and bicarbonate. Understanding how CFTR conducts ions is critical to ongoing drug discovery efforts to treat Cystic Fibrosis. Existing structures of CFTR raise unresolved qustions as to how CFTR conducts ions as they exhibit a constriction smaller than the ions themselves. This indicates that there must be some level of conformational change for ions to pass through this structure. Here we present innovative simulation techniques combining principal component analysis of 8 microseconds of protein simulations and new advances in free energy calculations to resolve the full conduction pathway in human CFTR. We also propose experimental single ion channel electrophysiology techniques to experimetnally test whether this conformation is physiologically important.  

The findings of this study demonstrate that computational power and protein forcefields are now sufficiently developed to take experimental protein structures as a starting point. We can use them to explore the conformational neighbourhood around a given structure to discover more physiologically relevant protein conformations. 


\section{Introduction}

\section{Results}

\subsection{The outer pore in the CryoEM structure of phosphorylated human CFTR is not sufficiently open to conduct anions}

\subsection{Using metadynamics and long simulations we can dilate the pore to discover a conducting conformation}

\subsection{Proving the discovered conformation is capable of ion conduction with umbrella sampling}

\subsection{This open conformation gives rise to a novel salt bridge}
\label{salt_bridge}

\subsection{This open state can be used to study disease causing mutations in the outer pore such as R334W}
Mutagenesis studies of the R334 amino acid noticed that many different mutations appear to result in ephys readings which would indicate a loss of function, including mutation to lysine (K) which seems surprising \cite{ge2004, gong2004, linsdell2021}. One of the more common disease causing missense mutations is R334W. This motivated the use of umbrella sampling to test the energy landscape of the chloride permeation pathway in the presence of this mutation. 

\section{Discussion}

The present study diverges in an important way from existing \textit{in silico} investigations of protein conformational changes. Past studies have focussed on recreating intermediate free energy landscapes between \textit{known} endpoints \cite{lev2020, bergh2021}. Such approaches are critical to the development of molecular techniques in order to understand the energetics and kinetics of protein systems. However, these studies are inherently limited to the availability of high quality experimental 3d structures. By using structures as a starting point we can explore the conformational space around a region to understand more about a protein system. This has been made possible by increasing the increasing accuracy of potein forcefields to reproduce structural ensembles \cite{huang2016} and considerable increases to computer power. The present study is akin to the development of an unsupervised vs. supervised machine learning algorithms. Each approach is powerful but has its own domain of applicability and drawbacks. 

I predict that as free energy calculations are increasingly used to study protein systems we will see a delineation between \textit{untargetted} and \textit{targetted} MD methods, similar to how we see a delineation between supervised and unsupervised machine learning techniques.

As shown by the results in the present study, the difficulty in converging a free energy landscape with collective variables derived \textit {ab initio} from long classical MD simulations can be difficult as the CVs are very likely going to be suboptimal. Machine learning techniques, more sophisticated than the simple PCA algorithm used here would likely do a much better job of choosing quickly converging CVs \cite{}. 

The conformational changes investigated in this study occur within the lipid bilayer. This means that the kinetics and energetics of the transitions we have discovered will be highly dependent on the composition of lipids of the epithelium. It is well understood that the bilayer composition plays an important role in CFTR regulation and the clinical implications of this are an active area of research \cite{cui2020, cottrill2020}. It would therefore be an interesting study to repeat similar free energy calculations with different bilayer compositions to understand how they might regulate such conformational changes.

Understanding the open structure of CFTR has important implications for the drug discovery efforts to treat Cystic Fibrosis and sheds light on other important clinical questions. Specificlaly, selectivity of bicarbonate has been found to play an important role in pancreatic sufficiency of patients. The elucidation of basic CFTR function using simulations heralds an exciting new era of Cystic Fibrosis research. We are performing molecular medicine with atomic precision. 

The predictions of a stable salt bridge in section \ref{salt_bridge} fill a recent gap in the literature. The elegant study on the R117H mutation from Simon and Csnady's group \cite{simon2021}  discovered that a long standing conclusion that R117 made a connection with E1126 was incorrect and in fact R117 makes a stable hydrogen bond with E1124. This study did not closely investigate the role of E1126, observing that the E1126P mutant had slower closing kinetics but was unable to definitely explain why, suggesting that ECL4 might move slower in this mutation. This leaves the partner of E1126 unknown. One study investigating the blockage of CFTR by zinc postulated at an interaction between R334 and E1126 \cite{wang2016}. Here, the researchers tested the inhibition of chloride conduction in the presence of zinc in R334C-CFTR. They found strong evidence that R334C-CFTR was blocked by Zinc ions, as no current was recorded in the presence of zinc. Because zinc has a +2 charge they suspected that a nearby negative amino acid might might play a role in binding the zinc cation. Subsequent experiments They found that the mutant R334C/E1126A-CFTR was no longer inhibited by zinc ions. This is consistent with our findings that R334 and E1126 may indeed form a salt bridge, coming closer together in the conducting conformation compared to how far they are in the cryoEM structure.  

Previously it would have been very difficult to discover this interaction experimentally because R334 has plays such an important role in the conductance and selectivity of the channel. With the use of the atomic resolution offered by MD simulations we have been able to fill this gap in the experimental literature, demonstrating the power of \textit {in silico} methods for studying protein dynamics. We propose that single ion channel patch clamp experiments could demonstrate teh importance of teh E1126-R334 salt bridge to the open state of the channel. 

Similar to how this study was motivated by the lack of a structural picture of CFTR's conducting state, there are many other places in the proteome which would likely be amenable to a similar mode of investigation. For example, the important ABC transporter P-glycoprotein has been resolved in an outward facing occluded state, but the important outward facing conformation remains unimaged \cite{}. A similar technique to the one used here could likely resolve such a conformation \cite{kim2018a}.

\textit {in vitro} studies of the state dependent formation of the extracellular helices of the distances between  TM1, TM6, and TM12 indicated that in the closed state these helices are tightly bound together (as can be seen in the 6MSM structure) and in the open state they move apart \cite{negoda2018}. Additionally, similar studies of TM8 demonstrate that Y914 and Y917 are solvent exposed, pore lining helices, position 914 linked to position 102 and 334 when they were each replaced by cysteines. Interestingly, the distance in our proposed open structure is similar to that found in 6MSM, consistent with these experiments \cite{negoda2019}. 

Mutations at R334 also exhibit a gating defect, with infrequent transitions to the fully open state \cite{fuller2005, cui2013a}.

\section{Conclusion}
With innovative application of free energy techniques we have been able to study an open, conducting conformation of CFTR. The conducting state of the channel has critical importance to drug discovery efforts for potentiators. If a small molecule drug could be developed to favor this state it would be a highly effective drug, as potentiators have demonstrated life changing clinical efficacy.

\section{Methods Details}
