%=======================================================================================%
\chapter{Concluding Remarks}
\label{chap:conc}
\chapquote{``There is no real ending. It’s just the place where you stop 
the story."}{Frank Herbert}

\section{Summary of Research}
The use of atomistic computer models provides a valuable tool to probe systems at the 
molecular level that cannot be reached by current experimental methods. The rise of and 
continued improvements to computer technology, both hardware and software, is paving the 
way for research avenues that seemed impossible only a few decades ago. With current 
hardware technology, it is now possible to perform advanced free energy calculations to 
predict quantities that can be compared to experiments, which is one of the goals of 
computer simulations. This thesis presents the investigation of using free energy methods 
to explain observations in biological molecules.

The first three results chapters (4-6) investigate the interaction of ligands in glutamate 
transporters. Due to the availability of the crystal structure, the focus is on the archaeal 
homologue. The first study investigates the apparent discrepancy between experiment and 
simulation of the Na2 site. \Na\ ion is observed in simulations to be unstable in the Na2 
site and always leaves the binding site. The root cause of this issue is the undercharging 
of the S atom in the methionine residue (\GltPh:M311 and \GltTk:M314). Increasing the 
partial charge on the S atom by a factor of three from the default value brings the ion 
closer to the binding site. Calculation of the binding free energy also agrees with the 
experimental value, which was previously estimated as a positive value. The work done in 
this project reconcile the discrepancy between simulation and experiment for the Na2 site. 
In addition, the methionine residue is conserved throughout the SLC1 family, and methionine 
mutations are involved in diseases like Alzheimer's disease. Thus this study highlights the 
importance of having a properly parametrised force field.

The second study of glutamate transporters focuses on the ligand-binding mechanism before 
the closing of the HP1-HP2 gate and the binding of the second ion at Na2. Previous binding 
free energy calculations indicated high-affinity binding for the second \Na\ ion and 
low-affinity binding for Asp. Experimental results published after these computational 
studies show the opposite behaviour. In this second study, other potential ion binding 
states are searched. Starting with an intermediate site called Na1\prim, observed in 
previous computational studies, the possibility of a second \Na\ binding to \GltPh\ at this 
state is probed. Remarkably the \Na\ ion binds to the D405 residue that was initially bound 
to Na1\prim. The two ions are each bound to an aspartate residue (D405 and D312) with a 
separation distance of 3.5~\angs. The binding free energy of the outermost ion in this state 
(Na1--Na1\prim) is less than the value in the previous state (Na1--Na3), and this value agrees 
with experimental results. In addition, Asp also binds to \GltPh\ with the Na1--Na1\prim\ ion 
state with a binding free energy in good agreement with the experimental value, resolving the 
discrepancy in previous calculations. This suggests that the substrate binds to the protein 
with two ions bound at the Na1--Na1\prim\ state, which is not a state known before in glutamate 
transporters.

Previous studies established that the Na3 site is the deepest binding site and hypothesised 
as the final ligand to leave the protein. The final work in \GltPh\ reported in 
\chapref{chap:unbind} attempts to estimate the release time of \Na\ from the Na3 site to the 
bulk. Previous calculations employed path-independent free energy methods; however, recent 
experimental observations indicate ligand binding triggers conformational changes in the 
protein. Thus a path-dependent method is used to capture the effect of protein conformation. 
The PMF profile for Na3 $\rightarrow$ Na1\prim\ reveals a 17 kcal/mol energy barrier. Along 
the transition, significant conformational changes are observed in the NMDGT motif primarily 
in the N310 and D312 residues. Inspection of the umbrella windows reveals that D312 plays a 
chaperoning role for \Na\ while N310 behaves like a ratchet that prevents \Na\ from falling 
back to the Na3 site. The transition time estimated with the Smoluchowski equation is around 
7 sec, which is about 4\% of the total turnover time of \GltPh\ at 3 min. Two other transitions 
take place after this before \Na\ reaches the bulk environment, and the transition times are 
significantly faster. This study reveals that the release of the last \Na\ is definitely a slow 
process, but it is not the rate-limiting step in the transport cycle. In addition, this finding 
may explain the need for a \K\ ion to speed the process in the mammalian variants.

The first side project in \chapref{chap:gA} attempts to investigate the cause for the difference 
in ion conductance of gA embedded in different lipids as observed in experiments. Indeed, the 
PMF profile obtained from umbrella sampling indicates a difference in energy barrier in \K\ 
ion permeation in gA between POPC and NODS lipids. A breakdown of the potential energy reveals 
significant differences in the interactions of the \K\ ion with the lipid head group and water 
molecules at the interface. The phosphate head group in POPC has a stronger dipole moment 
compared to NODS, and this orients water molecules at the lipid-water interface. The orientation 
of the water molecules at the interface produces the larger energy barrier in the PMF and 
explains the apparent impedance in ion conductance.

The last project reported in \chapref{chap:ions} deals with the simplest system investigated, 
the solvation of ions and small molecules. The purpose of this project is to compare results 
for solvation free energy using two different boundary conditions; PBC and SBC. The 
investigation found that the energies converge to a unique value above a critical volume, 
which is similar in both systems. In particular, for SBC this means that the boundary effects 
(water polarisation at the liquid-vacuum boundary) become negligible. This was a problem back 
in the 1990s as the hardware technology limits the calculations to very small systems (radius 
of 8--12~\angs). With small systems, the polarisation of water molecules at the boundary affects 
the final solvation free energy, and thus angular restraints are applied, creating a buffer zone. 
Increasing the radius to $>$21~\angs\ reduces the boundary effects but increases the computation 
time. Running these calculations on the GPUs, however, removes this limitation and allows for 
the calculations of larger systems. In addition to the solvation of ions, the solvation of amino 
acid side chain analogues reveals problems with the current force field parameters. The results 
show an overestimation of the energies for the negative amino acids by about 9 kcal/mol. This 
explains the apparent stickiness between aspartate/glutamate with arginine during peptide-protein 
dissociation.

\section{Potential Future Research}
For glutamate transporters, there are a number of avenues for future work. As noted in 
\chapref{chap:bind} it is not possible to estimate the transition time from Na1\prim\ to 
Na3 in the presence of Asp and Na1 with the MD/FEP. A potential project would be to use 
a path-dependent method to obtain the PMF profile between the two states. One difficulty 
with this, however, is the flipping of N310 and conformational changes in the protein, 
which requires the construction of a multidimensional PMF. Although this increases the 
computation time of the calculations, future generation hardware technology will definitely 
make it easier. In addition to ligand binding, the transition time for the translocation 
of the transport domain from the outward to inward conformation is an important avenue to 
research. An estimate of such transition time may reveal the rate-limiting step in the 
transport cycle. One hurdle with calculating this is the choice of collective variable (or 
reaction coordinate) as a simple distance with respect to the membrane normal may not be 
adequate. Recently, a new method for choosing appropriate collective variables with machine 
learning is reported~\cite{Sultan2018}. In this method, the algorithm studies the endpoints 
between transitions and identifies suitable collective variables by supervised learning. 
More than one collective variable might be needed to describe the motion. A dimensionality 
reduction representation can be used in this case with either the string method 
\cite{Gan2009,Maragliano2006} or the path-collective variables (PCV)~\cite{Branduardi2007}. 
The string method coupled with quaternion based collective variables has been used to describe 
the transition of glycerol-3-phosphate: phosphate antiporter (GlpT) from the OF to IF 
conformation~\cite{Moradi2015}. Thus estimating the transition energy from the OF to IF in 
\GltPh\ may not be an impossible task. Given the PMF, it is also possible to estimate the 
diffusion profile more accurately with a Bayesian estimator based on the ABF 
method~\cite{Comer2013a}. A comparative study with the EAATs will be an appropriate study 
to explain the faster turnover rate of EAATs potentially. The recent crystal structure of an 
engineered EAAT1 protein is a potential candidate for the template of homology 
models~\cite{Canul-Tec2017}. As mentioned previously in the conclusion of \chapref{chap:bind}, 
the two-ion intermediate state may be the key to the \K-\Na\ exchange process, which may 
explain the faster turnover rate. It may also be possible to use the current methods used in 
this thesis to study the effect of certain mutations in neurological diseases. For example, 
if a certain mutation in EAATs is linked to a neurological disease, we can study the effect 
of the mutation to ligand binding. If the mutation disrupts the ligand binding free energy, 
then the signals in the nervous system are impaired. The predictions from simulations can be 
compared and verified with physiological experiments. This is just one way to study the impact 
of mutations on certain neurological diseases. Also, once the effect of the mutation is 
defined, it is possible to use this as a target for drug design (can be studied using 
the free energy methods used in this thesis).

One of the significant problems with computational studies of gA is the lack of explicit 
polarisation in classical force fields. A recent computational study with the polarisable 
AMOEBA force field reduces the barrier to about 5 kcal/mol~\cite{Peng2016}. This improvement 
comes from an increase in the dipole moment of the neighbouring water molecules from 2.59 D 
to 2.70 D. However, QM/MM simulations indicate that the dipole moment can vary from 1.80 to 
As mentioned previously2.85 D~\cite{Timko2012}. Although the AMOEBA force field improves the 
PMF and ultimately the conductance, the value is still about 30\% off. Thus, further study 
can be done in investigating the effect of polarisation in gA, and the extension to divalent 
ions like \Ca\ can be studied with the AMOEBA force fields. Besides parametrisation, gA can 
be used as a benchmark system to compare the current polarisable force field models (the fluctuating 
charge, Drude oscillator and induced polarisation) or potential future models  that can determine 
the robustness of the models.

A major issue with using SBC for MD simulation is the internal pressure that arises due 
to the vacuum that is not investigated in this thesis. The water molecules minimise its 
surface area, much like a soap bubble in the air. Many authors have attempted to fix this issue 
by applying an empirical potential that pushes water molecules out of the 
centre~\cite{Essex1995,Marelius1999,Deng2004}. However, these potentials need to be 
parametrised for specific droplet sizes. Thus one possible future project is to investigate 
the pressure as a function of droplet radius. An analytical correction to the pressure may 
then be constructed based on this study. Another approach to correct for the pressure is to 
apply a potential on the water molecules that is independent of the droplet size. This will 
require testing on different potential functions that can recover the radial density of water 
profile to that of the experimental value. In addition to the pressure issue in SBC, the 
negatively charged amino acids in the CHARMM force field overestimates the solvation energies 
compared to the experimental values from Kelly et al.~\cite{Kelly2006}. This has a crucial 
implication for not only CHARMM but for other force fields. Taking acetate as an example, 
the CHARMM force field applies NBFix that replaces the normal LJ values with different 
values between pairs to match the osmotic pressure from experimental data~\cite{Venable2013}. 
It might be possible to obtain parameters without NBFix by first calculating the solvation 
free energy with different parameters for the O atom ($q$, $\epsilon$, R$_{\text{min}}$). 
Since there may be more than one minimum in the parameter set from the solvation energy, 
the global minimum must be chosen by fitting the osmotic pressure for \Na\ or \K\ to 
experimental data \cite{Robinson2002}. Data for arginine and lysine side-chain analogues 
also exists. Thus it might be possible to obtain a self-consistent parameter set without 
the need for NBFix. This may resolve the issue with the stickiness observed in peptide-protein 
interactions, and the work can be extended to polarisable force fields.
%=======================================================================================%