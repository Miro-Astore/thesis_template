%=======================================================================================%
% the main points of this introduction are 
% outline the complexity of biological systems for physicists
% 	> even though our theoretical models should preidct everything on the energy and length scales of biology we can't because of their heterogeneity.
%   > Give examples of the heterogenaity 
% Give some points on the history of molecular biophysics  
%   >  Hodgkin-Huxley Models
%   > Gramicidin 
% Point out how Cystic Fibrosis is an expression of this progression, going from genotype to phenotype using an ion channel to teach us biophysics. 
% conclusion.
\pagenumbering{arabic}
\chapter*{Foreword}
\setcounter{page}{1}
\label{chap:foreward}
\chapquote{} {}
\vspace
The past few years I have been captivated by the fabulous complexity exhibited by biological systems. The mindset for solving biological problems feels very different to the focus we cultivate in students when they study idealised problems in mathematics and physics. The problems are broader, and many hands are needed to solve them. Note that all the publications arising from this thesis have many many authors. Each researcher specialises like their cells.

The more I wrote this thesis the more I found myself writing for my past self, so I think this thesis will be best read by my future students. A second year grasp of physics should be sufficient to digest all of the contents herein, as the tone is quite pedagogical and the mathematics light. What will be less familiar to these students is the breadth of pre requisits to understand teh contents. What I have not had time to do is write an introduction to molecular biology, so there may be much chemistry missed by my students.

An extensive review of the literature in the introductory chapter \ref{chap:introduction} and \ref{chap:methods} will hopefully serve as a road map for my future students. The issue is that the field is now progressing so quickly that I'm sure much of this thesis will be out of date by the time I've given it to a student to read. In a rapidly evolving field like this I think my advice would be to seek out members of the biophysics community and continue to collaborate

