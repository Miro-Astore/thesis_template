%=======================================================================================%
\chapter{Effect of Lipid Membranes on \K\ Permeation in gA}
\label{chap:gA}
ABSTRACT \newline

Membrane proteins are embedded in a lipid bilayer and interact with the lipid molecules 
in subtle ways. This can be studied experimentally by examining the effect of different 
lipid bilayers on the function of membrane proteins. Understanding the causes of the 
functional effects of lipids is difficult to dissect experimentally but more amenable to 
a computational approach. Here we perform molecular dynamics simulations and free energy 
calculations to study the effect of two lipid types (POPC and NODS) on the conductance 
of the gramicidin A (gA) channel. A larger energy barrier is found for the \K\ potential 
of mean force in gA embedded in POPC compared to that in NODS, which is consistent with 
the enhanced experimental conductance of cations in gA embedded in NODS. Further analysis 
of the contributions to the potential energy of \K\ reveals that gA and water molecules in 
gA make similar contributions in both bilayers but there are significant differences 
between the two bilayers when the lipid molecules and interfacial waters are considered. 
It is shown that the stronger dipole moments of the POPC head groups create a thicker 
layer of interfacial waters with better orientation, which ultimately is responsible for 
the larger energy barrier in the \K\ PMF in POPC.

\newpage
\section{Introduction}
Gramicidin A (gA) is an antibiotic peptide that disrupts bacteria by allowing unimpeded flow of 
cations into the plasma membrane. This small hydrophobic peptide is composed of two identical 
helical subunits with 16 residues on each. The two subunits form a stable compound only inside a 
lipid bilayer environment. When stable, the gA dimer forms a narrow cylindrical hole across the 
bilayer, through which water and monovalent cations can permeate near diffusion rates. The channel 
structure was determined using solution~\cite{Arseniev1985} and solid-state NMR~\cite{Ketchem1993}, 
and consists of a single-stranded, right-handed $\beta$-helical dimer. Each subunit is made up of 
formyl-VGA\underline{L}A\underline{V}V\underline{V}W\underline{L}W\underline{L}W\underline{L}W-ethanolamine 
(underlined residues indicate D-amino acids and L-amino acids otherwise)~\cite{Sarges1965}. The 
alternating L-D amino acid sequence allows the peptide to fold into a helix with the side chains 
aligned on the exterior of the helix~\cite{RamachandranGNChandrasekaran1972}. There are two 
high-resolution structures available, PDB:1MAG~\cite{Ketchem1996} and PDB:1JNO~\cite{Townsley2001}. 
These two structures have been used in a large number of computational studies, investigating the 
ion permeation properties of gA~\cite{Allen2003,Allen2004,Allen2006,Bastug2006c}. Due to its simple 
structure, gA has often been used as a model for membrane proteins. For example, it was used as a 
prototype ion channel model long before the first potassium ion channel protein was 
crystallised~\cite{RouxBenoitandKarplus1994,Partenskii1992}. It has also been used as a testing 
model for developing and validating computational methods from continuum theories to 
\textit{ab initio} molecular dynamics (MD)~\cite{Edwards2002,Allen2003,Bastug2006a,Bastug2007a,
Timko2012}.

Biological membranes play an important role in cell biology. They consist of lipid molecules
organised in a bilayer formation, leaving a $\sim$30 \angs\ hydrophobic layer between the 
intracellular and extracellular environment of cells~\cite{VanMeer2008a}. Membrane proteins 
embedded in lipid bilayers may have their functions modulated by the surrounding environment as 
shown in some studies~\cite{Lee2005a,JensenMortenandMouritsen2004,Elmore2003a,Valiyaveetil2002}. 
Because gA is a small protein, the effects of lipids on its function is expected to be greater 
than those of larger membrane proteins. Hence, gA is a good candidate to investigate the effects 
of lipid environment on protein function.

The effect of membrane environment on gA function have been investigated in several studies
previously~\cite{DeGodoy2011,Wyatt2009,Qin2007,Cukierman2000,Dreyer2013}. Of particular interest 
is the work of Cukierman and co-workers, where gA is embedded in four different lipid environments; 
monoglycerides, ceramides, sphingolipids and phospholipids~\cite{DeGodoy2011}. The experiments 
showed a markedly decreased ion conductance for gA in a phospholipid environment compared to 
ceramides. On the other hand, the proton conductance exhibits the opposite behaviour with 
phospholipids enhancing conductance~\cite{Wyatt2009}. Enhancement in proton transport is attributed 
to the orientation of water molecules at the membrane-water interface due to the polar phosphate 
head groups~\cite{Qin2007}. Proton transport involves jumps between hydronium and water molecules 
connected through hydrogen-bonds~\cite{Cukierman2000}. Alignment of water molecules at the interface, 
due to the membrane dipole potential~\cite{Dreyer2013}, reduces entropy, thereby decreasing the energy 
barrier for proton transport. Ion transport, however, proceeds through hydrodynamic diffusion. The 
role of the phosphate head groups and the aligned water molecules at the lipid interface in ion 
transport across gA is still not clear.

In this article, we attempt to explain the origin of the effects of the lipid environment on ion 
conductance of gA channel. We test two different lipid molecules, comparing phospholipids (POPC) 
and ceramides (NODS) (\figref{gA:fig1}). This is accomplished using MD simulations and free 
energy calculations. Our results show that ion conductance is indeed enhanced by ceramide lipids 
consistent with the experimental observations. The reduction in conductance for phospholipids 
comes from two sources, the phosphate head groups and the orientation of water molecules in the 
interface layer.

\begin{figure}[t!]
\centering
\includegraphics[width=8.5cm]{Figures/Gramicidin/fig1.jpg}
\caption{Molecular structure of phospholipids (POPC) (left) and ceramides (NODS) 
         (right) lipid molecules. Hydrogen atoms are not shown for clarity.}
\label{gA:fig1}
\end{figure}

\section{Method}
\subsection{Model System}
The two lipid molecules considered are 1-palmitoyl-2-oleoyl-sn-glycero-3-phosphatidyl choline 
(POPC) and N-oleoyl-D-erythrosphingosine or ceramide C18 (NODS). The structure of both molecules 
are generated using the Avogadro software~\cite{Hanwell2012}. For POPC, we use the standard 
topology and parameters available in CHARMM36~\cite{Brooks2009}. The topology for NODS is generated 
from the ParamChem server~\cite{Vanommeslaeghe2012a,Vanommeslaeghe2012} and is derived from 
sphingomyelin parameters available in CHARMM36~\cite{Venable2014}. One dihedral parameter with a 
penalty higher than 10 is optimised using the force-field toolkit (FFTK)~\cite{Mayne2013} available 
in VMD~\cite{Humphrey1996}. The target data for optimisation are generated using Gaussian 
09~\cite{Frisch2009}.

The 1NJO structure~\cite{Townsley2001} is used for the gA dimer embedded in a lipid bilayer. The gA 
dimer isplaced in a hexagonal cell with 20 lipid molecules placed around the peptide per layer. We 
solvate the gA–membrane complex with TIP3P water molecules and neutralise the system with 0.15 M 
of KCl. The system is equilibrated in two stages to ensure the stability of the peptide-membrane 
complex. First, the gA atoms are fixed, and the cell is allowed to fluctuate isotropically until the 
correct lipid and water densities are obtained. The cell in the $x$-$y$ plane is then fixed, and only 
fluctuations in the $z$-direction are allowed. In the second stage, the gA atoms are gradually relaxed 
over 3 ns of equilibration. The system is further equilibrated without any restraints for 10 ns. We 
have performed the production runs without any restraints on the gA atoms because the flexibility of 
the protein affects ion permeation as demonstrated in a previous work~\cite{Bastug2006a}. To keep 
the protein stable inside the bilayer, we apply harmonic restraints on the orientation and position 
of the centre of mass of the protein. The restraint on the centre of mass does not affect the 
internal degrees of freedom of the protein.

\subsection{PMF Calculations}
We calculate the potential of mean force (PMF) of potassium ion along the gA channel using the
equilibrium umbrella sampling simulations~\cite{Torrie1977} with the weighted histogram analysis 
method~\cite{Kumar1992}. We use the same protocols as in an earlier work~\cite{Bastug2007} for 
configuring the calculations, which are briefly described here. The reaction coordinate for the 
ion is its distance from the gA centre along the channel axis. We perform the PMF calculations in 
the range [0, 20] \angs, assuming symmetric behaviour between the two monomers. For umbrella sampling 
calculations, we use a total of 41 windows at 0.5 \angs\ intervals. The umbrella windows are 
generated via steered MD simulations, starting from the equilibrated system with the \K\ ion at the 
centre of gA. We apply a harmonic potential with a force constant of 10 kcal/mol, which is reduced 
to 7 kcal/mol outside the pore ($z > 15$ \angs) to improve sampling between windows. In the bulk 
region, a harmonic potential of 1 kcal/mol is applied in the radial direction to prevent the ion 
drifting away from the central axis. Each window is run for 3 ns with 1 ns for equilibration, giving 
a total simulation time of 123 ns.

\subsection{MD Simulation}
All MD simulations in this work are performed with the NAMD package (version 2.10)~\cite{Phillips2005} 
with the CHARMM force field~\cite{Brooks2009}. We employ the NPT ensemble and kept the simulation 
temperature constant at 300 K using the Langevin thermostat with a damping factor of 1 ps$^{-1}$. 
The pressure is kept constant at 1 atm using the Langevin piston method with a damping factor of 
50 ps$^{-1}$~\cite{Feller1995}. Periodic boundary conditions are used, and electrostatic interactions 
are calculated using the particle-mesh Ewald (PME) method~\cite{Darden1993} without truncation. 
Non-bonded interactions are truncated at 12 \angs\ and replaced with a smooth switching function 
starting from 10 \angs. In all simulations, a time step of 2 fs is employed for the integrator.

\section{Results and Discussion}
\subsection{Membrane Bilayer}
\begin{figure}[b!]
\centering
\includegraphics[width=0.75\textwidth]{Figures/Gramicidin/fig2.jpg}
\caption{Analysis of the pure bilayer systems: (a) mass density of head groups; (b) water number 
density; and (c) average orientation of water molecules with respect to the bilayer plane (the 
angle between the dipole vector and the $x$-$y$ plane). Black and red lines represent systems with 
POPC and NODS lipids, respectively. Densities are calculated using the density profile extension 
in VMD~\cite{Giorgino2014a}.}
\label{gA:fig2}
\end{figure}
Before investigating ion conductance in gA we studied the behaviour of the lipid bilayers with
POPC and NODS. The molecular structures of POPC and NODS are shown in \figref{gA:fig1}. For the 
purpose of investigating pure bilayers, we used a simulation system consisting of 64 lipid molecules 
per leaflet. To compare the thickness, we calculated the mass density of the head groups for each 
lipid type and plotted the densities in \figref{gA:fig2}. For POPC, we chose the phosphorus atom 
as the reference and the oxygen on the hydroxyl group in NODS. From 4 ns of MD simulations, we 
sampled the location of the head groups along the bilayer normal ($z$-axis). From the maximums, 
it can be seen that the oxygen atoms are located slightly further away from the bilayer centre 
compared to the phosphorus atoms (21.0 and 20.3 \angs, respectively). Thus, the average bilayer 
thickness is about 41 \angs\ and 42 \angs\ for POPC and NODS, respectively, in agreement with 
experiments and simulations~\cite{Dutagaci2014a,Kucerka2011}. It has been shown that the asymmetric 
hydrocarbon chains of ceramide results in a thicker bilayer formation~\cite{Carrer}. We observed a 
larger difference in the horizontal densities (i.e., area per lipid) of POPC and NODS lipids. For 
POPC, we found an area of $\sim$60 \angs$^2$ per lipid in good agreement with the experimental 
values~\cite{Jurkiewicz2012a}. For NODS bilayer, the area per lipid we obtained is 
$\sim$46 \angs$^2$, again very similar to the experimental value~\cite{Paloncyova2015}. These give 
confidence to the force field parameters used in the simulations. The large difference between POPC 
and NODS densities is simply due to the larger head group in POPC. We also plotted the water density 
along the bilayer normal in \hyperref[gA:fig2]{Fig.~\ref{gA:fig2}}. The water density profiles show 
a thicker layer of water molecules at the interface for POPC than NODS. This behaviour is a direct 
result of the tighter packing of NODS molecules in the bilayer formation.

To further analyse the previous observation, we plotted the average orientation of water molecules 
as a function of the bilayer normal. The orientation is defined as the angle between the dipole 
moment of water molecules and the bilayer plane. The angles are averaged over a 2 ns trajectory 
collected in 0.5 \angs\ bins. The peak of the profiles shown in \figref{gA:fig2} is roughly 
20$^{\circ}$ and 10$^{\circ}$ for POPC and NODS, respectively. Qin et al. obtained a peak 
close to 30$^{\circ}$ for a DiPhPC bilayer~\cite{Qin2007}. The difference between our results and 
Ref.~\cite{Qin2007} for the PC head groups can be attributed to polarisation effects that is 
included semi-empirically in the multistate empirical valence bond model~\cite{Schmitt1998}. However, 
even without polarisation, our results clearly show that the PC head groups orient water molecules 
over a wider range than for ceramide. Thus the larger head groups in POPC not only increases the 
area per lipid but also results in a greater dipole potential. It is interesting to compare the 
water orientation with GMO bilayers~\cite{Qin2007}, where water molecules exhibit close to a random 
orientation on average at the interface. Experimental measurements of caesium conductance through 
gA embedded in GMO bilayers show very similar behaviour to NODS at concentrations greater than 250 
mM~\cite{DeGodoy2011}. The orientation of water molecules at the interface may not be the only 
factor contributing to the enhanced conductance.

\subsection{gA Embedded in Membrane Bilayer}

\begin{figure}[b!]
 \centering
 \includegraphics[width=1.0\textwidth]{Figures/Gramicidin/fig3.jpg}
\caption{RMSD of gA embedded in POPC and NODS lipids: (a) total and (b) average per residue. 
Error bars are not shown for RMSD per residue for clarity.}
\label{gA:fig3}
\end{figure}

After embedding the gA in bilayers, we have simulated the equilibrated system for 10 ns with only 
water molecules inside. The resulting RMSD, calculated by excluding the ethanolamine groups, is 
shown in \figref{gA:fig3}. The RMSD fluctuates around 0.5--0.8 \angs, and there are some 
slight changes in RMSD over the trajectory, but there are no perceptible differences between the 
RMSDs of gA embedded in POPC and NODS. When we include ethanolamine in the calculations, the RMSD 
time series fluctuates closer to 1 \angs. To show this effect quantitatively, we plot the average 
RMSD per residue, including the ethanolamine, which is also shown in \figref{gA:fig3}. For POPC, 
all the residues have RMSD values less than 0.75 \angs. In NODS, however, the residues near the 
entrance of the channel fluctuate more than POPC because the gA side chains have much weaker 
interactions with the NODS head groups compared to those of POPC. In particular, the ethanolamine 
group swings instantaneously and this is also observed in MD simulations for gA with other 
bilayers~\cite{Kim2012}. The swing motion perturbs the neighbouring residues thereby explaining the 
slight turns in the RMSD time series even without ethanolamine. The swing motion of ethanolamine is 
due to loss of hydrogen-bonds~\cite{Ingolfsson2011}, and we circumvent this problem by sampling the 
systems longer. Despite these minor differences, gA is essentially stable inside the hydrophobic 
environment of both lipids. With gA embedded inside the membrane, we find that the bilayer thickness 
of the lipid molecules in the first shell decreases appreciably as observed by Kim et al. 
previously~\cite{Kim2012}. The bilayer thickness is recovered in the second lipid shell, and hence 
only lipids in the first shell are affected. Because we have embedded gA in 20 lipid molecules per 
leaflet configured in a hexagonal cell, we report only the first shell results here. The bilayer 
thickness, as defined in the previous subsection, decreases from 41 to 31 \angs\ for POPC, and from 
42 to 37 \angs\ for NODS. The larger decrease in the thickness of a POPC bilayer is again related to 
the much stronger coupling of the POPC head groups with the gA side chains compared to those of NODS.

\begin{figure}[t!]
 \centering
 \includegraphics[width=1.0\textwidth]{Figures/Gramicidin/fig4.jpg}
  \caption{Water molecules inside the channel with \K\ located at the centre of gA. (a) Structure 
           showing \K\ positioned in the centre of gA with aligned water molecules (lipids not shown); 
           and (b) average orientation of water molecules. The orientation is defined as the angle 
           between the dipole vector and the x-y-plane. Labels for water molecules are shown on the 
           diagram.}
\label{gA:fig4}
\end{figure}  

In the previous trajectory of the gA system with water molecules, we have replaced a water molecule 
near the centre of gA with a \K\ ion. We have applied a harmonic restraint to keep the \K\ ion at 
the centre and sampled the systems for a further 4 ns (the ethanolamine group is stable at this 
point). To gain further insights into the potential contributors to the ion conductance, we have 
analysed the behaviour of water molecules inside the channel. When the \K\ ion is positioned at 
the centre of gA, there are three water molecules aligned on either side of the ion inside the 
channel. Similar to the results in \figref{gA:fig2}, we have calculated the average orientation of 
the six water molecules in gA with respect to the bilayer plane (\figref{gA:fig4}). As expected, 
the alignment of the water dipoles with the channel axis, and hence the strength of the interaction 
between the ion and water molecules, decreases as a function of the distance from the ion. We note 
that the water dipoles are not aligned closer to 90$^{\circ}$ with the plane because they make 
hydrogen bonds with the carbonyl groups of gA. Taking into account the statistical fluctuations, 
the interaction between the \K\ ion and water molecules inside the channel are very similar for the 
POPC and NODS bilayers.

\subsection{Potential of Mean Force of \K\ in gA}
\begin{figure}[b!]
\centering
\includegraphics[width=0.65\textwidth]{Figures/Gramicidin/fig5.jpg}
\caption{Potential of mean force (PMF) profiles of a \K\ ion along the gA channel axis.}
\label{gA:fig5}
\end{figure}

We have performed umbrella sampling simulations to determine the PMFs of a \K\ ion along the 
central axis of gA embedded in the POPC and NODS bilayers. Details of the umbrella sampling 
simulations are described in the Methods section. Each umbrella window is simulated for 3 ns. 
Using the stability of the ethanolamine group and the convergence of the PMFs as criteria, 
the first 1 ns of data are discarded for equilibration, and the PMFs are constructed from the 
final 2 ns of production data. The PMFs obtained with the POPC and NODS bilayers are shown in 
\figref{gA:fig5}. The PMFs exhibit similar behaviour from the bulk region up to $\sim$7.5 \angs\ 
inside the channel but start diverging from there to the centre of gA. There is also some difference 
in the binding free energies at the binding pocket ($\sim$11.3~\angs). For the POPC bilayer, we 
obtain a well depth of 2.5 kcal/mol at the binding site with respect to the bulk. In previous PMF 
calculations for a \K\ ion, well depths in the range of 2--3 kcal/mol were obtained for PC
bilayers~\cite{Allen2003,Allen2006,Bastug2006c,Bastug2007}. As the accuracy of the PMF calculations 
is about 1 kcal/mol, the present result for the well depth is consistent with those earlier results. 
In the case of NODS, the well depth at the binding site is 3.7 kcal/mol, which is 1.2 kcal/mol deeper 
compared to the POPC bilayer. Thus we predict an eight-fold difference between the binding constants 
of \K\ ions for gA in POPC vs NODS bilayers, which can be easily distinguished in experiments.

The two PMFs start diverging around 7.5 \angs, and the difference becomes quite substantial near
the centre of gA. The energy barrier measured from the binding site to the peak in the PMF is 10.9 
kcal/mol for POPC and 8.2 kcal/mol for NODS. The barrier value for POPC is again in agreement with 
the previous PMF calculations in gA~\cite{Allen2003,Allen2006,Bastug2006c,Bastug2007}. The lower 
energy barrier observed in the NODS bilayer compared to that in the POPC bilayer is consistent with 
the experimental data which shows a four-fold increase in cation conductance of gA embedded in NODS 
compared to POPC~\cite{DeGodoy2011}. We stress that polarisation plays a significant role in ion 
permeation across the gA channel, hence MD simulations with a non-polarisable force field provide 
only qualitative results

It is of interest to find out how the change in the lipid bilayer affects the ion PMFs. To determine
the cause for this difference in the PMFs, we calculate the average potential energy acting on the 
\K\ ion from four different components of the system individually. The potential energy is calculated 
from the umbrella sampling trajectory data at 0.5 \angs\ intervals. The four components we chose to 
calculate are the protein, the lipid molecules, and the water molecules inside the channel and at the 
lipid interface. As shown in \figrefs{gA:fig2} and~\ref{gA:fig4}, these water molecules exhibit 
alignment with the channel axis or the bilayer normal, hence will contribute to the potential energy 
of the ion. The channel water molecules are defined as the water molecules inside the region 
[$-$10, 10]~\angs\ and interfacial water is defined by the range of molecules oriented by the lipid 
head groups as shown in \figref{gA:fig2}. The \K\ potential energies due to these four components 
are plotted as a function of gA channel axis in \figref{gA:fig6}. We note that the potential energy 
does not include entropy effects, hence we focus on qualitative rather than quantitative results.

We first consider the potential energy of the \K\ ion due to gA. As expected, this potential energy
approaches zero as the ion moves further into the bulk region. As the ion moves closer to the binding 
site the potential energy decreases to a minimum and then slowly increases as it reaches the centre. 
This trend is consistent with Allen et al.~\cite{Allen2004}, but the values we calculate are of a 
different scale because we consider the potential energy rather than integrating the mean force. 
For both POPC and NODS, the potential energy profiles are very similar, as evident from \figref{gA:fig6}. 
This result indicates that the lipid environment does not directly affect the interaction between gA 
and the ion inside the channel. Kim et al. investigated the behaviour of gA in different phospholipid 
environments and observed variations in the dynamics of gA~\cite{Kim2012}. From our results, we 
conjecture that although the lipid environment can perturb gA dynamics due to hydrophobic mismatch, 
these perturbations may not be significant enough to change the nature of the interaction between gA 
and \K.

Considering next the potential energy of the \K\ ion due to the lipid molecules, we see a noticeable 
difference between the POPC and NODS bilayers \figref{gA:fig6}. The potential energy due to NODS remains 
close to zero throughout, with a slight change from negative near the binding site to positive as the 
ion moves towards the bulk. This is mainly due to the interaction between \K\ and the NODS head groups, 
which have small dipole moments and a short-range. For POPC, however, the potential energy is quite 
substantial, which is the result of the strong dipole potential of the PC head groups. The potential 
energy remains negative within gA and approaches zero as the ion moves towards the bulk as expected. 
On average, the difference in energy between the two lipid bilayers is $\sim$20 kcal/mol within gA, 
and it favours POPC relative to NODS. This is clearly in the opposite direction to the PMF profiles, 
where the NODS PMF is lower than the POPC PMF (\figref{gA:fig5}). Thus other contributions to the \K\ 
potential energy are needed to explain the difference in the PMFs.

\begin{figure}[b!]
\centering
\includegraphics[width=0.85\textwidth]{Figures/Gramicidin/fig6.jpg}
\caption{Potential energy acting on \K\ as a function of ion positions in the gA channel. 
Interactions calculated include protein, lipid molecules, channel water and interfacial water.
The red and black line represents POPC and NODS lipid molecules respectively.}
\label{gA:fig6}
\end{figure}

The other components that could influence the potential energy of the \K\ ion in gA are water 
molecules. Water molecules can be separated into three groups based on their orientation in the 
system. Firstly, water molecules with random orientation are located in the bulk region, where 
interactions with protein and membrane are negligible. Next, as discussed earlier, a layer of 
water molecules tend to orient away from the bilayer due to membrane dipole potential. Finally, 
the single-file water molecules inside the channel align their dipole moments with the electric 
field of the ion, which is modulated by the formation of hydrogen bonds with the carbonyl oxygens 
of gA. Here we are interested only in water molecules with a specific orientation, and hence consider 
only the potential energy due to the channel and interfacial water molecules.

We start with the interaction of the channel waters with the \K\ ion. We have stated earlier that
the behaviour of the dipole moments of water molecules in gA, in the presence of a \K\ ion at the 
centre, is approximately the same in the NODS and POPC bilayers 
(\figref{gA:fig3}). This behaviour is seen to be maintained for other positions 
of the \K\ ion in gA$-$the potential energies of the \K\ ion due to the channel waters are seen to 
overlap well for the two lipid molecules (\hyperref[gA:fig6]{Fig.~\ref{gA:fig6}}). The small 
differences observed when the ion is outside gA is likely arising from the flipping of the dipole 
moments of water molecules. This verifies that the water molecules inside the channel do not directly 
contribute to the attenuation and enhancement of ion conductance in gA embedded in POPC and NODS 
bilayers, respectively. It is interesting to note the trend of the interaction inside the channel. 
There are three different stages in the range [0, 20]~\angs\ where the interaction changes. Starting 
from 20 \angs, the potential energy is on average close to zero as expected. As the ion is moved 
closer to the binding site, the potential decreases to $-$30 kcal/mol. This sudden change in 
potential is the result of one water molecule being pushed out of the channel on the other side to 
accommodate the \K\ ion. From the binding site to 7.5 \angs, the potential energy remains steady. 
However, as the ion is pushed further to the centre, there is a second dip in the potential energy, 
bringing it down to about $-$50 kcal/mol. We have observed that this results from the reorientation 
of water molecules inside the channel. The potential energy comes to a minimum at the centre with 
$-$60 kcal/mol, where the dipole moments of three water molecules are aligned in the direction of 
the ion's electric field on either side of the channel in a symmetric configuration. The potential 
energy profile demonstrates the stages that take place as the ion moves through gA, which cannot be 
traced from the PMF alone.

Lastly, we consider the interaction of \K\ with the interfacial water molecules, which exhibits a
significant difference in the potential energy of \K\ with the NODS and POPC bilayers. For NODS, the 
potential energy is zero at the centre and decreases to about $-$10 kcal/mol at the binding site. 
For POPC on the other hand, the potential energy is around $+$25 kcal/mol in gA and only starts to 
decrease around the binding site. As the ion is moved further away from gA, the potential energy of 
both systems converges to the same level. We note that the potential due to the interfacial waters 
will be strongly screened by bulk water, and the free energy of \K\ will vanish in bulk as seen in 
the \K\ PMF (\figref{gA:fig5}). To understand the difference between the POPC and NODS results, we 
refer to \figref{gA:fig2} which shows that there is a thicker layer of water molecules in POPC, 
better oriented by the PC head groups with larger dipole moments. This results in a much stronger 
dipole potential due to the interfacial waters when gA embedded in a POPC bilayer compared to that 
of NODS. Comparing the lipid contribution to the potential energy of \K\ with that of the interfacial 
waters (\figref{gA:fig6}), it is seen that the latter contribution more than compensates the former 
in POPC, resulting in a positive potential energy in gA. In contrast, the sum of the two contributions 
remain near zero but slightly negative in NODS. Thus our results indicate that the larger energy 
barrier in the POPC--PMF relative to the NODS--PMF is most likely due to the better orientation of 
the interfacial water molecules by the stronger dipole potential of the POPC head groups.

\section{Conclusion}
Our objective in this work is to understand the cause of the enhancement of ion conductance in
gA embedded in a NODS bilayer compared to that of POPC. For pure bilayer systems, we have observed 
different lipid–water interactions with the result that water molecules tend to be more structured 
in phospholipids than ceramide. This behaviour was also demonstrated in comparisons of phospholipids 
with monoglycerides~\cite{Qin2007}. Our simulations of the gA system show that the peptide behaves 
in a similar fashion when it is embedded in both lipid environments. However, the PMF of a \K\ ion 
in gA embedded in POPC results in a distinctly larger energy barrier compared to that of NODS, which 
is consistent with the observed enhancement of the conductance in NODS~\cite{DeGodoy2011}. To 
understand the origin of this enhancement, we have analysed the different contributions to the 
potential energy of the \K\ ion from gA, lipid molecules and water molecules within gA and at the 
lipid interface. We find that the interaction of gA and water molecules inside gA with \K\ are 
virtually identical for both lipid molecules. But there are substantial differences between the 
POPC and NODS bilayers when we consider the interaction of lipid molecules and interfacial waters 
with \K. In NODS, the weak dipole moments of the head groups result in loosely structured interfacial 
water, and the contribution from either group to the ion's potential energy is very 
small. In POPC, the strong dipole moments of the head groups give rise to a thicker layer of 
interfacial waters with better orientation. Thus, both groups make a substantial contribution to 
the ion's potential energy, but the positive contribution from the interfacial waters overcomes 
the negative one from lipids. This difference is consistent with the higher energy barrier found 
in the \K\ PMF in POPC compared to the \K\ PMF in NODS.
%=======================================================================================%