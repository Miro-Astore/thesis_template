%=======================================================================================%
\chapter{Molecular Dynamics and Theratyping in Airway and Gut Organoids Reveal R352Q-CFTR Conductance Defect}
\label{chap:r352q}
\chapquote{Cells have a mind of their own-Shafagh Waters} {}

\section*{\centering Abstract} 
A significant challenge to making targeted CFTR modulator therapies accessible to all individuals with cystic fibrosis (CF) are many mutations in the CFTR gene that can cause CF, most of which remain uncharacterized. Here, we characterized the structural and functional defects of the rare CFTR mutation R352Q – with potential role contributing to intrapore chloride ion permeation – in patient-derived cell models of the airway and gut. CFTR function in differentiated nasal epithelial cultures and matched intestinal organoids was assessed using ion transport assay and forskolin-induced swelling (FIS) assay respectively. CFTR potentiators (VX-770, GLPG1837 and VX-445) and correctors (VX-809, VX-445 +/- VX-661) were tested.  Data from R352Q-CFTR were compared to that of twenty participants with mutations with known impact on CFTR function. R352Q-CFTR has residual CFTR function which was restored to functional CFTR activity by CFTR potentiators but not the corrector. Molecular dynamics (MD) simulations of R352Q-CFTR were carried out which indicated the presence of a chloride conductance defect, with little evidence supporting a gating defect. The combination approach of in vitro patient-derived cell models and in silico MD simulations to characterize rare CFTR mutations can improve the specificity and sensitivity of modulator response predictions and aid in their translational use for CF precision medicine.
